% Options for packages loaded elsewhere
\PassOptionsToPackage{unicode}{hyperref}
\PassOptionsToPackage{hyphens}{url}
\documentclass[
]{book}
\usepackage{xcolor}
\usepackage{amsmath,amssymb}
\setcounter{secnumdepth}{-\maxdimen} % remove section numbering
\usepackage{iftex}
\ifPDFTeX
  \usepackage[T1]{fontenc}
  \usepackage[utf8]{inputenc}
  \usepackage{textcomp} % provide euro and other symbols
\else % if luatex or xetex
  \usepackage{unicode-math} % this also loads fontspec
  \defaultfontfeatures{Scale=MatchLowercase}
  \defaultfontfeatures[\rmfamily]{Ligatures=TeX,Scale=1}
\fi
\usepackage{lmodern}
\ifPDFTeX\else
  % xetex/luatex font selection
\fi
% Use upquote if available, for straight quotes in verbatim environments
\IfFileExists{upquote.sty}{\usepackage{upquote}}{}
\IfFileExists{microtype.sty}{% use microtype if available
  \usepackage[]{microtype}
  \UseMicrotypeSet[protrusion]{basicmath} % disable protrusion for tt fonts
}{}
\makeatletter
\@ifundefined{KOMAClassName}{% if non-KOMA class
  \IfFileExists{parskip.sty}{%
    \usepackage{parskip}
  }{% else
    \setlength{\parindent}{0pt}
    \setlength{\parskip}{6pt plus 2pt minus 1pt}}
}{% if KOMA class
  \KOMAoptions{parskip=half}}
\makeatother
\usepackage{graphicx}
\makeatletter
\newsavebox\pandoc@box
\newcommand*\pandocbounded[1]{% scales image to fit in text height/width
  \sbox\pandoc@box{#1}%
  \Gscale@div\@tempa{\textheight}{\dimexpr\ht\pandoc@box+\dp\pandoc@box\relax}%
  \Gscale@div\@tempb{\linewidth}{\wd\pandoc@box}%
  \ifdim\@tempb\p@<\@tempa\p@\let\@tempa\@tempb\fi% select the smaller of both
  \ifdim\@tempa\p@<\p@\scalebox{\@tempa}{\usebox\pandoc@box}%
  \else\usebox{\pandoc@box}%
  \fi%
}
% Set default figure placement to htbp
\def\fps@figure{htbp}
\makeatother
\setlength{\emergencystretch}{3em} % prevent overfull lines
\providecommand{\tightlist}{%
  \setlength{\itemsep}{0pt}\setlength{\parskip}{0pt}}
\usepackage{bookmark}
\IfFileExists{xurl.sty}{\usepackage{xurl}}{} % add URL line breaks if available
\urlstyle{same}
\hypersetup{
  hidelinks,
  pdfcreator={LaTeX via pandoc}}

\author{}
\date{}

\begin{document}
\frontmatter

\mainmatter
\chapter{Kalkulus dengan EMT}\label{kalkulus-dengan-emt}

Nama : Nazwa Yuan Adelia Putri NIM : 23030630095 Kelas : Matematika B 2023

\begin{center}\rule{0.5\linewidth}{0.5pt}\end{center}

Materi Kalkulus mencakup di antaranya:

\begin{itemize}
\item
  Fungsi (fungsi aljabar, trigonometri, eksponensial, logaritma,
\item
  komposisi fungsi)
\item
  Limit Fungsi,
\item
  Turunan Fungsi,
\item
  Integral Tak Tentu,
\item
  Integral Tentu dan Aplikasinya,
\item
  Barisan dan Deret (kekonvergenan barisan dan deret).
\end{itemize}

EMT (bersama Maxima) dapat digunakan untuk melakukan semua perhitungan di dalam kalkulus, baik secara numerik maupun analitik (eksak).

\section{Mendefinisikan Fungsi}\label{mendefinisikan-fungsi}

Terdapat beberapa cara mendefinisikan fungsi pada EMT, yakni:

\begin{itemize}
\item
  Menggunakan format nama\_fungsi := rumus fungsi (untuk fungsi
\item
  numerik),
\item
  Menggunakan format nama\_fungsi \&= rumus fungsi (untuk fungsi
\item
  simbolik, namun dapat dihitung secara numerik),
\item
  Menggunakan format nama\_fungsi \&\&= rumus fungsi (untuk fungsi
\item
  simbolik murni, tidak dapat dihitung langsung),
\item
  Fungsi sebagai program EMT.
\end{itemize}

Setiap format harus diawali dengan perintah function (bukan sebagai ekspresi).

Berikut adalah adalah beberapa contoh cara mendefinisikan fungsi.

\textgreater function f(x) := 2*x\^{}2+exp(sin(x)) // fungsi numerik

\textgreater f(0), f(1), f(pi)

\begin{verbatim}
1
4.31977682472
20.7392088022
\end{verbatim}

\textgreater function g(x) := sqrt(x\^{}2-3*x)/(x+1)

\textgreater g(3)

\begin{verbatim}
0
\end{verbatim}

\textgreater g(0)

\begin{verbatim}
0
\end{verbatim}

\textgreater g(1)

\begin{verbatim}
Floating point error!
Error in sqrt
Try "trace errors" to inspect local variables after errors.
g:
    useglobal; return sqrt(x^2-3*x)/(x+1) 
Error in:
g(1) ...
    ^
\end{verbatim}

Nb: Floating point error karena untuk x=1, g(x) akan bernilai imajiner

yaitu

\textgreater f(g(5)) // komposisi fungsi

\begin{verbatim}
2.20920171961
\end{verbatim}

\textgreater g(f(5))

\begin{verbatim}
0.950898070639
\end{verbatim}

\textgreater f(0:10) // nilai-nilai f(1), f(2), \ldots, f(10)

\begin{verbatim}
[1,  4.31978,  10.4826,  19.1516,  32.4692,  50.3833,  72.7562,
99.929,  130.69,  163.51,  200.58]
\end{verbatim}

\textgreater fmap(0:10) // sama dengan f(0:10), berlaku untuk semua fungsi

\begin{verbatim}
[1,  4.31978,  10.4826,  19.1516,  32.4692,  50.3833,  72.7562,
99.929,  130.69,  163.51,  200.58]
\end{verbatim}

Misalkan kita akan mendefinisikan fungsi

\[f(x) = \begin{cases} x^3 & x>0 \\ x^2 & x\le 0. \end{cases}\]Fungsi tersebut tidak dapat didefinisikan sebagai fungsi numerik secara ``inline'' menggunakan format :=, melainkan didefinisikan sebagai program. Perhatikan, kata ``map'' digunakan agar fungsi dapat menerima vektor sebagai input, dan hasilnya berupa vektor. Jika tanpa kata ``map'' fungsinya hanya dapat menerima input satu nilai.

\textgreater function map f(x) \ldots{}

\begin{verbatim}
  if x>0 then return x^3
  else return x^2
  endif;
endfunction
\end{verbatim}

\textgreater f(1)

\begin{verbatim}
1
\end{verbatim}

\textgreater f(-2)

\begin{verbatim}
4
\end{verbatim}

\textgreater f(-5:5)

\begin{verbatim}
[25,  16,  9,  4,  1,  0,  1,  8,  27,  64,  125]
\end{verbatim}

\textgreater aspect(1.5); plot2d(``f(x)'',-5,5):

\begin{figure}
\centering
\pandocbounded{\includegraphics[keepaspectratio]{images/Nazwa Yuan Adelia Putri_23030630095_EMT KALKULUS-002.png}}
\caption{images/Nazwa\%20Yuan\%20Adelia\%20Putri\_23030630095\_EMT\%20KALKULUS-002.png}
\end{figure}

\textgreater function f(x) \&= 2*E\^{}x // fungsi simbolik

\begin{verbatim}
                                    x
                                 2 E
\end{verbatim}

\textgreater function g(x) \&= 3*x+1

\begin{verbatim}
                               3 x + 1
\end{verbatim}

\textgreater function h(x) \&= f(g(x)) // komposisi fungsi

\begin{verbatim}
                                 3 x + 1
                              2 E
\end{verbatim}

\chapter{Latihan}\label{latihan}

Bukalah buku Kalkulus. Cari dan pilih beberapa (paling sedikit 5 fungsi berbeda tipe/bentuk/jenis) fungsi dari buku tersebut, kemudian definisikan di EMT pada baris-baris perintah berikut (jika perlu tambahkan lagi). Untuk setiap fungsi, hitung beberapa nilainya, baik untuk satu nilai maupun vektor. Gambar grafik tersebut.

Juga, carilah fungsi beberapa (dua) variabel. Lakukan hal sama seperti di atas.

\begin{enumerate}
\def\labelenumi{\arabic{enumi}.}
\tightlist
\item
  Untuk fungsi
\end{enumerate}

\[k(x) = x^2-4\]

tentukan nilai

\begin{enumerate}
\def\labelenumi{\alph{enumi}.}
\item
  k(-4)
\item
  k(4)
\end{enumerate}

\textgreater function k(x) := x\^{}2 -4

\textgreater k(-4), k(4)

\begin{verbatim}
12
12
\end{verbatim}

\textgreater plot2d(``k'',-4,4):

\begin{figure}
\centering
\pandocbounded{\includegraphics[keepaspectratio]{images/Nazwa Yuan Adelia Putri_23030630095_EMT KALKULUS-004.png}}
\caption{images/Nazwa\%20Yuan\%20Adelia\%20Putri\_23030630095\_EMT\%20KALKULUS-004.png}
\end{figure}

\begin{enumerate}
\def\labelenumi{\arabic{enumi}.}
\setcounter{enumi}{1}
\tightlist
\item
  Untuk fungsi
\end{enumerate}

\[z(x) = \frac{x^2-16}{x-4}\]

hitunglah masing-masing nilai.

\begin{enumerate}
\def\labelenumi{\alph{enumi}.}
\item
  z(6)
\item
  z(2)
\end{enumerate}

\textgreater function z(x) := (x\^{}2-16)/(x-4)

\textgreater z(6), z(2)

\begin{verbatim}
10
6
\end{verbatim}

\textgreater plot2d(``z'',-4,6):

\begin{figure}
\centering
\pandocbounded{\includegraphics[keepaspectratio]{images/Nazwa Yuan Adelia Putri_23030630095_EMT KALKULUS-006.png}}
\caption{images/Nazwa\%20Yuan\%20Adelia\%20Putri\_23030630095\_EMT\%20KALKULUS-006.png}
\end{figure}

\begin{enumerate}
\def\labelenumi{\arabic{enumi}.}
\setcounter{enumi}{2}
\tightlist
\item
  Untuk fungsi
\end{enumerate}

\[r(x) = x^3 - 3x^2 + 2x - 4\]

tentukan nilai r(4), r(-6), r(8)

\textgreater function f(x) := x\textsuperscript{3-3*x}2+2*x-4

\textgreater f(4), f(-6), f(8)

\begin{verbatim}
20
-340
332
\end{verbatim}

\textgreater plot2d(``x\textsuperscript{3-3*x}2+2*x-4'',-2,9):

\begin{figure}
\centering
\pandocbounded{\includegraphics[keepaspectratio]{images/Nazwa Yuan Adelia Putri_23030630095_EMT KALKULUS-008.png}}
\caption{images/Nazwa\%20Yuan\%20Adelia\%20Putri\_23030630095\_EMT\%20KALKULUS-008.png}
\end{figure}

\begin{enumerate}
\def\labelenumi{\arabic{enumi}.}
\setcounter{enumi}{3}
\tightlist
\item
  Tentukan nilai f(200) dari fungsi berikut
\end{enumerate}

\[f(x) = \sqrt{x-64}\]\textgreater function f(x) := sqrt(x-64)

\textgreater f(200)

\begin{verbatim}
11.6619037897
\end{verbatim}

\textgreater plot2d(``sqrt(x-64)'',0,200):

\begin{figure}
\centering
\pandocbounded{\includegraphics[keepaspectratio]{images/Nazwa Yuan Adelia Putri_23030630095_EMT KALKULUS-010.png}}
\caption{images/Nazwa\%20Yuan\%20Adelia\%20Putri\_23030630095\_EMT\%20KALKULUS-010.png}
\end{figure}

\begin{enumerate}
\def\labelenumi{\arabic{enumi}.}
\setcounter{enumi}{4}
\tightlist
\item
  Untuk fungsi
\end{enumerate}

\[f(x) = x^2-3x+2\]\[dan\]\[g(x) = x+3\]

cari nilai fog(-4), gof(0)

\textgreater function f(x) := x\^{}2-3*x+2; \$f(x)

\textgreater function g(x) := x+3; \$g(x)

\textgreater f(g(-4)), g(f(0))

\begin{verbatim}
6
5
\end{verbatim}

\textgreater plot2d(``(x+3)\^{}2-3*(x+3)+2'',-2,2):

\begin{figure}
\centering
\pandocbounded{\includegraphics[keepaspectratio]{images/Nazwa Yuan Adelia Putri_23030630095_EMT KALKULUS-014.png}}
\caption{images/Nazwa\%20Yuan\%20Adelia\%20Putri\_23030630095\_EMT\%20KALKULUS-014.png}
\end{figure}

\begin{enumerate}
\def\labelenumi{\arabic{enumi}.}
\setcounter{enumi}{5}
\tightlist
\item
  Tentukan nilai dari
\end{enumerate}

\[f(x,y):=x^2+y^2+2x-2y+1\]

dengan x=1 dan y=3

\textgreater function f(x,y):= x\textsuperscript{2+y}2+2*x-2*y+1

\textgreater f(1,3)

\begin{verbatim}
7
\end{verbatim}

\textgreater plot3d(``x\textsuperscript{2+y}2+2*x-2*y+1''):

\begin{figure}
\centering
\pandocbounded{\includegraphics[keepaspectratio]{images/Nazwa Yuan Adelia Putri_23030630095_EMT KALKULUS-016.png}}
\caption{images/Nazwa\%20Yuan\%20Adelia\%20Putri\_23030630095\_EMT\%20KALKULUS-016.png}
\end{figure}

\chapter{Menghitung Limit}\label{menghitung-limit}

Perhitungan limit pada EMT dapat dilakukan dengan menggunakan fungsi Maxima, yakni ``limit''. Fungsi ``limit'' dapat digunakan untuk menghitung limit fungsi dalam bentuk ekspresi maupun fungsi yang sudah didefinisikan sebelumnya. Nilai limit dapat dihitung pada sebarang nilai atau pada tak hingga (-inf, minf, dan inf). Limit kiri dan limit kanan juga dapat dihitung, dengan cara memberi opsi ``plus'' atau ``minus''. Hasil limit dapat berupa nilai, ``und' (tak definisi),''ind'' (tak tentu namun terbatas), ``infinity'' (kompleks tak hingga).

Perhatikan beberapa contoh berikut. Perhatikan cara menampilkan perhitungan secara lengkap, tidak hanya menampilkan hasilnya saja.

\textgreater\$showev('limit(1/(2*x-1),x,0))

\[\lim_{x\rightarrow 0}{\frac{1}{2\,x-1}}=-1\]\textgreater\$showev('limit((x\^{}2-3*x-10)/(x-5),x,5))

\[\lim_{x\rightarrow 5}{\frac{x^2-3\,x-10}{x-5}}=7\]\textgreater\$showev('limit(sin(x)/x,x,0))

\[\lim_{x\rightarrow 0}{\frac{\sin x}{x}}=1\]\textgreater plot2d(``sin(x)/x'',-pi,pi):

\begin{figure}
\centering
\pandocbounded{\includegraphics[keepaspectratio]{images/Nazwa Yuan Adelia Putri_23030630095_EMT KALKULUS-020.png}}
\caption{images/Nazwa\%20Yuan\%20Adelia\%20Putri\_23030630095\_EMT\%20KALKULUS-020.png}
\end{figure}

\textgreater\$showev('limit(sin(x\^{}3)/x,x,0))

\[\lim_{x\rightarrow 0}{\frac{\sin x^3}{x}}=0\]\textgreater\$showev('limit(log(x), x, minf))

\[\lim_{x\rightarrow  -\infty }{\log x}={\it infinity}\]\textgreater\$showev('limit((-2)\^{}x,x, inf))

\[\lim_{x\rightarrow \infty }{\left(-2\right)^{x}}={\it infinity}\]\textgreater\$showev('limit(t-sqrt(2-t),t,2,minus))

\[\lim_{t\uparrow 2}{t-\sqrt{2-t}}=2\]\textgreater\$showev('limit(t-sqrt(2-t),t,5,plus)) // Perhatikan hasilnya

\[\lim_{t\downarrow 5}{t-\sqrt{2-t}}=5-\sqrt{3}\,i\]\textgreater plot2d(``x-sqrt(2-x)'',-2,5):

\begin{figure}
\centering
\pandocbounded{\includegraphics[keepaspectratio]{images/Nazwa Yuan Adelia Putri_23030630095_EMT KALKULUS-026.png}}
\caption{images/Nazwa\%20Yuan\%20Adelia\%20Putri\_23030630095\_EMT\%20KALKULUS-026.png}
\end{figure}

\textgreater\$showev('limit((x\textsuperscript{2-9)/(2*x}2-5*x-3),x,3))

\[\lim_{x\rightarrow 3}{\frac{x^2-9}{2\,x^2-5\,x-3}}=\frac{6}{7}\]\textgreater\$showev('limit((1-cos(x))/x,x,0))

\[\lim_{x\rightarrow 0}{\frac{1-\cos x}{x}}=0\]\textgreater\$showev('limit((x\textsuperscript{2+abs(x))/(x}2-abs(x)),x,0))

\[\lim_{x\rightarrow 0}{\frac{\left| x\right| +x^2}{x^2-\left| x
 \right| }}=-1\]\textgreater\$showev('limit((1+1/x)\^{}x,x,inf))

\[\lim_{x\rightarrow \infty }{\left(\frac{1}{x}+1\right)^{x}}=e\]\textgreater\$showev('limit((1+k/x)\^{}x,x,inf))

\[\lim_{x\rightarrow \infty }{\left(\frac{k}{x}+1\right)^{x}}=e^{k}\]\textgreater\$showev('limit((1+x)\^{}(1/x),x,0))

\[\lim_{x\rightarrow 0}{\left(x+1\right)^{\frac{1}{x}}}=e\]\textgreater\$showev('limit((x/(x+k))\^{}x,x,inf))

\[\lim_{x\rightarrow \infty }{\left(\frac{x}{x+k}\right)^{x}}=e^ {- k
  }\]\textgreater{}

\textgreater\$showev('limit(sin(1/x),x,0))

\[\lim_{x\rightarrow 0}{\sin \left(\frac{1}{x}\right)}={\it ind}\]\textgreater\$showev('limit(sin(1/x),x,inf))

\[\lim_{x\rightarrow \infty }{\sin \left(\frac{1}{x}\right)}=0\]\textgreater plot2d(``sin(1/x)'',-5,5):

\begin{figure}
\centering
\pandocbounded{\includegraphics[keepaspectratio]{images/Nazwa Yuan Adelia Putri_23030630095_EMT KALKULUS-036.png}}
\caption{images/Nazwa\%20Yuan\%20Adelia\%20Putri\_23030630095\_EMT\%20KALKULUS-036.png}
\end{figure}

\chapter{Latihan}\label{latihan-1}

Bukalah buku Kalkulus. Cari dan pilih beberapa (paling sedikit 5 fungsi berbeda tipe/bentuk/jenis) fungsi dari buku tersebut, kemudian definisikan di EMT pada baris-baris perintah berikut (jika perlu tambahkan lagi). Untuk setiap fungsi, hitung nilai limit fungsi tersebut di beberapa nilai dan di tak hingga. Gambar grafik fungsi tersebut untuk mengkonfirmasi nilai-nilai limit tersebut.

\begin{enumerate}
\def\labelenumi{\arabic{enumi}.}
\tightlist
\item
  Hitunglah nilai limit berikut.
\end{enumerate}

\[\lim_{x\to 3}(x-8)\]\textgreater\$showev('limit((x-8),x,3))

\[\lim_{x\rightarrow 3}{x-8}=-5\]2. Hitunglah nilai limit berikut.

\[\lim_{x\to 2}\frac{x^2-4}{x+2}\]\textgreater\$showev('limit((x\^{}2-4)/(x=2),x,2))

\[\lim_{x\rightarrow 2}{\frac{x^2-4}{x}=\frac{x^2-4}{2}}=\left(0=0
 \right)\]3. Hitunglah nilai limit berikut dan gambarlah grafiknya.

\[\lim_{t\to 1}\frac{t^2-1}{sin(t-1)}\]\textgreater\$showev('limit((t\^{}2-1)/sin(t-1),t,1))

\[\lim_{t\rightarrow 1}{\frac{t^2-1}{\sin \left(t-1\right)}}=2\]\textgreater(plot2d(``(x\^{}2-1)/sin(x-1)'', -10,10)):

\begin{figure}
\centering
\pandocbounded{\includegraphics[keepaspectratio]{images/Nazwa Yuan Adelia Putri_23030630095_EMT KALKULUS-043.png}}
\caption{images/Nazwa\%20Yuan\%20Adelia\%20Putri\_23030630095\_EMT\%20KALKULUS-043.png}
\end{figure}

\begin{enumerate}
\def\labelenumi{\arabic{enumi}.}
\setcounter{enumi}{3}
\tightlist
\item
  Tentukan nilai limit berikut.
\end{enumerate}

\[\lim_{x\to -1}\frac{\sqrt{1-2x}}{(4x+2)^2}\]\textgreater\$showev('limit((sqrt(1-2*x))/((4*x+2)\^{}2), x, -1))

\[\lim_{x\rightarrow -1}{\frac{\sqrt{1-2\,x}}{\left(4\,x+2\right)^2}}=
 \frac{\sqrt{3}}{4}\]5. Tentukan nilai limit berikut.

\[\lim_{t\to 0}\frac{(t-sin(t))^2}{t^2}\]\textgreater\$showev('limit(((t-sin(t))\textsuperscript{2)/(t}2),t,0))

\[\lim_{t\rightarrow 0}{\frac{\left(t-\sin t\right)^2}{t^2}}=0\]\textgreater(plot2d(``((x-sin(x))\textsuperscript{2)/(x}2)'',-5,5)):

\begin{figure}
\centering
\pandocbounded{\includegraphics[keepaspectratio]{images/Nazwa Yuan Adelia Putri_23030630095_EMT KALKULUS-048.png}}
\caption{images/Nazwa\%20Yuan\%20Adelia\%20Putri\_23030630095\_EMT\%20KALKULUS-048.png}
\end{figure}

\textgreater{}

\chapter{Turunan Fungsi}\label{turunan-fungsi}

Definisi turunan:

\[f'(x) = \lim_{h\to 0} \frac{f(x+h)-f(x)}{h}\]Berikut adalah contoh-contoh menentukan turunan fungsi dengan menggunakan definisi turunan (limit).

\textgreater\$showev('limit(((x+h)\textsuperscript{n-x}n)/h,h,0)) // turunan x\^{}n

\[\lim_{h\rightarrow 0}{\frac{\left(x+h\right)^{n}-x^{n}}{h}}=n\,x^{n
 -1}\]Mengapa hasilnya seperti itu? Tuliskan atau tunjukkan bahwa hasil limit tersebut benar, sehingga benar turunan fungsinya benar. Tulis penjelasan Anda di komentar ini.

Sebagai petunjuk, ekspansikan (x+h)\^{}n dengan menggunakan teorema binomial.

\section{BUKTI}\label{bukti}

Untuk

Dengan

maka

Jadi, terbukti benar bahwa

\begin{center}\rule{0.5\linewidth}{0.5pt}\end{center}

\textgreater\$showev('limit((sin(x+h)-sin(x))/h,h,0)) // turunan sin(x)

\[\lim_{h\rightarrow 0}{\frac{\sin \left(x+h\right)-\sin x}{h}}=\cos 
 x\]Mengapa hasilnya seperti itu? Tuliskan atau tunjukkan bahwa hasil limit tersebut

benar, sehingga benar turunan fungsinya benar. Tulis penjelasan Anda di komentar ini.

Sebagai petunjuk, ekspansikan sin(x+h) dengan menggunakan rumus jumlah dua sudut.

\section{Bukti}\label{bukti-1}

\[f'(x) = \lim_{h\to 0} \frac{sin(x+h)-sin(x)}{h}\]\[sin(a+b)=sin(a)cos(a)+cos(a)sin(b)\]\[= \lim_{h\to 0} \frac{sin(x)cos(h)+cos(x)sin(h)-sin(x)}{h}\]\[= \lim_{h\to 0} sinx.\frac{cos(h)-1}{h}+\lim_{h\to 0} cos(x).\frac{sin(h)}{h}\]\[= sin(x).0+cos(x).1\]\[= cos(x)\]Jadi, terbukti benar bahwa

\[f'(sin(x)) = cos(x)\]---

\textgreater\$showev('limit((log(x+h)-log(x))/h,h,0)) // turunan log(x)

\[\lim_{h\rightarrow 0}{\frac{\log \left(x+h\right)-\log x}{h}}=
 \frac{1}{x}\]Mengapa hasilnya seperti itu? Tuliskan atau tunjukkan bahwa hasil limit tersebut

benar, sehingga benar turunan fungsinya benar. Tulis penjelasan Anda di komentar ini.

Sebagai petunjuk, gunakan sifat-sifat logaritma dan hasil limit pada bagian sebelumnya di atas.

\section{Bukti}\label{bukti-2}

\[f'(x) = \lim_{h\to 0} \frac{log(x+h)-log x}{h}\]\[=\lim_{h\to 0} \frac{\frac{d}{dh}(log(x+h)-log x)}{\frac{d}{dh}(h)}\]\[=\lim_{h\to 0} \frac{\frac{1}{x+h}}{1}\]\[=\lim_{h\to 0} \frac{1}{x+h}\]\[=\frac{1}{x}\]

Jadi, terbukti benar bahwa

\[f'(x) = \lim_{h\to 0} \frac{log(x+h)-log x}{h} = \frac{1}{x}\]---

\textgreater\$showev('limit((1/(x+h)-1/x)/h,h,0)) // turunan 1/x

\[\lim_{h\rightarrow 0}{\frac{\frac{1}{x+h}-\frac{1}{x}}{h}}=-\frac{1
 }{x^2}\]\textgreater\$showev('limit((E\textsuperscript{(x+h)-E}x)/h,h,0)) // turunan f(x)=e\^{}x

\begin{verbatim}
Answering "Is x an integer?" with "integer"
Answering "Is x an integer?" with "integer"
Answering "Is x an integer?" with "integer"
Answering "Is x an integer?" with "integer"
Answering "Is x an integer?" with "integer"
Maxima is asking
Acceptable answers are: yes, y, Y, no, n, N, unknown, uk
Is x an integer?

Use assume!
Error in:
$showev('limit((E^(x+h)-E^x)/h,h,0)) // turunan f(x)=e^x ...
                                     ^
\end{verbatim}

Maxima bermasalah dengan limit:

\[\lim_{h\to 0}\frac{e^{x+h}-e^x}{h}.\]Oleh karena itu diperlukan trik khusus agar hasilnya benar.

\textgreater\$showev('limit((E\^{}h-1)/h,h,0))

\[\lim_{h\rightarrow 0}{\frac{e^{h}-1}{h}}=1\]\textgreater\$factor(E\textsuperscript{(x+h)-E}x)

\[\left(e^{h}-1\right)\,e^{x}\]\textgreater\$showev('limit(factor((E\textsuperscript{(x+h)-E}x)/h),h,0)) // turunan f(x)=e\^{}x

\[\left(\lim_{h\rightarrow 0}{\frac{e^{h}-1}{h}}\right)\,e^{x}=e^{x}\]\textgreater function f(x) \&= x\^{}x

\begin{verbatim}
                                   x
                                  x
\end{verbatim}

\textgreater\$showev('limit((f(x+h)-f(x))/h,h,0)) // turunan f(x)=x\^{}x

\[\lim_{h\rightarrow 0}{\frac{\left(x+h\right)^{x+h}-x^{x}}{h}}=
 {\it infinity}\]Di sini Maxima juga bermasalah terkait limit:

\[\lim_{h\to 0} \frac{(x+h)^{x+h}-x^x}{h}.\]Dalam hal ini diperlukan asumsi nilai x.

\textgreater\&assume(x\textgreater0); \$showev('limit((f(x+h)-f(x))/h,h,0)) // turunan f(x)=x\^{}x

\[\lim_{h\rightarrow 0}{\frac{\left(x+h\right)^{x+h}-x^{x}}{h}}=x^{x}
 \,\left(\log x+1\right)\]\textgreater\&forget(x\textgreater0) // jangan lupa, lupakan asumsi untuk kembali ke semula

\begin{verbatim}
                               [x &gt; 0]
\end{verbatim}

\textgreater\&forget(x\textless0)

\begin{verbatim}
                               [x &lt; 0]
\end{verbatim}

\textgreater\&facts()

\begin{verbatim}
                                  []
\end{verbatim}

\textgreater\$showev('limit((asin(x+h)-asin(x))/h,h,0)) // turunan arcsin(x)

\[\lim_{h\rightarrow 0}{\frac{\arcsin \left(x+h\right)-\arcsin x}{h}}=
 \frac{1}{\sqrt{1-x^2}}\]\textgreater\$showev('limit((tan(x+h)-tan(x))/h,h,0)) // turunan tan(x)

\[\lim_{h\rightarrow 0}{\frac{\tan \left(x+h\right)-\tan x}{h}}=
 \frac{1}{\cos ^2x}\]\textgreater function f(x) \&= sinh(x) // definisikan f(x)=sinh(x)

\begin{verbatim}
                               sinh(x)
\end{verbatim}

\textgreater function df(x) \&= limit((f(x+h)-f(x))/h,h,0); \$df(x) // df(x) = f'(x)

\[\frac{e^ {- x }\,\left(e^{2\,x}+1\right)}{2}\]Hasilnya adalah cosh(x), karena

\[\frac{e^x+e^{-x}}{2}=\cosh(x).\]\textgreater plot2d({[}``f(x)'',``df(x)''{]},-pi,pi,color={[}blue,red{]}):

\begin{figure}
\centering
\pandocbounded{\includegraphics[keepaspectratio]{images/Nazwa Yuan Adelia Putri_23030630095_EMT KALKULUS-078.png}}
\caption{images/Nazwa\%20Yuan\%20Adelia\%20Putri\_23030630095\_EMT\%20KALKULUS-078.png}
\end{figure}

\chapter{Latihan}\label{latihan-2}

Bukalah buku Kalkulus. Cari dan pilih beberapa (paling sedikit 5 fungsi berbeda tipe/bentuk/jenis) fungsi dari buku tersebut, kemudian definisikan di EMT pada baris-baris perintah berikut (jika perlu tambahkan lagi). Untuk setiap fungsi, tentukan turunannya dengan menggunakan definisi turunan (limit), seperti contoh-contoh tersebut. Gambar grafik fungsi asli dan fungsi turunannya pada sumbu koordinat yang sama.

\begin{enumerate}
\def\labelenumi{\arabic{enumi}.}
\tightlist
\item
  Tentukan nilai turunan berikut dan sketsakan grafiknya.
\end{enumerate}

\textgreater function f(x) \&= 4*x\^{}2+8; \$f(x)

\[4\,x^2+8\]\textgreater function df(x) \&= limit((f(x+h)-f(x))/h,h,0); \&df(x)//df(x)=f'(x)

\begin{verbatim}
                                 8 x
\end{verbatim}

\textgreater plot2d({[}``f(x)'',``df(x)''{]},-pi,pi,color={[}red,green{]}):

\begin{figure}
\centering
\pandocbounded{\includegraphics[keepaspectratio]{images/Nazwa Yuan Adelia Putri_23030630095_EMT KALKULUS-080.png}}
\caption{images/Nazwa\%20Yuan\%20Adelia\%20Putri\_23030630095\_EMT\%20KALKULUS-080.png}
\end{figure}

\begin{enumerate}
\def\labelenumi{\arabic{enumi}.}
\setcounter{enumi}{1}
\tightlist
\item
  Carilah turunan dari fungsi berikut
\end{enumerate}

\[f(x)=\frac{4x-1}{x-2}\]\textgreater function f(x) \&= (x-1)/(x-2); \$f(x)

\[\frac{x-1}{x-2}\]\textgreater function df(x) \&= limit((f(x+h)-f(x))/h,h,0); \$df(x) // df(x) = f'(x)

\[-\frac{1}{x^2-4\,x+4}\]\textgreater plot2d({[}``f(x)'',``df(x)''{]},-10,10,color={[}blue,red{]}):

\begin{figure}
\centering
\pandocbounded{\includegraphics[keepaspectratio]{images/Nazwa Yuan Adelia Putri_23030630095_EMT KALKULUS-084.png}}
\caption{images/Nazwa\%20Yuan\%20Adelia\%20Putri\_23030630095\_EMT\%20KALKULUS-084.png}
\end{figure}

\begin{enumerate}
\def\labelenumi{\arabic{enumi}.}
\setcounter{enumi}{2}
\tightlist
\item
  Carilah turunan dari fungsi berikut
\end{enumerate}

\[f(x)= \frac{3}{\sqrt{x-2}}\]\textgreater function f(x) \&= 3/sqrt(x-2); \$f(x)

\[\frac{3}{\sqrt{x-2}}\]\textgreater function df(x) \&= limit((f(x+h)-f(x))/h,h,0); \$df(x) // df(x) = f'(x)function f(x) \&= 3/sqrt(x-2); \$f(x)

\[-\frac{3}{2\,\left(x-2\right)^{\frac{3}{2}}}\]\textgreater plot2d({[}``f(x)'',``df(x)''{]},-10,10,color={[}yellow,red{]}):

\begin{figure}
\centering
\pandocbounded{\includegraphics[keepaspectratio]{images/Nazwa Yuan Adelia Putri_23030630095_EMT KALKULUS-088.png}}
\caption{images/Nazwa\%20Yuan\%20Adelia\%20Putri\_23030630095\_EMT\%20KALKULUS-088.png}
\end{figure}

\begin{enumerate}
\def\labelenumi{\arabic{enumi}.}
\setcounter{enumi}{3}
\tightlist
\item
  Carilah turunan fungsi berikut.
\end{enumerate}

\[f(x) = 2sin(x)+3cos(x)\]\textgreater function f(x) \&= (4*sin(x)+6*cos(x)); \$f(x)

\[4\,\sin x+6\,\cos x\]\textgreater function df(x) \&= limit((f(x+h)-f(x))/h,h,0); \&df(x)

\begin{verbatim}
                      - 2 (3 sin(x) - 2 cos(x))
\end{verbatim}

\textgreater plot2d({[}``f(x)'',``df(x)''{]},-pi,pi,color={[}blue,yellow{]}):

\begin{figure}
\centering
\pandocbounded{\includegraphics[keepaspectratio]{images/Nazwa Yuan Adelia Putri_23030630095_EMT KALKULUS-091.png}}
\caption{images/Nazwa\%20Yuan\%20Adelia\%20Putri\_23030630095\_EMT\%20KALKULUS-091.png}
\end{figure}

\begin{enumerate}
\def\labelenumi{\arabic{enumi}.}
\setcounter{enumi}{4}
\tightlist
\item
  Tentukan turunan dan grafik fungsi berikut.
\end{enumerate}

\[f(x) = \frac{sin(x)+cos(x)}{cos(x)}\]\textgreater function f(x) \&= (sin(x)+cos(x))/(cos(x)); \$f(x)

\[\frac{\sin x+\cos x}{\cos x}\]\textgreater function df(x) \&= limit((f(x+h)-f(x))/h,h,0); \$df(x) // df(x) = f'(x)

\[\frac{\sin ^2x+\cos ^2x}{\cos ^2x}\]\textgreater plot2d({[}``f(x)'',``df(x)''{]},-pi,pi,color={[}blue,yellow{]}):

\begin{figure}
\centering
\pandocbounded{\includegraphics[keepaspectratio]{images/Nazwa Yuan Adelia Putri_23030630095_EMT KALKULUS-095.png}}
\caption{images/Nazwa\%20Yuan\%20Adelia\%20Putri\_23030630095\_EMT\%20KALKULUS-095.png}
\end{figure}

\textgreater{}

\chapter{Integral}\label{integral}

EMT dapat digunakan untuk menghitung integral, baik integral tak tentu maupun integral tentu. Untuk integral tak tentu (simbolik) sudah tentu EMT menggunakan Maxima, sedangkan untuk perhitungan integral tentu EMT sudah menyediakan beberapa fungsi yang mengimplementasikan algoritma kuadratur (perhitungan integral tentu menggunakan metode numerik).

Pada notebook ini akan ditunjukkan perhitungan integral tentu dengan menggunakan Teorema Dasar Kalkulus:

\[\int_a^b f(x)\ dx = F(b)-F(a), \quad \text{ dengan  } F'(x) = f(x).\]Fungsi untuk menentukan integral adalah integrate. Fungsi ini dapat digunakan untuk menentukan, baik integral tentu maupun tak tentu (jika fungsinya memiliki antiderivatif). Untuk perhitungan integral tentu fungsi integrate menggunakan metode numerik (kecuali fungsinya tidak integrabel, kita tidak akan menggunakan metode ini).

\textgreater\$showev('integrate(x\^{}n,x))

\begin{verbatim}
Answering "Is n equal to -1?" with "no"
\end{verbatim}

\[\int {x^{n}}{\;dx}=\frac{x^{n+1}}{n+1}\]\textgreater\$showev('integrate(1/(1+x),x))

\[\int {\frac{1}{x+1}}{\;dx}=\log \left(x+1\right)\]\textgreater\$showev('integrate(1/(1+x\^{}2),x))

\[\int {\frac{1}{x^2+1}}{\;dx}=\arctan x\]\textgreater\$showev('integrate(1/sqrt(1-x\^{}2),x))

\[\int {\frac{1}{\sqrt{1-x^2}}}{\;dx}=\arcsin x\]\textgreater\$showev('integrate(sin(x),x,0,pi))

\[\int_{0}^{\pi}{\sin x\;dx}=2\]\textgreater\$showev('integrate(sin(x),x,a,b))

\[\int_{a}^{b}{\sin x\;dx}=\cos a-\cos b\]\textgreater\$showev('integrate(x\^{}n,x,a,b))

\begin{verbatim}
Answering "Is n positive, negative or zero?" with "positive"
\end{verbatim}

\[\int_{a}^{b}{x^{n}\;dx}=\frac{b^{n+1}}{n+1}-\frac{a^{n+1}}{n+1}\]\textgreater\$showev('integrate(x\^{}2*sqrt(2*x+1),x))

\[\int {x^2\,\sqrt{2\,x+1}}{\;dx}=\frac{\left(2\,x+1\right)^{\frac{7
 }{2}}}{28}-\frac{\left(2\,x+1\right)^{\frac{5}{2}}}{10}+\frac{\left(
 2\,x+1\right)^{\frac{3}{2}}}{12}\]\textgreater\$showev('integrate(x\^{}2*sqrt(2*x+1),x,0,2))

\[\int_{0}^{2}{x^2\,\sqrt{2\,x+1}\;dx}=\frac{2\,5^{\frac{5}{2}}}{21}-
 \frac{2}{105}\]\textgreater\$ratsimp(\%)

\[\int_{0}^{2}{x^2\,\sqrt{2\,x+1}\;dx}=\frac{2\,5^{\frac{7}{2}}-2}{
 105}\]\textgreater\$showev('integrate((sin(sqrt(x)+a)*E\textsuperscript{sqrt(x))/sqrt(x),x,0,pi}2))

\[\int_{0}^{\pi^2}{\frac{\sin \left(\sqrt{x}+a\right)\,e^{\sqrt{x}}}{
 \sqrt{x}}\;dx}=\left(-e^{\pi}-1\right)\,\sin a+\left(e^{\pi}+1
 \right)\,\cos a\]\textgreater\$factor(\%)

\[\int_{0}^{\pi^2}{\frac{\sin \left(\sqrt{x}+a\right)\,e^{\sqrt{x}}}{
 \sqrt{x}}\;dx}=\left(-e^{\pi}-1\right)\,\left(\sin a-\cos a\right)\]\textgreater function map f(x) \&= E\textsuperscript{(-x}2)

\begin{verbatim}
                                    2
                                 - x
                                E
\end{verbatim}

\textgreater\$showev('integrate(f(x),x))

\[\int {e^ {- x^2 }}{\;dx}=\frac{\sqrt{\pi}\,\mathrm{erf}\left(x
 \right)}{2}\]Fungsi f tidak memiliki antiturunan, integralnya masih memuat integral lain.

\[erf(x) = \int \frac{e^{-x^2}}{\sqrt{\pi}} \ dx.\]Kita tidak dapat menggunakan teorema Dasar kalkulus untuk menghitung integral tentu fungsi tersebut jika semua batasnya berhingga. Dalam hal ini dapat digunakan metode numerik (rumus kuadratur).

Misalkan kita akan menghitung:

\[\int_{0}^{\pi}{e^ {- x^2 }\;dx}\]\textgreater x=0:0.1:pi-0.1; plot2d(x,f(x+0.1),\textgreater bar); plot2d(``f(x)'',0,pi,\textgreater add):

\begin{figure}
\centering
\pandocbounded{\includegraphics[keepaspectratio]{images/Nazwa Yuan Adelia Putri_23030630095_EMT KALKULUS-112.png}}
\caption{images/Nazwa\%20Yuan\%20Adelia\%20Putri\_23030630095\_EMT\%20KALKULUS-112.png}
\end{figure}

Integral tentu

\[\int_{0}^{\pi}{e^ {- x^2 }\;dx}\]dapat dihampiri dengan jumlah luas persegi-persegi panjang di bawah kurva y=f(x) tersebut. Langkah-langkahnya adalah sebagai berikut.

\textgreater t \&= makelist(a,a,0,pi-0.1,0.1); // t sebagai list untuk menyimpan nilai-nilai x

\textgreater fx \&= makelist(f(t{[}i{]}+0.1),i,1,length(t)); // simpan nilai-nilai f(x)

\textgreater// jangan menggunakan x sebagai list, kecuali Anda pakar Maxima!

Hasilnya adalah:

\[\int_{0}^{\pi}{e^ {- x^2 }\;dx}=0.8362196102528469\]Jumlah tersebut diperoleh dari hasil kali lebar sub-subinterval (=0.1) dan jumlah nilai-nilai f(x) untuk x = 0.1, 0.2, 0.3, \ldots, 3.2.

\textgreater0.1*sum(f(x+0.1)) // cek langsung dengan perhitungan numerik EMT

\begin{verbatim}
0.836219610253
\end{verbatim}

Untuk mendapatkan nilai integral tentu yang mendekati nilai sebenarnya, lebar sub-intervalnya dapat diperkecil lagi, sehingga daerah di bawah kurva tertutup semuanya, misalnya dapat digunakan lebar subinterval 0.001. (Silakan dicoba!)

Meskipun Maxima tidak dapat menghitung integral tentu fungsi tersebut untuk batas-batas yang berhingga, namun integral tersebut dapat dihitung secara eksak jika batas-batasnya tak hingga. Ini adalah salah satu keajaiban di dalam matematika, yang terbatas tidak dapat dihitung secara eksak, namun yang tak hingga malah dapat dihitung secara eksak.

\textgreater\$showev('integrate(f(x),x,0,inf))

\[\int_{0}^{\infty }{e^ {- x^2 }\;dx}=\frac{\sqrt{\pi}}{2}\]Berikut adalah contoh lain fungsi yang tidak memiliki antiderivatif, sehingga integral tentunya hanya dapat dihitung dengan metode numerik.

\textgreater function f(x) \&= x\^{}x

\begin{verbatim}
                                   x
                                  x
\end{verbatim}

\textgreater\$showev('integrate(f(x),x,0,1))

\[\int_{0}^{1}{x^{x}\;dx}=\int_{0}^{1}{x^{x}\;dx}\]\textgreater x=0:0.1:1-0.01; plot2d(x,f(x+0.01),\textgreater bar); plot2d(``f(x)'',0,1,\textgreater add):

\begin{figure}
\centering
\pandocbounded{\includegraphics[keepaspectratio]{images/Nazwa Yuan Adelia Putri_23030630095_EMT KALKULUS-117.png}}
\caption{images/Nazwa\%20Yuan\%20Adelia\%20Putri\_23030630095\_EMT\%20KALKULUS-117.png}
\end{figure}

Maxima gagal menghitung integral tentu tersebut secara langsung menggunakan perintah integrate. Berikut kita lakukan seperti contoh sebelumnya untuk mendapat hasil atau pendekatan nilai integral tentu tersebut.

\textgreater t \&= makelist(a,a,0,1-0.01,0.01);

\textgreater fx \&= makelist(f(t{[}i{]}+0.01),i,1,length(t));

\[\int_{0}^{1}{x^{x}\;dx}=0.7834935879025506\]Apakah hasil tersebut cukup baik? perhatikan gambarnya.

\chapter{Latihan}\label{latihan-3}

\begin{itemize}
\item
  Bukalah buku Kalkulus.
\item
  Cari dan pilih beberapa (paling sedikit 5 fungsi berbeda
\item
  tipe/bentuk/jenis) fungsi dari buku tersebut, kemudian definisikan di
\item
  EMT pada baris-baris perintah berikut (jika perlu tambahkan lagi).
\item
  Untuk setiap fungsi, tentukan anti turunannya (jika ada), hitunglah
\item
  integral tentu dengan batas-batas yang menarik (Anda tentukan
\item
  sendiri), seperti contoh-contoh tersebut.
\item
  Lakukan hal yang sama untuk fungsi-fungsi yang tidak dapat
\item
  diintegralkan (cari sedikitnya 3 fungsi).
\item
  Gambar grafik fungsi dan daerah integrasinya pada sumbu koordinat
\item
  yang sama.
\item
  Gunakan integral tentu untuk mencari luas daerah yang dibatasi oleh
\item
  dua kurva yang berpotongan di dua titik. (Cari dan gambar kedua kurva
\item
  dan arsir (warnai) daerah yang dibatasi oleh keduanya.)
\item
  Gunakan integral tentu untuk menghitung volume benda putar kurva y=
\item
  f(x) yang diputar mengelilingi sumbu x dari x=a sampai x=b, yakni
\end{itemize}

\[V = \int_a^b \pi (f(x)^2\ dx.\](Pilih fungsinya dan gambar kurva dan benda putar yang dihasilkan. Anda dapat mencari contoh-contoh bagaimana cara menggambar benda hasil perputaran suatu kurva.)

\begin{itemize}
\tightlist
\item
  Gunakan integral tentu untuk menghitung panjang kurva y=f(x) dari x=a sampai x=b dengan menggunakan rumus:
\end{itemize}

\[S = \int_a^b \sqrt{1+(f'(x))^2} \ dx.\](Pilih fungsi dan gambar kurvanya.)

\begin{enumerate}
\def\labelenumi{\arabic{enumi}.}
\tightlist
\item
  Tentukan panjang kurva dan volume benda putar kurva y=f(x) yang diputar mengelilingi sumbu x dari x=0 sampai x=4
\end{enumerate}

\[f(x)= x^3\]\textgreater function f(x)\&=x\^{}3; \$f(x)

\[x^3\]\textgreater\$showev('integrate(pi*(f(x))\^{}2,x,0,4))

\[\pi\,\int_{0}^{4}{x^6\;dx}=\frac{16384\,\pi}{7}\]turunan fungsi f(x)

\textgreater function df(x) \&= limit((f(x+h)-f(x))/h,h,0); \$df(x) // df(x) = f'(x)

\[3\,x^2\]panjang kurva

\textgreater\$showev('integrate(sqrt((1+df(x)\^{}2)),x,0,4))

\[\int_{0}^{4}{\sqrt{9\,x^4+1}\;dx}=\int_{0}^{4}{\sqrt{9\,x^4+1}\;dx}\]grafik benda putar mengelilingi sumbu x

\textgreater{} plot3d(``x\^{}3'',2,0,2,rotate=2):

\begin{figure}
\centering
\pandocbounded{\includegraphics[keepaspectratio]{images/Nazwa Yuan Adelia Putri_23030630095_EMT KALKULUS-126.png}}
\caption{images/Nazwa\%20Yuan\%20Adelia\%20Putri\_23030630095\_EMT\%20KALKULUS-126.png}
\end{figure}

\begin{enumerate}
\def\labelenumi{\arabic{enumi}.}
\setcounter{enumi}{1}
\tightlist
\item
  Tentukan panjang kurva dan volume benda putar kurva y=f(x) yang diputar mengelilingi sumbu x dari x=-1 sampai x=2
\end{enumerate}

volume benda putar

\textgreater function f(x)\&=3*x\^{}2+2*x+1; \$f(x)

\[3\,x^2+2\,x+1\]\textgreater\$showev('integrate(pi*(f(x))\^{}2,x,-1,2))

\[\pi\,\int_{-1}^{2}{\left(3\,x^2+2\,x+1\right)^2\;dx}=\frac{717\,\pi
 }{5}\]menentukan turunan fungsi f(x)

\textgreater function df(x) \&= limit((f(x+h)-f(x))/h,h,0); \$df(x) // df(x) = f'(x)

\[6\,x+2\]menentukan panjang kurva

\textgreater\$showev('integrate(sqrt((1+df(x)\^{}2)),x,-1,2))

\[\int_{-1}^{2}{\sqrt{\left(6\,x+2\right)^2+1}\;dx}=\frac{
 {\rm asinh}\; 14+14\,\sqrt{197}}{12}+\frac{{\rm asinh}\; 4+4\,\sqrt{
 17}}{12}\]menggambar plot benda putar mengelilingi sumbu x

\textgreater plot3d(``3x\^{}2+2x+1'',2,-1,2,rotate=2):

\begin{figure}
\centering
\pandocbounded{\includegraphics[keepaspectratio]{images/Nazwa Yuan Adelia Putri_23030630095_EMT KALKULUS-131.png}}
\caption{images/Nazwa\%20Yuan\%20Adelia\%20Putri\_23030630095\_EMT\%20KALKULUS-131.png}
\end{figure}

\begin{enumerate}
\def\labelenumi{\arabic{enumi}.}
\setcounter{enumi}{2}
\tightlist
\item
  Integral tentu dengan batas {[}-1,1{]} dari fungsi berikut
\end{enumerate}

\textgreater function map f(x) \&= (x\^{}2+8*x-9); \$f(x)

\[x^2+8\,x-9\]mencari nilai dari integral tentu fungsi f(x) dengan batas {[}-1,1{]}

\textgreater\$showev('integrate(f(x), x, -1, 1))

\[\int_{-1}^{1}{x^2+8\,x-9\;dx}=-\frac{52}{3}\]4. Tentukan panjang kurva dan volume benda putar kurva y=f(x) yang diputar mengelilingi sumbu x dari x=0 sampai x=2

\textgreater\$showev('integrate(pi*(f(x))\^{}2,x,0,2))

\[\pi\,\int_{0}^{2}{\left(x^2+8\,x-9\right)^2\;dx}=\frac{1006\,\pi}{
 15}\]\textgreater function df(x) \&= limit((f(x+h)-f(x))/h,h,0); \$df(x) // df(x) = f'(x)

\[2\,x+8\]menentukan pangjang kurva

\textgreater\$showev('integrate(sqrt((1+df(x)\^{}2)),x,0,2))

\[\int_{0}^{2}{\sqrt{\left(2\,x+8\right)^2+1}\;dx}=\frac{
 {\rm asinh}\; 12+12\,\sqrt{145}}{4}-\frac{{\rm asinh}\; 8+8\,\sqrt{
 65}}{4}\]menggambar plot

\textgreater plot3d(``4x\^{}2+1'',2,0,2,rotate=2):

\begin{figure}
\centering
\pandocbounded{\includegraphics[keepaspectratio]{images/Nazwa Yuan Adelia Putri_23030630095_EMT KALKULUS-137.png}}
\caption{images/Nazwa\%20Yuan\%20Adelia\%20Putri\_23030630095\_EMT\%20KALKULUS-137.png}
\end{figure}

\begin{enumerate}
\def\labelenumi{\arabic{enumi}.}
\setcounter{enumi}{4}
\tightlist
\item
  Tentukan integral fungsi berikut.
\end{enumerate}

\textgreater function f(x) \&= (sqrt(2*x))/x +3/x\^{}5 ; \$f(x)

\[\frac{\sqrt{2}}{\sqrt{x}}+\frac{3}{x^5}\]\textgreater\$showev('integrate((((sqrt(2*x))/x) +(3/x\^{}5)),x))

\[\int {\frac{\sqrt{2}}{\sqrt{x}}+\frac{3}{x^5}}{\;dx}=2^{\frac{3}{2}
 }\,\sqrt{x}-\frac{3}{4\,x^4}\]\textgreater x=0:0.1:5-0.1; plot2d(x,f(x+0.1),\textgreater bar); plot2d(``f(x)'',0,5,\textgreater add):

\begin{figure}
\centering
\pandocbounded{\includegraphics[keepaspectratio]{images/Nazwa Yuan Adelia Putri_23030630095_EMT KALKULUS-140.png}}
\caption{images/Nazwa\%20Yuan\%20Adelia\%20Putri\_23030630095\_EMT\%20KALKULUS-140.png}
\end{figure}

\chapter{Barisan dan Deret}\label{barisan-dan-deret}

(Catatan: bagian ini belum lengkap. Anda dapat membaca contoh-contoh pengguanaan EMT dan Maxima untuk menghitung limit barisan, rumus jumlah parsial suatu deret, jumlah tak hingga suatu deret konvergen, dan sebagainya. Anda dapat mengeksplor contoh-contoh di EMT atau perbagai panduan penggunaan Maxima di software Maxima atau dari Internet.)

Barisan dapat didefinisikan dengan beberapa cara di dalam EMT, di antaranya:

\begin{itemize}
\item
  dengan cara yang sama seperti mendefinisikan vektor dengan
\item
  elemen-elemen beraturan (menggunakan titik dua ``:'');
\item
  menggunakan perintah ``sequence'' dan rumus barisan (suku ke -n);
\item
  menggunakan perintah ``iterate'' atau ``niterate'';
\item
  menggunakan fungsi Maxima ``create\_list'' atau ``makelist'' untuk
\item
  menghasilkan barisan simbolik;
\item
  menggunakan fungsi biasa yang inputnya vektor atau barisan;
\item
  menggunakan fungsi rekursif.
\end{itemize}

EMT menyediakan beberapa perintah (fungsi) terkait barisan, yakni:

\begin{itemize}
\item
  sum: menghitung jumlah semua elemen suatu barisan
\item
  cumsum: jumlah kumulatif suatu barisan
\item
  differences: selisih antar elemen-elemen berturutan
\end{itemize}

EMT juga dapat digunakan untuk menghitung jumlah deret berhingga maupun deret tak hingga, dengan menggunakan perintah (fungsi) ``sum''. Perhitungan dapat dilakukan secara numerik maupun simbolik dan eksak.

Berikut adalah beberapa contoh perhitungan barisan dan deret menggunakan EMT.

\textgreater1:10 // barisan sederhana

\begin{verbatim}
[1,  2,  3,  4,  5,  6,  7,  8,  9,  10]
\end{verbatim}

\textgreater1:2:30

\begin{verbatim}
[1,  3,  5,  7,  9,  11,  13,  15,  17,  19,  21,  23,  25,  27,  29]
\end{verbatim}

\textgreater sum(1:2:30), sum(1/(1:2:30))

\begin{verbatim}
225
2.33587263431
\end{verbatim}

\textgreater\$'sum(k, k, 1, n) = factor(ev(sum(k, k, 1, n),simpsum=true)) // simpsum:menghitung deret secara simbolik

\[\sum_{k=1}^{n}{k}=\frac{n\,\left(n+1\right)}{2}\]\textgreater\$'sum(1/(3\^{}k+k), k, 0, inf) = factor(ev(sum(1/(3\^{}k+k), k, 0, inf),simpsum=true))

\[\sum_{k=0}^{\infty }{\frac{1}{3^{k}+k}}=\sum_{k=0}^{\infty }{\frac{
 1}{3^{k}+k}}\]Di sini masih gagal, hasilnya tidak dihitung.

\textgreater\$'sum(1/x\^{}2, x, 1, inf)= ev(sum(1/x\^{}2, x, 1, inf),simpsum=true) // ev: menghitung nilai ekspresi

\[\sum_{x=1}^{\infty }{\frac{1}{x^2}}=\frac{\pi^2}{6}\]\textgreater\$'sum((-1)\^{}(k-1)/k, k, 1, inf) = factor(ev(sum((-1)\^{}(x-1)/x, x, 1, inf),simpsum=true))

\[\sum_{k=1}^{\infty }{\frac{\left(-1\right)^{k-1}}{k}}=-\sum_{x=1}^{
 \infty }{\frac{\left(-1\right)^{x}}{x}}\]Di sini masih gagal, hasilnya tidak dihitung.

\textgreater\$'sum((-1)\^{}k/(2*k-1), k, 1, inf) = factor(ev(sum((-1)\^{}k/(2*k-1), k, 1, inf),simpsum=true))

\[\sum_{k=1}^{\infty }{\frac{\left(-1\right)^{k}}{2\,k-1}}=\sum_{k=1
 }^{\infty }{\frac{\left(-1\right)^{k}}{2\,k-1}}\]\textgreater\$ev(sum(1/n!, n, 0, inf),simpsum=true)

\[\sum_{n=0}^{\infty }{\frac{1}{n!}}\]Di sini masih gagal, hasilnya tidak dihitung, harusnya hasilnya e.

\textgreater\&assume(abs(x)\textless1); \$'sum(a*x\^{}k, k, 0, inf)=ev(sum(a*x\^{}k, k, 0, inf),simpsum=true), \&forget(abs(x)\textless1);

\[a\,\sum_{k=0}^{\infty }{x^{k}}=\frac{a}{1-x}\]Deret geometri tak hingga, dengan asumsi rasional antara -1 dan 1.

\textgreater{}

\chapter{Deret Taylor}\label{deret-taylor}

Deret Taylor suatu fungsi f yang diferensiabel sampai tak hingga di sekitar x=a adalah:

\[f(x) = \sum_{k=0}^\infty \frac{(x-a)^k f^{(k)}(a)}{k!}.\]\textgreater\$'e\^{}x =taylor(exp(x),x,0,10) // deret Taylor e\^{}x di sekitar x=0, sampai suku ke-11

\[e^{x}=\frac{x^{10}}{3628800}+\frac{x^9}{362880}+\frac{x^8}{40320}+
 \frac{x^7}{5040}+\frac{x^6}{720}+\frac{x^5}{120}+\frac{x^4}{24}+
 \frac{x^3}{6}+\frac{x^2}{2}+x+1\]\textgreater\$'log(x)=taylor(log(x),x,1,10)// deret log(x) di sekitar x=1

\[\log x=x-\frac{\left(x-1\right)^{10}}{10}+\frac{\left(x-1\right)^9
 }{9}-\frac{\left(x-1\right)^8}{8}+\frac{\left(x-1\right)^7}{7}-
 \frac{\left(x-1\right)^6}{6}+\frac{\left(x-1\right)^5}{5}-\frac{
 \left(x-1\right)^4}{4}+\frac{\left(x-1\right)^3}{3}-\frac{\left(x-1
 \right)^2}{2}-1\]---

Contoh Soal

\begin{center}\rule{0.5\linewidth}{0.5pt}\end{center}

\begin{enumerate}
\def\labelenumi{\arabic{enumi}.}
\tightlist
\item
  Tentukan integral tak tentu dari fungi aljabar berikut:
\end{enumerate}

\textgreater\$showev('integrate(x\^{}7,x)+C)

\[C+\int {x^7}{\;dx}=C+\frac{x^8}{8}\]2.Buatlah plot kontur dari fungsi berikut

\[f(x,y) = \frac{1}{2}(x^2+y^2) \text{  dengan level bilangan genap dari 0 sampai 20}\]\textgreater f \&= 1/2*(x\textsuperscript{2+y}2)

\begin{verbatim}
                                2    2
                               y  + x
                               -------
                                  2
\end{verbatim}

\textgreater aspect(1.5):

\begin{figure}
\centering
\pandocbounded{\includegraphics[keepaspectratio]{images/Nazwa Yuan Adelia Putri_23030630095_EMT KALKULUS-153.png}}
\caption{images/Nazwa\%20Yuan\%20Adelia\%20Putri\_23030630095\_EMT\%20KALKULUS-153.png}
\end{figure}

\textgreater plot3d(f) :

\begin{figure}
\centering
\pandocbounded{\includegraphics[keepaspectratio]{images/Nazwa Yuan Adelia Putri_23030630095_EMT KALKULUS-154.png}}
\caption{images/Nazwa\%20Yuan\%20Adelia\%20Putri\_23030630095\_EMT\%20KALKULUS-154.png}
\end{figure}

\textgreater plot2d(f,level=0:2:20,\textgreater hue,\textgreater spectral,n=200,grid=4,r=5):

\begin{figure}
\centering
\pandocbounded{\includegraphics[keepaspectratio]{images/Nazwa Yuan Adelia Putri_23030630095_EMT KALKULUS-155.png}}
\caption{images/Nazwa\%20Yuan\%20Adelia\%20Putri\_23030630095\_EMT\%20KALKULUS-155.png}
\end{figure}

\begin{enumerate}
\def\labelenumi{\arabic{enumi}.}
\setcounter{enumi}{2}
\tightlist
\item
  Seorang ahli geologi sedang melakukan penelitian di daerah
\end{enumerate}

vulkanik. Dia tertarik untuk memahami bagaimana ketinggian permukaan

tanah di sekitar gunung berapi berubah. Ketinggian di suatu titik (x,

\begin{enumerate}
\def\labelenumi{\alph{enumi})}
\setcounter{enumi}{24}
\tightlist
\item
  dalam kilometer di sekitar gunung berapi tersebut dapat dijelaskan
\end{enumerate}

oleh fungsi ketinggian H(x, y), di mana x dan y adalah koordinat dalam

kilometer. Fungsi ketinggian H(x, y) diberikan oleh persamaan:

Gambarkan plot kontur dari fungsi ketinggian H(x, y) untuk

memvisualisasikan perubahan ketinggian di sekitar gunung berapi dengan

level ketinggian 1 sampai 7 km.

\textgreater h \&= 3*x\textsuperscript{2-2*y}2

\begin{verbatim}
                                2      2
                             3 x  - 2 y
\end{verbatim}

\textgreater plot3d(h):

\begin{figure}
\centering
\pandocbounded{\includegraphics[keepaspectratio]{images/Nazwa Yuan Adelia Putri_23030630095_EMT KALKULUS-156.png}}
\caption{images/Nazwa\%20Yuan\%20Adelia\%20Putri\_23030630095\_EMT\%20KALKULUS-156.png}
\end{figure}

\textgreater aspect(1.5):

\begin{figure}
\centering
\pandocbounded{\includegraphics[keepaspectratio]{images/Nazwa Yuan Adelia Putri_23030630095_EMT KALKULUS-157.png}}
\caption{images/Nazwa\%20Yuan\%20Adelia\%20Putri\_23030630095\_EMT\%20KALKULUS-157.png}
\end{figure}

\textgreater plot2d(h,level=1:1:7,\textgreater hue,\textgreater spectral,n=200,grid=4,r=2):

\begin{figure}
\centering
\pandocbounded{\includegraphics[keepaspectratio]{images/Nazwa Yuan Adelia Putri_23030630095_EMT KALKULUS-158.png}}
\caption{images/Nazwa\%20Yuan\%20Adelia\%20Putri\_23030630095\_EMT\%20KALKULUS-158.png}
\end{figure}

\begin{enumerate}
\def\labelenumi{\arabic{enumi}.}
\setcounter{enumi}{3}
\tightlist
\item
  Tentukan integral tak tentu berikut
\end{enumerate}

\[\int (3x^2-4x+5) dx\]\textgreater\$showev('integrate((3*x\^{}2-4*x+5),x))

\[\int {3\,x^2-4\,x+5}{\;dx}=x^3-2\,x^2+5\,x\]5. Diberikan kurva

hitunglah volume benda putar yang dihasilkan ketika kuva tersebut diputar mengelilingi sumbu x dari x=0 hingga x=3

\textgreater function f(x):=x\^{}2

\textgreater\$\& (expand(x\textsuperscript{2)}2)

\[x^4\]\textgreater\$showev ('integrate((x\^{}4),x,0,3))

\[\int_{0}^{3}{x^4\;dx}=\frac{243}{5}\]

\backmatter
\end{document}
