% Options for packages loaded elsewhere
\PassOptionsToPackage{unicode}{hyperref}
\PassOptionsToPackage{hyphens}{url}
\documentclass[
]{book}
\usepackage{xcolor}
\usepackage{amsmath,amssymb}
\setcounter{secnumdepth}{-\maxdimen} % remove section numbering
\usepackage{iftex}
\ifPDFTeX
  \usepackage[T1]{fontenc}
  \usepackage[utf8]{inputenc}
  \usepackage{textcomp} % provide euro and other symbols
\else % if luatex or xetex
  \usepackage{unicode-math} % this also loads fontspec
  \defaultfontfeatures{Scale=MatchLowercase}
  \defaultfontfeatures[\rmfamily]{Ligatures=TeX,Scale=1}
\fi
\usepackage{lmodern}
\ifPDFTeX\else
  % xetex/luatex font selection
\fi
% Use upquote if available, for straight quotes in verbatim environments
\IfFileExists{upquote.sty}{\usepackage{upquote}}{}
\IfFileExists{microtype.sty}{% use microtype if available
  \usepackage[]{microtype}
  \UseMicrotypeSet[protrusion]{basicmath} % disable protrusion for tt fonts
}{}
\makeatletter
\@ifundefined{KOMAClassName}{% if non-KOMA class
  \IfFileExists{parskip.sty}{%
    \usepackage{parskip}
  }{% else
    \setlength{\parindent}{0pt}
    \setlength{\parskip}{6pt plus 2pt minus 1pt}}
}{% if KOMA class
  \KOMAoptions{parskip=half}}
\makeatother
\usepackage{graphicx}
\makeatletter
\newsavebox\pandoc@box
\newcommand*\pandocbounded[1]{% scales image to fit in text height/width
  \sbox\pandoc@box{#1}%
  \Gscale@div\@tempa{\textheight}{\dimexpr\ht\pandoc@box+\dp\pandoc@box\relax}%
  \Gscale@div\@tempb{\linewidth}{\wd\pandoc@box}%
  \ifdim\@tempb\p@<\@tempa\p@\let\@tempa\@tempb\fi% select the smaller of both
  \ifdim\@tempa\p@<\p@\scalebox{\@tempa}{\usebox\pandoc@box}%
  \else\usebox{\pandoc@box}%
  \fi%
}
% Set default figure placement to htbp
\def\fps@figure{htbp}
\makeatother
\setlength{\emergencystretch}{3em} % prevent overfull lines
\providecommand{\tightlist}{%
  \setlength{\itemsep}{0pt}\setlength{\parskip}{0pt}}
\usepackage{bookmark}
\IfFileExists{xurl.sty}{\usepackage{xurl}}{} % add URL line breaks if available
\urlstyle{same}
\hypersetup{
  hidelinks,
  pdfcreator={LaTeX via pandoc}}

\author{}
\date{}

\begin{document}
\frontmatter

\mainmatter
\chapter{Aplikasi Komputer}\label{aplikasi-komputer}

Nama: Nazwa Yuan Adelia Putri Kelas : Matematika B NIM : 23030630095

\begin{center}\rule{0.5\linewidth}{0.5pt}\end{center}

\chapter{Visualisasi dan Perhitungan Geometri dengan EMT}\label{visualisasi-dan-perhitungan-geometri-dengan-emt}

Euler menyediakan beberapa fungsi untuk melakukan visualisasi dan perhitungan geometri, baik secara numerik maupun analitik (seperti biasanya tentunya, menggunakan Maxima). Fungsi-fungsi untuk visualisasi dan perhitungan geometeri tersebut disimpan di dalam file program ``geometry.e'', sehingga file tersebut harus dipanggil sebelum menggunakan fungsi-fungsi atau perintah-perintah untuk geometri.

\textgreater load geometry

\begin{verbatim}
Numerical and symbolic geometry.
\end{verbatim}

\section{Fungsi-fungsi Geometri}\label{fungsi-fungsi-geometri}

Fungsi-fungsi untuk Menggambar Objek Geometri:

defaultd:=textheight()*1.5: nilai asli untuk parameter d\\
setPlotrange(x1,x2,y1,y2): menentukan rentang x dan y pada bidang koordinat\\
setPlotRange(r): pusat bidang koordinat (0,0) dan batas-batas sumbu-x dan y adalah -r sd r\\
plotPoint (P, ``P''): menggambar titik P dan diberi label ``P''\\
plotSegment (A,B, ``AB'', d): menggambar ruas garis AB, diberi label ``AB'' sejauh d\\
plotLine (g, ``g'', d): menggambar garis g diberi label ``g'' sejauh d\\
plotCircle (c,``c'',v,d): Menggambar lingkaran c dan diberi label ``c''\\
plotLabel (label, P, V, d): menuliskan label pada posisi P

Fungsi-fungsi Geometri Analitik (numerik maupun simbolik):

turn(v, phi): memutar vektor v sejauh phi\\
turnLeft(v): memutar vektor v ke kiri\\
turnRight(v): memutar vektor v ke kanan\\
normalize(v): normal vektor v\\
crossProduct(v, w): hasil kali silang vektorv dan w.\\
lineThrough(A, B): garis melalui A dan B, hasilnya {[}a,b,c{]} sdh. ax+by=c.\\
lineWithDirection(A,v): garis melalui A searah vektor v\\
getLineDirection(g): vektor arah (gradien) garis g\\
getNormal(g): vektor normal (tegak lurus) garis g\\
getPointOnLine(g): titik pada garis g\\
perpendicular(A, g): garis melalui A tegak lurus garis g\\
parallel (A, g): garis melalui A sejajar garis g\\
lineIntersection(g, h): titik potong garis g dan h\\
projectToLine(A, g): proyeksi titik A pada garis g\\
distance(A, B): jarak titik A dan B\\
distanceSquared(A, B): kuadrat jarak A dan B\\
quadrance(A, B): kuadrat jarak A dan B\\
areaTriangle(A, B, C): luas segitiga ABC\\
computeAngle(A, B, C): besar sudut \textless ABC\\
angleBisector(A, B, C): garis bagi sudut \textless ABC\\
circleWithCenter (A, r): lingkaran dengan pusat A dan jari-jari r\\
getCircleCenter(c): pusat lingkaran c\\
getCircleRadius(c): jari-jari lingkaran c\\
circleThrough(A,B,C): lingkaran melalui A, B, C\\
middlePerpendicular(A, B): titik tengah AB\\
lineCircleIntersections(g, c): titik potong garis g dan lingkran c\\
circleCircleIntersections (c1, c2): titik potong lingkaran c1 dan c2\\
planeThrough(A, B, C): bidang melalui titik A, B, C

Fungsi-fungsi Khusus Untuk Geometri Simbolik:

getLineEquation (g,x,y): persamaan garis g dinyatakan dalam x dan y\\
getHesseForm (g,x,y,A): bentuk Hesse garis g dinyatakan dalam x dan y dengan titik A pada\\
sisi positif (kanan/atas) garis\\
quad(A,B): kuadrat jarak AB\\
spread(a,b,c): Spread segitiga dengan panjang sisi-sisi a,b,c, yakni sin(alpha)\^{}2 dengan\\
alpha sudut yang menghadap sisi a.\\
crosslaw(a,b,c,sa): persamaan 3 quads dan 1 spread pada segitiga dengan panjang sisi a, b, c.\\
triplespread(sa,sb,sc): persamaan 3 spread sa,sb,sc yang memebntuk suatu segitiga\\
doublespread(sa): Spread sudut rangkap Spread 2*phi, dengan sa=sin(phi)\^{}2 spread a.

\section{Contoh 1: Luas, Lingkaran Luar, Lingkaran Dalam Segitiga}\label{contoh-1-luas-lingkaran-luar-lingkaran-dalam-segitiga}

Untuk menggambar objek-objek geometri, langkah pertama adalah menentukan rentang sumbu-sumbu koordinat. Semua objek geometri akan digambar pada satu bidang koordinat, sampai didefinisikan bidang koordinat yang baru.

\textgreater setPlotRange(-0.5,2.5,-0.5,2.5); // mendefinisikan bidang koordinat baru

Sekarang, tetapkan tiga titik dan plotlah.

\textgreater A={[}1,0{]}; plotPoint(A,``A''); // definisi dan gambar tiga titik

\textgreater B={[}0,1{]}; plotPoint(B,``B'');

\textgreater C={[}2,2{]}; plotPoint(C,``C'');

Kemudian tiga segmen.

\textgreater plotSegment(A,B,``c''); // c=AB

\textgreater plotSegment(B,C,``a''); // a=BC

\textgreater plotSegment(A,C,``b''); // b=AC

Fungsi geometri mencakup fungsi untuk membuat garis dan lingkaran. Format untuk garis adalah {[}a,b,c{]}, yang merepresentasikan garis dengan persamaan ax+by=c.

\textgreater lineThrough(B,C) // garis yang melalui B dan C

\begin{verbatim}
[-1,  2,  2]
\end{verbatim}

Hitung garis tegak lurus yang melalui A pada BC.

\textgreater h=perpendicular(A,lineThrough(B,C)); // garis h tegak lurus BC melalui A

Dan persimpangannya dengan BC.

\textgreater D=lineIntersection(h,lineThrough(B,C)); // D adalah titik potong h dan BC

Plot itu.

\textgreater plotPoint(D,value=1); // koordinat D ditampilkan

\textgreater aspect(1); plotSegment(A,D): // tampilkan semua gambar hasil plot\ldots()

\begin{figure}
\centering
\pandocbounded{\includegraphics[keepaspectratio]{images/Nazwa Yuan Adelia Putri_23030630095_Geometri-001.png}}
\caption{images/Nazwa\%20Yuan\%20Adelia\%20Putri\_23030630095\_Geometri-001.png}
\end{figure}

Hitung luas ABC:

\[L_{\triangle ABC}= \frac{1}{2}AD.BC.\]\textgreater norm(A-D)*norm(B-C)/2 // AD=norm(A-D), BC=norm(B-C)

\begin{verbatim}
1.5
\end{verbatim}

Bandingkan dengan rumus determinan.

\textgreater areaTriangle(A,B,C) // hitung luas segitiga langusng dengan fungsi

\begin{verbatim}
1.5
\end{verbatim}

Cara lain menghitung luas segitigas ABC:

\textgreater distance(A,D)*distance(B,C)/2

\begin{verbatim}
1.5
\end{verbatim}

Sudut pada C.

\textgreater degprint(computeAngle(B,C,A))

\begin{verbatim}
36°52'11.63''
\end{verbatim}

Sekarang, lingkarilah segitiga tersebut.

\textgreater c=circleThrough(A,B,C); // lingkaran luar segitiga ABC

\textgreater R=getCircleRadius(c); // jari2 lingkaran luar

\textgreater O=getCircleCenter(c); // titik pusat lingkaran c

\textgreater plotPoint(O,``O''); // gambar titik ``O''

\textgreater plotCircle(c,``Lingkaran luar segitiga ABC''):

\begin{figure}
\centering
\pandocbounded{\includegraphics[keepaspectratio]{images/Nazwa Yuan Adelia Putri_23030630095_Geometri-003.png}}
\caption{images/Nazwa\%20Yuan\%20Adelia\%20Putri\_23030630095\_Geometri-003.png}
\end{figure}

Tampilkan koordinat titik pusat dan jari-jari lingkaran luar.

\textgreater O, R

\begin{verbatim}
[1.16667,  1.16667]
1.17851130198
\end{verbatim}

Sekarang akan digambar lingkaran dalam segitiga ABC. Titik pusat lingkaran dalam adalah titik potong garis-garis bagi sudut.

\textgreater l=angleBisector(A,C,B); // garis bagi \textless ACB

\textgreater g=angleBisector(C,A,B); // garis bagi \textless CAB

\textgreater P=lineIntersection(l,g) // titik potong kedua garis bagi sudut

\begin{verbatim}
[0.86038,  0.86038]
\end{verbatim}

Tambahkan semuanya ke plot.

\textgreater color(5); plotLine(l); plotLine(g); color(1); // gambar kedua garis bagi sudut

\textgreater plotPoint(P,``P''); // gambar titik potongnya

\textgreater r=norm(P-projectToLine(P,lineThrough(A,B))) // jari-jari lingkaran dalam

\begin{verbatim}
0.509653732104
\end{verbatim}

\textgreater plotCircle(circleWithCenter(P,r),``Lingkaran dalam segitiga ABC''): // gambar lingkaran dalam

\begin{figure}
\centering
\pandocbounded{\includegraphics[keepaspectratio]{images/Nazwa Yuan Adelia Putri_23030630095_Geometri-004.png}}
\caption{images/Nazwa\%20Yuan\%20Adelia\%20Putri\_23030630095\_Geometri-004.png}
\end{figure}

\section{Latihan}\label{latihan}

\begin{enumerate}
\def\labelenumi{\arabic{enumi}.}
\item
  Tentukan ketiga titik singgung lingkaran dalam dengan sisi-sisi segitiga ABC.
\item
  Gambar segitiga dengan titik-titik sudut ketiga titik singgung tersebut. Merupakan segitiga apakah itu?
\item
  Hitung luas segitiga tersebut.
\item
  Tunjukkan bahwa garis bagi sudut yang ke tiga juga melalui titik pusat lingkaran dalam.
\item
  Gambar jari-jari lingkaran dalam.
\item
  Hitung luas lingkaran luar dan luas lingkaran dalam segitiga ABC. Adakah hubungan antara luas kedua lingkaran tersebut dengan luas segitiga ABC?
\item
  Tentukan ketiga titik singgung lingkaran dalam dengan sisi-sisi segitiga ABC.
\end{enumerate}

\textgreater setPlotRange(-2.5,4.5,-2.5,4.5);

\textgreater A={[}-2,1{]}; plotPoint(A,``A'');

\textgreater B={[}1,-2{]}; plotPoint(B,``B'');

\textgreater C={[}4,4{]}; plotPoint(C,``C'');

\begin{enumerate}
\def\labelenumi{\arabic{enumi}.}
\setcounter{enumi}{1}
\tightlist
\item
  Gambar segitiga dengan titik-titik sudut ketiga titik singgung tersebut. Merupakan segitiga apakah itu?
\end{enumerate}

\textgreater plotSegment(A,B,``c'');

\textgreater plotSegment(B,C,``a'');

\textgreater plotSegment(A,C,``b'');

\textgreater aspect(1):

\begin{figure}
\centering
\pandocbounded{\includegraphics[keepaspectratio]{images/Nazwa Yuan Adelia Putri_23030630095_Geometri-005.png}}
\caption{images/Nazwa\%20Yuan\%20Adelia\%20Putri\_23030630095\_Geometri-005.png}
\end{figure}

\begin{enumerate}
\def\labelenumi{\arabic{enumi}.}
\setcounter{enumi}{2}
\tightlist
\item
  Hitung luas segitiga tersebut.
\end{enumerate}

\textgreater distance(A,D)*distance(B,C)/2

\begin{verbatim}
8.0777472107
\end{verbatim}

\begin{enumerate}
\def\labelenumi{\arabic{enumi}.}
\setcounter{enumi}{3}
\tightlist
\item
  Tunjukkan bahwa garis bagi sudut yang ke tiga juga melalui titik pusat lingkaran dalam.
\end{enumerate}

\textgreater l=angleBisector(A,C,B);

\textgreater g=angleBisector(C,A,B);

\textgreater P=lineIntersection(l,g)

\begin{verbatim}
[0.581139,  0.581139]
\end{verbatim}

\textgreater color(5); plotLine(l); plotLine(g); color(1);

\textgreater plotPoint(P,``P'');

\textgreater plotCircle(circleWithCenter(P,r),``Lingkaran dalam segitiga ABC''):

\begin{figure}
\centering
\pandocbounded{\includegraphics[keepaspectratio]{images/Nazwa Yuan Adelia Putri_23030630095_Geometri-006.png}}
\caption{images/Nazwa\%20Yuan\%20Adelia\%20Putri\_23030630095\_Geometri-006.png}
\end{figure}

\begin{enumerate}
\def\labelenumi{\arabic{enumi}.}
\setcounter{enumi}{4}
\tightlist
\item
  Gambar jari-jari lingkaran dalam.
\end{enumerate}

\textgreater r=norm(P-projectToLine(P,lineThrough(A,B)))

\begin{verbatim}
1.52896119631
\end{verbatim}

\textgreater plotCircle(circleWithCenter(P,r),``Lingkaran dalam segitiga ABC''):

\begin{figure}
\centering
\pandocbounded{\includegraphics[keepaspectratio]{images/Nazwa Yuan Adelia Putri_23030630095_Geometri-007.png}}
\caption{images/Nazwa\%20Yuan\%20Adelia\%20Putri\_23030630095\_Geometri-007.png}
\end{figure}

\begin{enumerate}
\def\labelenumi{\arabic{enumi}.}
\setcounter{enumi}{5}
\tightlist
\item
  Hitung luas lingkaran luar dan luas lingkaran dalam segitiga ABC. Adakah hubungan antara luas kedua lingkaran tersebut dengan luas segitiga ABC?
\end{enumerate}

\chapter{Contoh 2: Geometri Smbolik}\label{contoh-2-geometri-smbolik}

Kita dapat menghitung geometri eksak dan simbolik menggunakan Maxima.

File geometri.e menyediakan fungsi-fungsi yang sama (dan lebih banyak lagi) di Maxima. Namun, kita dapat menggunakan komputasi simbolik sekarang.

\textgreater A \&= {[}1,0{]}; B \&= {[}0,1{]}; C \&= {[}2,2{]}; // menentukan tiga titik A, B, C

Fungsi untuk garis dan lingkaran bekerja seperti fungsi Euler, tetapi menyediakan komputasi simbolis.

\textgreater c \&= lineThrough(B,C) // c=BC

\begin{verbatim}
                             [- 1, 2, 2]
\end{verbatim}

Kita bisa mendapatkan persamaan untuk sebuah garis dengan mudah.

\textgreater\$getLineEquation(c,x,y), \$solve(\%,y) \textbar{} expand // persamaan garis c

\[2\,y-x=2\]\[\left[ y=\frac{x}{2}+1 \right] \]\textgreater\$getLineEquation(lineThrough({[}x1,y1{]},{[}x2,y2{]}),x,y), \$solve(\%,y) // persamaan garis melalui(x1, y1) dan (x2, y2)

\[x\,\left({\it y_1}-{\it y_2}\right)+\left({\it x_2}-{\it x_1}
 \right)\,y={\it x_1}\,\left({\it y_1}-{\it y_2}\right)+\left(
 {\it x_2}-{\it x_1}\right)\,{\it y_1}\]\[\left[ y=\frac{-\left({\it x_1}-x\right)\,{\it y_2}-\left(x-
 {\it x_2}\right)\,{\it y_1}}{{\it x_2}-{\it x_1}} \right] \]\textgreater\$getLineEquation(lineThrough(A,{[}x1,y1{]}),x,y) // persamaan garis melalui A dan (x1, y1)

\[\left({\it x_1}-1\right)\,y-x\,{\it y_1}=-{\it y_1}\]\textgreater h \&= perpendicular(A,lineThrough(B,C)) // h melalui A tegak lurus BC

\begin{verbatim}
                              [2, 1, 2]
\end{verbatim}

\textgreater Q \&= lineIntersection(c,h) // Q titik potong garis c=BC dan h

\begin{verbatim}
                                 2  6
                                [-, -]
                                 5  5
\end{verbatim}

\textgreater\$projectToLine(A,lineThrough(B,C)) // proyeksi A pada BC

\[\left[ \frac{2}{5} , \frac{6}{5} \right] \]\textgreater\$distance(A,Q) // jarak AQ

\[\frac{3}{\sqrt{5}}\]\textgreater cc \&= circleThrough(A,B,C); \$cc // (titik pusat dan jari-jari) lingkaran melalui A, B, C

\[\left[ \frac{7}{6} , \frac{7}{6} , \frac{5}{3\,\sqrt{2}} \right] \]\textgreater r\&=getCircleRadius(cc); \$r , \$float(r) // tampilkan nilai jari-jari

\[\frac{5}{3\,\sqrt{2}}\]\[1.178511301977579\]\textgreater\$computeAngle(A,C,B) // nilai \textless ACB

\[\arccos \left(\frac{4}{5}\right)\]\textgreater\$solve(getLineEquation(angleBisector(A,C,B),x,y),y){[}1{]} // persamaan garis bagi \textless ACB

\[y=x\]\textgreater P \&= lineIntersection(angleBisector(A,C,B),angleBisector(C,B,A)); \$P // titik potong 2 garis bagi sudut

\[\left[ \frac{\sqrt{2}\,\sqrt{5}+2}{6} , \frac{\sqrt{2}\,\sqrt{5}+2
 }{6} \right] \]\textgreater P() // hasilnya sama dengan perhitungan sebelumnya

\begin{verbatim}
[0.86038,  0.86038]
\end{verbatim}

\section{Garis dan Lingkaran yang Berpotongan}\label{garis-dan-lingkaran-yang-berpotongan}

Tentu saja, kita juga bisa memotong garis dengan lingkaran, dan lingkaran dengan lingkaran.

\textgreater A \&:= {[}1,0{]}; c=circleWithCenter(A,4);

\textgreater B \&:= {[}1,2{]}; C \&:= {[}2,1{]}; l=lineThrough(B,C);

\textgreater setPlotRange(5); plotCircle(c); plotLine(l);

Perpotongan garis dengan lingkaran menghasilkan dua titik dan jumlah titik perpotongan.

\textgreater\{P1,P2,f\}=lineCircleIntersections(l,c);

\textgreater P1, P2, f

\begin{verbatim}
[4.64575,  -1.64575]
[-0.645751,  3.64575]
2
\end{verbatim}

\textgreater plotPoint(P1); plotPoint(P2):

\begin{figure}
\centering
\pandocbounded{\includegraphics[keepaspectratio]{images/Nazwa Yuan Adelia Putri_23030630095_Geometri-021.png}}
\caption{images/Nazwa\%20Yuan\%20Adelia\%20Putri\_23030630095\_Geometri-021.png}
\end{figure}

Hal yang sama pada Maxima.

\textgreater c \&= circleWithCenter(A,4) // lingkaran dengan pusat A jari-jari 4

\begin{verbatim}
                              [1, 0, 4]
\end{verbatim}

\textgreater l \&= lineThrough(B,C) // garis l melalui B dan C

\begin{verbatim}
                              [1, 1, 3]
\end{verbatim}

\textgreater\$lineCircleIntersections(l,c) \textbar{} radcan, // titik potong lingkaran c dan garis l

\[\left[ \left[ \sqrt{7}+2 , 1-\sqrt{7} \right]  , \left[ 2-\sqrt{7}
  , \sqrt{7}+1 \right]  \right] \]Akan ditunjukkan bahwa sudut-sudut yang menghadap bsuusr yang sama adalah sama besar.

\textgreater C=A+normalize({[}-2,-3{]})*4; plotPoint(C); plotSegment(P1,C); plotSegment(P2,C);

\textgreater degprint(computeAngle(P1,C,P2))

\begin{verbatim}
69°17'42.68''
\end{verbatim}

\textgreater C=A+normalize({[}-4,-3{]})*4; plotPoint(C); plotSegment(P1,C); plotSegment(P2,C);

\textgreater degprint(computeAngle(P1,C,P2))

\begin{verbatim}
69°17'42.68''
\end{verbatim}

\textgreater insimg;

\begin{figure}
\centering
\pandocbounded{\includegraphics[keepaspectratio]{images/Nazwa Yuan Adelia Putri_23030630095_Geometri-023.png}}
\caption{images/Nazwa\%20Yuan\%20Adelia\%20Putri\_23030630095\_Geometri-023.png}
\end{figure}

\section{Garis Sumbu}\label{garis-sumbu}

Berikut adalah langkah-langkah menggambar garis sumbu ruas garis AB:

\begin{enumerate}
\def\labelenumi{\arabic{enumi}.}
\item
  Gambar lingkaran dengan pusat A melalui B.
\item
  Gambar lingkaran dengan pusat B melalui A.
\item
  Tarik garis melallui kedua titik potong kedua lingkaran tersebut. Garis ini merupakan garis sumbu (melalui titik tengah dan tegak lurus) AB.
\end{enumerate}

\textgreater A={[}2,2{]}; B={[}-1,-2{]};

\textgreater c1=circleWithCenter(A,distance(A,B));

\textgreater c2=circleWithCenter(B,distance(A,B));

\textgreater\{P1,P2,f\}=circleCircleIntersections(c1,c2);

\textgreater l=lineThrough(P1,P2);

\textgreater setPlotRange(5); plotCircle(c1); plotCircle(c2);

\textgreater plotPoint(A); plotPoint(B); plotSegment(A,B); plotLine(l):

\begin{figure}
\centering
\pandocbounded{\includegraphics[keepaspectratio]{images/Nazwa Yuan Adelia Putri_23030630095_Geometri-024.png}}
\caption{images/Nazwa\%20Yuan\%20Adelia\%20Putri\_23030630095\_Geometri-024.png}
\end{figure}

Selanjutnya, kami melakukan hal yang sama di Maxima dengan koordinat umum.

\textgreater A \&= {[}a1,a2{]}; B \&= {[}b1,b2{]};

\textgreater c1 \&= circleWithCenter(A,distance(A,B));

\textgreater c2 \&= circleWithCenter(B,distance(A,B));

\textgreater P \&= circleCircleIntersections(c1,c2); P1 \&= P{[}1{]}; P2 \&= P{[}2{]};

Persamaan untuk persimpangan cukup rumit. Tetapi kita dapat menyederhanakannya, jika kita menyelesaikan untuk y.

\textgreater g \&= getLineEquation(lineThrough(P1,P2),x,y);

\textgreater\$solve(g,y)

\[\left[ y=\frac{-\left(2\,{\it b_1}-2\,{\it a_1}\right)\,x+{\it b_2}
 ^2+{\it b_1}^2-{\it a_2}^2-{\it a_1}^2}{2\,{\it b_2}-2\,{\it a_2}}
  \right] \]Ini memang sama dengan tegak lurus tengah, yang dihitung dengan cara yang sama sekali berbeda.

\textgreater\$solve(getLineEquation(middlePerpendicular(A,B),x,y),y)

\[\left[ y=\frac{-\left(2\,{\it b_1}-2\,{\it a_1}\right)\,x+{\it b_2}
 ^2+{\it b_1}^2-{\it a_2}^2-{\it a_1}^2}{2\,{\it b_2}-2\,{\it a_2}}
  \right] \]\textgreater h \&=getLineEquation(lineThrough(A,B),x,y);

\textgreater\$solve(h,y)

\[\left[ y=\frac{\left({\it b_2}-{\it a_2}\right)\,x-{\it a_1}\,
 {\it b_2}+{\it a_2}\,{\it b_1}}{{\it b_1}-{\it a_1}} \right] \]Perhatikan hasil kali gradien garis g dan h adalah:

\[\frac{-(b_1-a_1)}{(b_2-a_2)}\times \frac{(b_2-a_2)}{(b_1-a_1)} = -1.\]Artinya kedua garis tegak lurus.

\chapter{Contoh 3: Rumus Heron}\label{contoh-3-rumus-heron}

Rumus Heron menyatakan bahwa luas segitiga dengan panjang sisi-sisi a, b dan c adalah:

\[L = \sqrt{s(s-a)(s-b)(s-c)}\quad \text{ dengan } s=(a+b+c)/2,\]atau bisa ditulis dalam bentuk lain:

\[L = \frac{1}{4}\sqrt{(a+b+c)(b+c-a)(a+c-b)(a+b-c)}\]Untuk membuktikan hal ini kita misalkan C(0,0), B(a,0) dan A(x,y), b=AC, c=AB. Luas segitiga ABC adalah

\[L_{\triangle ABC}=\frac{1}{2}a\times y.\]Nilai y didapat dengan menyelesaikan sistem persamaan:

\[x^2+y^2=b^2, \quad (x-a)^2+y^2=c^2.\]\textgreater setPlotRange(-1,10,-1,8); plotPoint({[}0,0{]}, ``C(0,0)''); plotPoint({[}5.5,0{]}, ``B(a,0)''); \ldots{}\\
\textgreater{} plotPoint({[}7.5,6{]}, ``A(x,y)'');

\textgreater plotSegment({[}0,0{]},{[}5.5,0{]}, ``a'',25); plotSegment({[}5.5,0{]},{[}7.5,6{]},``c'',15); \ldots{}\\
\textgreater{} plotSegment({[}0,0{]},{[}7.5,6{]},``b'',25);

\textgreater plotSegment({[}7.5,6{]},{[}7.5,0{]},``t=y'',25):

\begin{figure}
\centering
\pandocbounded{\includegraphics[keepaspectratio]{images/Nazwa Yuan Adelia Putri_23030630095_Geometri-033.png}}
\caption{images/Nazwa\%20Yuan\%20Adelia\%20Putri\_23030630095\_Geometri-033.png}
\end{figure}

\textgreater\&assume(a\textgreater0); sol \&= solve({[}x\textsuperscript{2+y}2=b\textsuperscript{2,(x-a)}2+y\textsuperscript{2=c}2{]},{[}x,y{]})

\begin{verbatim}
                                  []
\end{verbatim}

Ektrak solusi dari y.

\textgreater ysol \&= y with sol{[}2{]}{[}2{]};\$'y=sqrt(factor(ysol\^{}2))

\begin{verbatim}
Maxima said:
part: invalid index of list or matrix.
 -- an error. To debug this try: debugmode(true);

Error in:
ysol &amp;= y with sol[2][2];$'y=sqrt(factor(ysol^2)) ...
                        ^
\end{verbatim}

Kami mendapatkan rumus Heron.

\textgreater function H(a,b,c) \&= sqrt(factor((ysol*a/2)\^{}2)); \$'H(a,b,c)=H(a,b,c)

\[H\left(a , b , \left[ 1 , 0 , 4 \right] \right)=\frac{a\,\left| 
 {\it ysol}\right| }{2}\]\textgreater\$'Luas=H(2,5,6) // luas segitiga dengan panjang sisi-sisi 2, 5, 6

\[{\it Luas}=\left| {\it ysol}\right| \]Tentu saja, setiap segitiga persegi panjang adalah kasus yang terkenal.

\textgreater H(3,4,5) //luas segitiga siku-siku dengan panjang sisi 3, 4, 5

\begin{verbatim}
Variable or function ysol not found.
Try "trace errors" to inspect local variables after errors.
H:
    useglobal; return a*abs(ysol)/2 
Error in:
H(3,4,5) //luas segitiga siku-siku dengan panjang sisi 3, 4, 5 ...
        ^
\end{verbatim}

Dan juga jelas, bahwa ini adalah segitiga dengan luas maksimal dan kedua sisi 3 dan 4.

\textgreater aspect (1.5); plot2d(\&H(3,4,x),1,7): // Kurva luas segitiga sengan panjang sisi 3, 4, x (1\textless= x \textless=7)

\begin{verbatim}
Variable or function ysol not found.
Error in expression: 3*abs(ysol)/2
%ploteval:
    y0=f$(x[1],args());
adaptiveevalone:
    s=%ploteval(g$,t;args());
Try "trace errors" to inspect local variables after errors.
plot2d:
    dw/n,dw/n^2,dw/n,auto;args());
\end{verbatim}

Kasus umum juga bisa digunakan.

\textgreater\$solve(diff(H(a,b,c)\^{}2,c)=0,c)

\begin{verbatim}
Maxima said:
diff: second argument must be a variable; found [1,0,4]
 -- an error. To debug this try: debugmode(true);

Error in:
$solve(diff(H(a,b,c)^2,c)=0,c) ...
                              ^
\end{verbatim}

Sekarang mari kita cari himpunan semua titik di mana b+c=d untuk suatu konstanta d.~Sudah diketahui bahwa ini adalah sebuah elips.

\textgreater s1 \&= subst(d-c,b,sol{[}2{]}); \$s1

\begin{verbatim}
Maxima said:
part: invalid index of list or matrix.
 -- an error. To debug this try: debugmode(true);

Error in:
s1 &amp;= subst(d-c,b,sol[2]); $s1 ...
                         ^
\end{verbatim}

Dan membuat fungsi-fungsi ini.

\textgreater function fx(a,c,d) \&= rhs(s1{[}1{]}); \$fx(a,c,d), function fy(a,c,d) \&= rhs(s1{[}2{]}); \$fy(a,c,d)

\[0\]\[0\]Sekarang kita dapat menggambar himpunan tersebut. Sisi b bervariasi dari 1 hingga 4. Sudah diketahui bahwa kita mendapatkan sebuah elips.

\textgreater aspect(1); plot2d(\&fx(3,x,5),\&fy(3,x,5),xmin=1,xmax=4,square=1):

\begin{figure}
\centering
\pandocbounded{\includegraphics[keepaspectratio]{images/Nazwa Yuan Adelia Putri_23030630095_Geometri-038.png}}
\caption{images/Nazwa\%20Yuan\%20Adelia\%20Putri\_23030630095\_Geometri-038.png}
\end{figure}

Kita dapat memeriksa persamaan umum untuk elips ini, yaitu

\[\frac{(x-x_m)^2}{u^2}+\frac{(y-y_m)}{v^2}=1,\]di mana (xm, ym) adalah pusat, dan u serta v adalah setengah sumbu.

\textgreater\$ratsimp((fx(a,c,d)-a/2)\textsuperscript{2/u}2+fy(a,c,d)\textsuperscript{2/v}2 with {[}u=d/2,v=sqrt(d\textsuperscript{2-a}2)/2{]})

\[\frac{a^2}{d^2}\]Kita melihat bahwa tinggi dan luas segitiga adalah maksimal untuk x=0. Dengan demikian, luas segitiga dengan a+b+c=d adalah maksimal, jika segitiga tersebut sama sisi. Kita ingin membuktikannya secara analitis.

\textgreater eqns \&= {[}diff(H(a,b,d-(a+b))\textsuperscript{2,a)=0,diff(H(a,b,d-(a+b))}2,b)=0{]}; \$eqns

\[\left[ \frac{a\,{\it ysol}^2}{2}=0 , 0=0 \right] \]Kita mendapatkan beberapa minima, yang termasuk dalam segitiga dengan satu sisi 0, dan solusi a = b = c = d / 3.

\textgreater\$solve(eqns,{[}a,b{]})

\[\left[ \left[ a=0 , b={\it \%r_1} \right]  \right] \]Ada juga metode Lagrange, yang memaksimalkan H(a,b,c)\^{}2 sehubungan dengan a+b+d=d.

\textgreater\&solve({[}diff(H(a,b,c)\textsuperscript{2,a)=la,diff(H(a,b,c)}2,b)=la, \ldots{}\\
\textgreater{} diff(H(a,b,c)\^{}2,c)=la,a+b+c=d{]},{[}a,b,c,la{]})

\begin{verbatim}
Maxima said:
diff: second argument must be a variable; found [1,0,4]
 -- an error. To debug this try: debugmode(true);

Error in:
... la,    diff(H(a,b,c)^2,c)=la,a+b+c=d],[a,b,c,la]) ...
                                                     ^
\end{verbatim}

Kita dapat membuat plot situasi

Pertama-tama, tetapkan titik-titik di Maxima.

\textgreater A \&= at({[}x,y{]},sol{[}2{]}); \$A

\begin{verbatim}
Maxima said:
part: invalid index of list or matrix.
 -- an error. To debug this try: debugmode(true);

Error in:
A &amp;= at([x,y],sol[2]); $A ...
                     ^
\end{verbatim}

\textgreater B \&= {[}0,0{]}; \$B, C \&= {[}a,0{]}; \$C

\[\left[ 0 , 0 \right] \]\[\left[ a , 0 \right] \]Kemudian, tetapkan kisaran plot, dan plot titik-titiknya.

\textgreater setPlotRange(0,5,-2,3); \ldots{}\\
\textgreater{} a=4; b=3; c=2; \ldots{}\\
\textgreater{} plotPoint(mxmeval(``B''),``B''); plotPoint(mxmeval(``C''),``C''); \ldots{}\\
\textgreater{} plotPoint(mxmeval(``A''),``A''):

\begin{verbatim}
Variable a1 not found!
Use global variables or parameters for string evaluation.
Error in Evaluate, superfluous characters found.
Try "trace errors" to inspect local variables after errors.
mxmeval:
    return evaluate(mxm(s));
Error in:
... otPoint(mxmeval("C"),"C"); plotPoint(mxmeval("A"),"A"): ...
                                                     ^
\end{verbatim}

Plot segmen-segmen tersebut.

\textgreater plotSegment(mxmeval(``A''),mxmeval(``C'')); \ldots{}\\
\textgreater{} plotSegment(mxmeval(``B''),mxmeval(``C'')); \ldots{}\\
\textgreater{} plotSegment(mxmeval(``B''),mxmeval(``A'')):

\begin{verbatim}
Variable a1 not found!
Use global variables or parameters for string evaluation.
Error in Evaluate, superfluous characters found.
Try "trace errors" to inspect local variables after errors.
mxmeval:
    return evaluate(mxm(s));
Error in:
plotSegment(mxmeval("A"),mxmeval("C")); plotSegment(mxmeval("B ...
                        ^
\end{verbatim}

Hitung garis tegak lurus tengah dalam Maxima.

\textgreater h \&= middlePerpendicular(A,B); g \&= middlePerpendicular(B,C);

Dan bagian tengah lingkar.

\textgreater U \&= lineIntersection(h,g);

Kita mendapatkan rumus untuk jari-jari lingkaran.

\textgreater\&assume(a\textgreater0,b\textgreater0,c\textgreater0); \$distance(U,B) \textbar{} radcan

\[\frac{\sqrt{{\it a_2}^2+{\it a_1}^2}\,\sqrt{{\it a_2}^2+{\it a_1}^2
 -2\,a\,{\it a_1}+a^2}}{2\,\left| {\it a_2}\right| }\]Mari kita tambahkan ini ke dalam plot.

\textgreater plotPoint(U()); \ldots{}\\
\textgreater{} plotCircle(circleWithCenter(mxmeval(``U''),mxmeval(``distance(U,C)''))):

\begin{verbatim}
Variable a2 not found!
Use global variables or parameters for string evaluation.
Error in ^
Error in expression: [a/2,(a2^2+a1^2-a*a1)/(2*a2)]
Error in:
plotPoint(U()); plotCircle(circleWithCenter(mxmeval("U"),mxmev ...
             ^
\end{verbatim}

Dengan menggunakan geometri, kami memperoleh rumus sederhana

untuk radius. Kita bisa mengecek, apakah hal ini benar dengan Maxima. Maxima akan memperhitungkannya hanya jika kita mengkuadratkannya.

\textgreater\$c\textsuperscript{2/sin(computeAngle(A,B,C))}2 \textbar{} factor

\[\left[ \frac{{\it a_2}^2+{\it a_1}^2}{{\it a_2}^2} , 0 , \frac{16\,
 \left({\it a_2}^2+{\it a_1}^2\right)}{{\it a_2}^2} \right] \]\# Contoh 4: Garis Euler dan Parabola

Garis Euler adalah garis yang ditentukan dari segitiga apa pun yang tidak sama sisi. Garis ini merupakan garis tengah segitiga, dan melewati beberapa titik penting yang ditentukan dari segitiga, termasuk ortosentrum, circumcentrum, centroid, titik Exeter, dan pusat lingkaran sembilan titik segitiga.

Sebagai demonstrasi, kami menghitung dan memplot garis Euler dalam sebuah segitiga.

Pertama, kita mendefinisikan sudut-sudut segitiga dalam Euler. Kami menggunakan definisi, yang terlihat dalam ekspresi simbolis.

\textgreater A::={[}-1,-1{]}; B::={[}2,0{]}; C::={[}1,2{]};

Untuk memplot objek geometris, kita menyiapkan area plot, dan menambahkan titik-titiknya. Semua plot objek geometris ditambahkan ke plot saat ini.

\textgreater setPlotRange(3); plotPoint(A,``A''); plotPoint(B,``B''); plotPoint(C,``C'');

Kita juga bisa menambahkan sisi-sisi segitiga.

\textgreater plotSegment(A,B,``\,``); plotSegment(B,C,''``); plotSegment(C,A,''\,``):

\begin{figure}
\centering
\pandocbounded{\includegraphics[keepaspectratio]{images/Nazwa Yuan Adelia Putri_23030630095_Geometri-047.png}}
\caption{images/Nazwa\%20Yuan\%20Adelia\%20Putri\_23030630095\_Geometri-047.png}
\end{figure}

Berikut ini adalah luas area segitiga, dengan menggunakan rumus determinan. Tentu saja, kita harus mengambil nilai absolut dari hasil ini.

\textgreater\$areaTriangle(A,B,C)

\[-\frac{7}{2}\]Kita dapat menghitung koefisien dari sisi c.

\textgreater c \&= lineThrough(A,B)

\begin{verbatim}
                            [- 1, 3, - 2]
\end{verbatim}

Dan juga mendapatkan formula untuk baris ini.

\textgreater\$getLineEquation(c,x,y)

\[3\,y-x=-2\]Untuk bentuk Hesse, kita perlu menentukan sebuah titik, sehingga titik tersebut berada di sisi positif dari bentuk Hesse. Memasukkan titik tersebut akan menghasilkan jarak positif ke garis.

\textgreater\$getHesseForm(c,x,y,C), \$at(\%,{[}x=C{[}1{]},y=C{[}2{]}{]})

\[\frac{3\,y-x+2}{\sqrt{10}}\]\[\frac{7}{\sqrt{10}}\]Sekarang kita menghitung keliling ABC.

\textgreater LL \&= circleThrough(A,B,C); \$getCircleEquation(LL,x,y)

\[\left(y-\frac{5}{14}\right)^2+\left(x-\frac{3}{14}\right)^2=\frac{
 325}{98}\]\textgreater O \&= getCircleCenter(LL); \$O

\[\left[ \frac{3}{14} , \frac{5}{14} \right] \]Plot lingkaran dan pusatnya. Cu dan U adalah simbolik. Kami mengevaluasi ekspresi ini untuk Euler.

\textgreater plotCircle(LL()); plotPoint(O(),``O''):

\begin{figure}
\centering
\pandocbounded{\includegraphics[keepaspectratio]{images/Nazwa Yuan Adelia Putri_23030630095_Geometri-054.png}}
\caption{images/Nazwa\%20Yuan\%20Adelia\%20Putri\_23030630095\_Geometri-054.png}
\end{figure}

Kita dapat menghitung perpotongan ketinggian di ABC (pusat ortosentrum) secara numerik dengan perintah berikut ini.

\textgreater H \&= lineIntersection(perpendicular(A,lineThrough(C,B)),\ldots{}\\
\textgreater{} perpendicular(B,lineThrough(A,C))); \$H

\[\left[ \frac{11}{7} , \frac{2}{7} \right] \]Sekarang kita dapat menghitung garis Euler dari segitiga tersebut.

\textgreater el \&= lineThrough(H,O); \$getLineEquation(el,x,y)

\[-\frac{19\,y}{14}-\frac{x}{14}=-\frac{1}{2}\]Tambahkan ke plot kami.

\textgreater plotPoint(H(),``H''); plotLine(el(),``Garis Euler''):

\begin{figure}
\centering
\pandocbounded{\includegraphics[keepaspectratio]{images/Nazwa Yuan Adelia Putri_23030630095_Geometri-057.png}}
\caption{images/Nazwa\%20Yuan\%20Adelia\%20Putri\_23030630095\_Geometri-057.png}
\end{figure}

Pusat gravitasi harus berada pada garis ini.

\textgreater M \&= (A+B+C)/3; \$getLineEquation(el,x,y) with

\[-\frac{1}{2}=-\frac{1}{2}\]\textgreater plotPoint(M(),``M''): // titik berat

\begin{figure}
\centering
\pandocbounded{\includegraphics[keepaspectratio]{images/Nazwa Yuan Adelia Putri_23030630095_Geometri-059.png}}
\caption{images/Nazwa\%20Yuan\%20Adelia\%20Putri\_23030630095\_Geometri-059.png}
\end{figure}

Teori mengatakan bahwa MH = 2*MO. Kita perlu menyederhanakan dengan radcan untuk mencapai hal ini.

\textgreater\$distance(M,H)/distance(M,O)\textbar radcan

\[2\]Fungsi-fungsi ini juga mencakup fungsi untuk sudut.

\textgreater\$computeAngle(A,C,B), degprint(\%())

\[\arccos \left(\frac{4}{\sqrt{5}\,\sqrt{13}}\right)\] 60°15'18.43'\,'

Persamaan untuk bagian tengah lingkaran tidak terlalu bagus.

\textgreater Q \&= lineIntersection(angleBisector(A,C,B),angleBisector(C,B,A))\textbar radcan; \$Q

\[\left[ \frac{\left(2^{\frac{3}{2}}+1\right)\,\sqrt{5}\,\sqrt{13}-15
 \,\sqrt{2}+3}{14} , \frac{\left(\sqrt{2}-3\right)\,\sqrt{5}\,\sqrt{
 13}+5\,2^{\frac{3}{2}}+5}{14} \right] \]Mari kita hitung juga ekspresi untuk jari-jari lingkaran yang tertulis.

\textgreater r \&= distance(Q,projectToLine(Q,lineThrough(A,B)))\textbar ratsimp; \$r

\[\frac{\sqrt{\left(-41\,\sqrt{2}-31\right)\,\sqrt{5}\,\sqrt{13}+115
 \,\sqrt{2}+614}}{7\,\sqrt{2}}\]\textgreater LD \&= circleWithCenter(Q,r); // Lingkaran dalam

Mari kita tambahkan ini ke dalam plot.

\textgreater color(5); plotCircle(LD()):

\begin{figure}
\centering
\pandocbounded{\includegraphics[keepaspectratio]{images/Nazwa Yuan Adelia Putri_23030630095_Geometri-064.png}}
\caption{images/Nazwa\%20Yuan\%20Adelia\%20Putri\_23030630095\_Geometri-064.png}
\end{figure}

\section{Parabola}\label{parabola}

Selanjutnya akan dicari persamaan tempat kedudukan titik-titik yang berjarak sama ke titik C dan ke garis AB.

\textgreater p \&= getHesseForm(lineThrough(A,B),x,y,C)-distance({[}x,y{]},C); \$p='0

\[\frac{3\,y-x+2}{\sqrt{10}}-\sqrt{\left(2-y\right)^2+\left(1-x
 \right)^2}=0\]Persamaan tersebut dapat digambar menjadi satu dengan gambar sebelumnya.

\textgreater plot2d(p,level=0,add=1,contourcolor=6):

\begin{figure}
\centering
\pandocbounded{\includegraphics[keepaspectratio]{images/Nazwa Yuan Adelia Putri_23030630095_Geometri-066.png}}
\caption{images/Nazwa\%20Yuan\%20Adelia\%20Putri\_23030630095\_Geometri-066.png}
\end{figure}

Ini seharusnya merupakan suatu fungsi, tetapi solver default Maxima hanya bisa menemukan solusinya jika kita mengkuadratkan persamaannya. Akibatnya, kita mendapatkan solusi palsu.

\textgreater akar \&= solve(getHesseForm(lineThrough(A,B),x,y,C)\textsuperscript{2-distance({[}x,y{]},C)}2,y)

\begin{verbatim}
        [y = - 3 x - sqrt(70) sqrt(9 - 2 x) + 26, 
                              y = - 3 x + sqrt(70) sqrt(9 - 2 x) + 26]
\end{verbatim}

Solusi pertama adalah

maxima: akar{[}1{]}

Dengan menambahkan solusi pertama ke dalam plot, maka akan terlihat bahwa ini adalah jalur yang kita cari. Teori mengatakan bahwa ini adalah sebuah parabola yang diputar.

\textgreater plot2d(\&rhs(akar{[}1{]}),add=1):

\begin{figure}
\centering
\pandocbounded{\includegraphics[keepaspectratio]{images/Nazwa Yuan Adelia Putri_23030630095_Geometri-067.png}}
\caption{images/Nazwa\%20Yuan\%20Adelia\%20Putri\_23030630095\_Geometri-067.png}
\end{figure}

\textgreater function g(x) \&= rhs(akar{[}1{]}); \$'g(x)= g(x)// fungsi yang mendefinisikan kurva di atas

\[g\left(x\right)=-3\,x-\sqrt{70}\,\sqrt{9-2\,x}+26\]\textgreater T \&={[}-1, g(-1){]}; // ambil sebarang titik pada kurva tersebut

\textgreater dTC \&= distance(T,C); \$fullratsimp(dTC), \$float(\%) // jarak T ke C

\[\sqrt{1503-54\,\sqrt{11}\,\sqrt{70}}\]\[2.135605779339061\]\textgreater U \&= projectToLine(T,lineThrough(A,B)); \$U // proyeksi T pada garis AB

\[\left[ \frac{80-3\,\sqrt{11}\,\sqrt{70}}{10} , \frac{20-\sqrt{11}\,
 \sqrt{70}}{10} \right] \]\textgreater dU2AB \&= distance(T,U); \$fullratsimp(dU2AB), \$float(\%) // jatak T ke AB

\[\sqrt{1503-54\,\sqrt{11}\,\sqrt{70}}\]\[2.135605779339061\]Ternyata jarak T ke C sama dengan jarak T ke AB. Coba Anda pilih titik T yang lain dan ulangi perhitungan-perhitungan di atas untuk menunjukkan bahwa hasilnya juga sama.

\chapter{Contoh 5: Trigonometri Rasional}\label{contoh-5-trigonometri-rasional}

Ini terinspirasi dari sebuah ceramah N.J. Wildberger. Dalam bukunya ``Proporsi Ilahi'', Wildberger mengusulkan untuk mengganti gagasan klasik tentang jarak dan sudut dengan kuadransi dan penyebaran. Dengan menggunakan ini, memang memungkinkan untuk menghindari fungsi trigonometri dalam banyak contoh, dan tetap ``rasional''.

Berikut ini, saya akan memperkenalkan konsep-konsep tersebut, dan memecahkan beberapa masalah. Saya menggunakan komputasi simbolik Maxima di sini, yang menyembunyikan keuntungan utama dari trigonometri rasional yaitu komputasi dapat dilakukan dengan kertas dan pensil saja. Anda dipersilakan untuk memeriksa hasilnya tanpa komputer.

Intinya adalah bahwa komputasi rasional simbolik sering kali memberikan hasil yang sederhana. Sebaliknya, trigonometri klasik menghasilkan hasil trigonometri yang rumit, yang dievaluasi dengan pendekatan numerik saja.

\textgreater load geometry;

Untuk pengenalan pertama, kita menggunakan segitiga persegi panjang dengan proporsi Mesir yang terkenal 3, 4, dan 5. Perintah berikut ini adalah perintah Euler untuk memplot geometri bidang yang terdapat pada file Euler ``geometry.e''.

\textgreater C\&:={[}0,0{]}; A\&:={[}4,0{]}; B\&:={[}0,3{]}; \ldots{}\\
\textgreater{} setPlotRange(-1,5,-1,5); \ldots{}\\
\textgreater{} plotPoint(A,``A''); plotPoint(B,``B''); plotPoint(C,``C''); \ldots{}\\
\textgreater{} plotSegment(B,A,``c''); plotSegment(A,C,``b''); plotSegment(C,B,``a''); \ldots{}\\
\textgreater{} insimg(30);

\begin{figure}
\centering
\pandocbounded{\includegraphics[keepaspectratio]{images/Nazwa Yuan Adelia Putri_23030630095_Geometri-074.png}}
\caption{images/Nazwa\%20Yuan\%20Adelia\%20Putri\_23030630095\_Geometri-074.png}
\end{figure}

Tentunya,

\[\sin(w_a)=\frac{a}{c},\]di mana wa adalah sudut di A. Cara biasa untuk menghitung sudut ini, adalah dengan mengambil kebalikan dari fungsi sinus. Hasilnya adalah sudut yang tidak dapat dicerna, yang hanya dapat dicetak kira-kira.

\textgreater wa := arcsin(3/5); degprint(wa)

\begin{verbatim}
36°52'11.63''
\end{verbatim}

Trigonometri rasional mencoba menghindari hal ini.

Gagasan pertama trigonometri rasional adalah kuadrat, yang menggantikan jarak. Sebenarnya, ini hanyalah jarak yang dikuadratkan. Berikut ini, a, b, dan c menunjukkan kuadran sisi-sisinya.

Teorema Pythogoras menjadi a + b = c.

\textgreater a \&= 3\^{}2; b \&= 4\^{}2; c \&= 5\^{}2; \&a+b=c

\begin{verbatim}
                               25 = 25
\end{verbatim}

Gagasan kedua dari trigonometri rasional adalah penyebaran. Penyebaran mengukur bukaan di antara garis-garis. Ini adalah 0, jika garis-garisnya sejajar, dan 1, jika garis-garisnya persegi panjang. Ini adalah kuadrat dari sinus sudut antara dua garis.

Penyebaran garis AB dan AC pada gambar di atas didefinisikan sebagai

\[s_a = \sin(\alpha)^2 = \frac{a}{c},\]di mana a dan c adalah kuadran dari segitiga persegi panjang dengan satu sudut di A.

\textgreater sa \&= a/c; \$sa

\[\frac{9}{25}\]Tentu saja, hal ini lebih mudah dihitung daripada sudut. Tetapi Anda kehilangan sifat bahwa sudut dapat ditambahkan dengan mudah.

Tentu saja, kita bisa mengonversi nilai perkiraan kita untuk sudut wa ke sprad, dan mencetaknya sebagai pecahan.

\textgreater fracprint(sin(wa)\^{}2)

\begin{verbatim}
9/25
\end{verbatim}

Hukum kosinus trgonometri klasik diterjemahkan ke dalam ``hukum silang'' berikut ini.

\[(c+b-a)^2 = 4 b c \, (1-s_a)\]Di sini a, b, dan c adalah kuadran dari sisi-sisi segitiga, dan sa adalah penyebaran di sudut A. Sisi a, seperti biasa, berlawanan dengan sudut A.

Hukum-hukum ini diimplementasikan dalam file geometri.e yang kita masukkan ke dalam Euler.

\textgreater\$crosslaw(aa,bb,cc,saa)

\[\left[ \left({\it bb}-{\it aa}+\frac{7}{6}\right)^2 , \left(
 {\it bb}-{\it aa}+\frac{7}{6}\right)^2 , \left({\it bb}-{\it aa}+
 \frac{5}{3\,\sqrt{2}}\right)^2 \right] =\left[ \frac{14\,{\it bb}\,
 \left(1-{\it saa}\right)}{3} , \frac{14\,{\it bb}\,\left(1-{\it saa}
 \right)}{3} , \frac{5\,2^{\frac{3}{2}}\,{\it bb}\,\left(1-{\it saa}
 \right)}{3} \right] \]Dalam kasus kami, kami mendapatkan

\textgreater\$crosslaw(a,b,c,sa)

\[1024=1024\]Mari kita gunakan crosslaw ini untuk mencari sebaran di A. Untuk melakukannya, kita buat crosslaw untuk kuadran a, b, dan c, dan selesaikan untuk sebaran sa yang tidak diketahui.

Anda bisa melakukan ini dengan tangan dengan mudah, tapi saya menggunakan Maxima. Tentu saja, kita mendapatkan hasil yang sudah kita dapatkan.

\textgreater\$crosslaw(a,b,c,x), \$solve(\%,x)

\[1024=1600\,\left(1-x\right)\]\[\left[ x=\frac{9}{25} \right] \]Kita sudah mengetahui hal ini. Definisi penyebaran adalah kasus khusus dari crosslaw.

Kita juga dapat menyelesaikannya untuk a, b, c secara umum. Hasilnya adalah sebuah rumus yang menghitung penyebaran sudut sebuah segitiga dengan kuadran ketiga sisinya.

\textgreater\$solve(crosslaw(aa,bb,cc,x),x)

\[\left[ \left[ \frac{168\,{\it bb}\,x+36\,{\it bb}^2+\left(-72\,
 {\it aa}-84\right)\,{\it bb}+36\,{\it aa}^2-84\,{\it aa}+49}{36} , 
 \frac{168\,{\it bb}\,x+36\,{\it bb}^2+\left(-72\,{\it aa}-84\right)
 \,{\it bb}+36\,{\it aa}^2-84\,{\it aa}+49}{36} , \frac{15\,2^{\frac{
 5}{2}}\,{\it bb}\,x+18\,{\it bb}^2+\left(-36\,{\it aa}-15\,2^{\frac{
 3}{2}}\right)\,{\it bb}+18\,{\it aa}^2-15\,2^{\frac{3}{2}}\,{\it aa}
 +25}{18} \right] =0 \right] \]Kita dapat membuat sebuah fungsi dari hasil tersebut. Fungsi seperti itu sudah didefinisikan dalam file geometry.e dari Euler.

\textgreater\$spread(a,b,c)

\[\frac{9}{25}\]Sebagai contoh, kita dapat menggunakannya untuk menghitung sudut segitiga dengan sisi

\[a, \quad a, \quad \frac{4a}{7}\]Hasilnya adalah rasional, yang tidak mudah didapat jika kita menggunakan trigonometri klasik.

\textgreater\$spread(a,a,4*a/7)

\[\frac{6}{7}\]Ini adalah sudut dalam derajat.

\textgreater degprint(arcsin(sqrt(6/7)))

\begin{verbatim}
67°47'32.44''
\end{verbatim}

\section{Contoh Lain}\label{contoh-lain}

Sekarang, mari kita coba contoh yang lebih lanjut.

Kita tentukan tiga sudut segitiga sebagai berikut.

\textgreater A\&:={[}1,2{]}; B\&:={[}4,3{]}; C\&:={[}0,4{]}; \ldots{}\\
\textgreater{} setPlotRange(-1,5,1,7); \ldots{}\\
\textgreater{} plotPoint(A,``A''); plotPoint(B,``B''); plotPoint(C,``C''); \ldots{}\\
\textgreater{} plotSegment(B,A,``c''); plotSegment(A,C,``b''); plotSegment(C,B,``a''); \ldots{}\\
\textgreater{} insimg;

\begin{figure}
\centering
\pandocbounded{\includegraphics[keepaspectratio]{images/Nazwa Yuan Adelia Putri_23030630095_Geometri-087.png}}
\caption{images/Nazwa\%20Yuan\%20Adelia\%20Putri\_23030630095\_Geometri-087.png}
\end{figure}

Dengan menggunakan Pythogoras, mudah untuk menghitung jarak antara dua titik. Pertama-tama saya menggunakan jarak fungsi dari file Euler untuk geometri. Jarak fungsi menggunakan geometri klasik.

\textgreater\$distance(A,B)

\[\sqrt{10}\]Euler juga memiliki fungsi untuk kuadranan antara dua titik.

Pada contoh berikut, karena c+b bukan a, maka segitiga tersebut tidak berbentuk persegi panjang.

\textgreater c \&= quad(A,B); \$c, b \&= quad(A,C); \$b, a \&= quad(B,C); \$a,

\[10\]\[5\]\[17\]Pertama, mari kita menghitung sudut tradisional. Fungsi computeAngle menggunakan metode yang biasa berdasarkan hasil kali titik dari dua vektor. Hasilnya adalah beberapa perkiraan titik mengambang.

\[A=<1,2>\quad B=<4,3>,\quad C=<0,4>\]\[\mathbf{a}=C-B=<-4,1>,\quad \mathbf{c}=A-B=<-3,-1>,\quad \beta=\angle ABC\]\[\mathbf{a}.\mathbf{c}=|\mathbf{a}|.|\mathbf{c}|\cos \beta\]\[\cos \angle ABC =\cos\beta=\frac{\mathbf{a}.\mathbf{c}}{|\mathbf{a}|.|\mathbf{c}|}=\frac{12-1}{\sqrt{17}\sqrt{10}}=\frac{11}{\sqrt{17}\sqrt{10}}\]\textgreater wb \&= computeAngle(A,B,C); \$wb, \$(wb/pi*180)()

\[\arccos \left(\frac{11}{\sqrt{10}\,\sqrt{17}}\right)\] 32.4711922908

Dengan menggunakan pensil dan kertas, kita dapat melakukan hal yang sama dengan hukum silang. Kita masukkan kuadran a, b, dan c ke dalam hukum silang dan selesaikan untuk x.

\textgreater\$crosslaw(a,b,c,x), \$solve(\%,x), //(b+c-a)\^{}=4b.c(1-x)

\[4=200\,\left(1-x\right)\]\[\left[ x=\frac{49}{50} \right] \]Itulah yang dilakukan oleh fungsi spread yang didefinisikan dalam ``geometry.e''.

\textgreater sb \&= spread(b,a,c); \$sb

\[\frac{49}{170}\]Maxima mendapatkan hasil yang sama dengan menggunakan trigonometri biasa, jika kita memaksakannya. Ia menyelesaikan suku sin(arccos(\ldots)) menjadi hasil pecahan. Sebagian besar siswa tidak dapat melakukan ini.

\textgreater\$sin(computeAngle(A,B,C))\^{}2

\[\frac{49}{170}\]Setelah kita memiliki penyebaran di B, kita dapat menghitung tinggi ha di sisi a. Ingatlah bahwa

\[s_b=\frac{h_a}{c}\]menurut definisi.

\textgreater ha \&= c*sb; \$ha

\[\frac{49}{17}\]Gambar berikut ini dibuat dengan program geometri C.a.R., yang dapat menggambar kuadran dan penyebaran.

image: (20) Rational\_Geometry\_CaR.png

Menurut definisi, panjang ha adalah akar kuadrat dari kuadrannya.

\textgreater\$sqrt(ha)

\[\frac{7}{\sqrt{17}}\]Sekarang kita dapat menghitung luas segitiga. Jangan lupa, bahwa kita berurusan dengan kuadran!

\textgreater\$sqrt(ha)*sqrt(a)/2

\[\frac{7}{2}\]Rumus penentu yang biasa menghasilkan hasil yang sama.

\textgreater\$areaTriangle(B,A,C)

\[\frac{7}{2}\]\#\# The Heron Formula

Sekarang, mari kita selesaikan masalah ini secara umum!

\textgreater\&remvalue(a,b,c,sb,ha);

Pertama-tama kita menghitung penyebaran di B untuk segitiga dengan sisi a, b, dan c.~Kemudian kita menghitung luas kuadrat (``quadrea''?), memfaktorkannya dengan Maxima, dan kita mendapatkan rumus Heron yang terkenal.

Memang, hal ini sulit dilakukan dengan pensil dan kertas.

\textgreater\$spread(b\textsuperscript{2,c}2,a\^{}2), \$factor(\%*c\textsuperscript{2*a}2/4)

\[\frac{-c^4-\left(-2\,b^2-2\,a^2\right)\,c^2-b^4+2\,a^2\,b^2-a^4}{4
 \,a^2\,c^2}\]\[\frac{\left(-c+b+a\right)\,\left(c-b+a\right)\,\left(c+b-a\right)\,
 \left(c+b+a\right)}{16}\]\#\# The Triple Spread Rule

Kerugian dari spread adalah bahwa mereka tidak lagi hanya menambahkan sudut seperti.

Namun, tiga spread dari sebuah segitiga memenuhi aturan ``triple spread'' berikut ini.

\textgreater\&remvalue(sa,sb,sc); \$triplespread(sa,sb,sc)

\[\left({\it sc}+{\it sb}+{\it sa}\right)^2=2\,\left({\it sc}^2+
 {\it sb}^2+{\it sa}^2\right)+4\,{\it sa}\,{\it sb}\,{\it sc}\]Aturan ini berlaku untuk tiga sudut yang berjumlah 180°.

\[\alpha+\beta+\gamma=\pi\]Karena spread dari

\[\alpha, \pi-\alpha\]sama, aturan triple spread juga benar, jika

\[\alpha+\beta=\gamma\]Karena penyebaran sudut negatifnya sama, aturan penyebaran tiga kali lipat juga berlaku, jika

\[\alpha+\beta+\gamma=0\]Contohnya, kita bisa menghitung penyebaran sudut 60°. Hasilnya adalah 3/4. Namun, persamaan ini memiliki solusi kedua, di mana semua penyebarannya adalah 0.

\textgreater\$solve(triplespread(x,x,x),x)

\[\left[ x=\frac{3}{4} , x=0 \right] \]Penyebaran 90° jelas adalah 1. Jika dua sudut ditambahkan ke 90°, penyebarannya akan menyelesaikan persamaan penyebaran tiga dengan a, b, 1. Dengan perhitungan berikut, kita mendapatkan a + b = 1.

\textgreater\$triplespread(x,y,1), \$solve(\%,x)

\[\left(y+x+1\right)^2=2\,\left(y^2+x^2+1\right)+4\,x\,y\]\[\left[ x=1-y \right] \]Karena penyebaran 180°-t sama dengan penyebaran t, rumus penyebaran tiga kali lipat juga berlaku, jika satu sudut adalah jumlah atau selisih dari dua sudut lainnya.

Jadi kita dapat menemukan penyebaran sudut dua kali lipat. Perhatikan bahwa ada dua solusi lagi. Kita jadikan ini sebuah fungsi.

\textgreater\$solve(triplespread(a,a,x),x), function doublespread(a) \&= factor(rhs(\%{[}1{]}))

\[\left[ x=4\,a-4\,a^2 , x=0 \right] \]\\
- 4 (a - 1) a

\section{Angle Bisectors}\label{angle-bisectors}

Ini adalah situasi yang sudah kita ketahui.

\textgreater{} 8C\&:={[}0,0{]}; A\&:={[}4,0{]}; B\&:={[}0,3{]}; \ldots{}\\
\textgreater{} setPlotRange(-1,5,-1,5); \ldots{}\\
\textgreater{} plotPoint(A,``A''); plotPoint(B,``B''); plotPoint(C,``C''); \ldots{}\\
\textgreater{} plotSegment(B,A,``c''); plotSegment(A,C,``b''); plotSegment(C,B,``a''); \ldots{}\\
\textgreater{} insimg;

\begin{verbatim}
Commands must be separated by semicolon or comma!
Found: &amp;:=[0,0]; A&amp;:=[4,0]; B&amp;:=[0,3]; setPlotRange(-1,5,-1,5); plotPoint(A,"A"); plotPoint(B,"B"); plotPoint(C,"C"); plotSegment(B,A,"c"); plotSegment(A,C,"b"); plotSegment(C,B,"a"); insimg; (character 38)
You can disable this in the Options menu.
Error in:
 8C&amp;:=[0,0]; A&amp;:=[4,0]; B&amp;:=[0,3]; setPlotRange(-1,5,-1,5); pl ...
   ^
\end{verbatim}

Mari kita hitung panjang garis bagi sudut di A. Tetapi kita ingin menyelesaikannya untuk a, b, c secara umum.

\textgreater\&remvalue(a,b,c);

Jadi, pertama-tama kita menghitung penyebaran sudut yang dibelah dua di A, menggunakan rumus penyebaran tiga.

Masalah dengan rumus ini muncul lagi. Rumus ini memiliki dua solusi. Kita harus memilih salah satu yang benar. Solusi lainnya mengacu pada sudut terbagi dua 180°-wa.

\textgreater\$triplespread(x,x,a/(a+b)), \$solve(\%,x), sa2 \&= rhs(\%{[}1{]}); \$sa2

\[\left(2\,x+\frac{a}{b+a}\right)^2=2\,\left(2\,x^2+\frac{a^2}{\left(
 b+a\right)^2}\right)+\frac{4\,a\,x^2}{b+a}\]\[\left[ x=\frac{-\sqrt{b}\,\sqrt{b+a}+b+a}{2\,b+2\,a} , x=\frac{
 \sqrt{b}\,\sqrt{b+a}+b+a}{2\,b+2\,a} \right] \]\[\frac{-\sqrt{b}\,\sqrt{b+a}+b+a}{2\,b+2\,a}\]Mari kita periksa persegi panjang Mesir.

\textgreater\$sa2 with {[}a=3\textsuperscript{2,b=4}2{]}

\[\frac{1}{10}\]Kita bisa mencetak sudut dalam Euler, setelah mentransfer penyebaran ke radian.

\textgreater wa2 := arcsin(sqrt(1/10)); degprint(wa2)

\begin{verbatim}
18°26'5.82''
\end{verbatim}

Titik P adalah perpotongan garis bagi sudut dengan sumbu y.

\textgreater P := {[}0,tan(wa2)*4{]}

\begin{verbatim}
[0,  1.33333]
\end{verbatim}

\textgreater plotPoint(P,``P''); plotSegment(A,P):

\begin{figure}
\centering
\pandocbounded{\includegraphics[keepaspectratio]{images/Nazwa Yuan Adelia Putri_23030630095_Geometri-121.png}}
\caption{images/Nazwa\%20Yuan\%20Adelia\%20Putri\_23030630095\_Geometri-121.png}
\end{figure}

Mari kita periksa sudut-sudutnya dalam contoh spesifik kita.

\textgreater computeAngle(C,A,P), computeAngle(P,A,B)

\begin{verbatim}
1.69515132134
2.87534060444
\end{verbatim}

Sekarang kita hitung panjang garis bagi AP.

Kita menggunakan teorema sinus dalam segitiga APC. Teorema ini menyatakan bahwa

\[\frac{BC}{\sin(w_a)} = \frac{AC}{\sin(w_b)} = \frac{AB}{\sin(w_c)}\]berlaku dalam segitiga apa pun. Kuadratkan, ini diterjemahkan ke dalam apa yang disebut ``hukum penyebaran''

\[\frac{a}{s_a} = \frac{b}{s_b} = \frac{c}{s_b}\]where a,b,c denote qudrances.

di mana a, b, c menunjukkan kuadrannya.

Karena spread CPA adalah 1-sa2, kita mendapatkan bisa/1=b/(1-sa2) dan bisa menghitung bisa (kuadran dari pembagi sudut).

\textgreater\&factor(ratsimp(b/(1-sa2))); bisa \&= \%; \$bisa

\[\frac{2\,b\,\left(b+a\right)}{\sqrt{b}\,\sqrt{b+a}+b+a}\]Mari kita periksa rumus ini untuk nilai-nilai Mesir kita.

\textgreater sqrt(mxmeval(``at(bisa,{[}a=3\textsuperscript{2,b=4}2{]})'')), distance(A,P)

\begin{verbatim}
4.21637021356
1.20185042515
\end{verbatim}

Kita juga dapat menghitung P dengan menggunakan rumus penyebaran.

\textgreater py\&=factor(ratsimp(sa2*bisa)); \$py

\[-\frac{b\,\left(\sqrt{b}\,\sqrt{b+a}-b-a\right)}{\sqrt{b}\,\sqrt{b+
 a}+b+a}\]Nilainya sama dengan yang kita dapatkan dengan rumus trigonometri.

\textgreater sqrt(mxmeval(``at(py,{[}a=3\textsuperscript{2,b=4}2{]})''))

\begin{verbatim}
1.33333333333
\end{verbatim}

\section{The Chord Angle}\label{the-chord-angle}

Lihatlah situasi berikut ini.

\textgreater setPlotRange(1.2); \ldots{}\\
\textgreater{} color(1); plotCircle(circleWithCenter({[}0,0{]},1)); \ldots{}\\
\textgreater{} A:={[}cos(1),sin(1){]}; B:={[}cos(2),sin(2){]}; C:={[}cos(6),sin(6){]}; \ldots{}\\
\textgreater{} plotPoint(A,``A''); plotPoint(B,``B''); plotPoint(C,``C''); \ldots{}\\
\textgreater{} color(3); plotSegment(A,B,``c''); plotSegment(A,C,``b''); plotSegment(C,B,``a''); \ldots{}\\
\textgreater{} color(1); O:={[}0,0{]}; plotPoint(O,``0''); \ldots{}\\
\textgreater{} plotSegment(A,O); plotSegment(B,O); plotSegment(C,O,``r''); \ldots{}\\
\textgreater{} insimg;

\begin{figure}
\centering
\pandocbounded{\includegraphics[keepaspectratio]{images/Nazwa Yuan Adelia Putri_23030630095_Geometri-126.png}}
\caption{images/Nazwa\%20Yuan\%20Adelia\%20Putri\_23030630095\_Geometri-126.png}
\end{figure}

Kita dapat menggunakan Maxima untuk menyelesaikan rumus penyebaran tiga untuk sudut-sudut di pusat O untuk r. Dengan demikian kita mendapatkan rumus untuk jari-jari kuadrat dari pericircle dalam hal kuadran sisi-sisinya.

Kali ini, Maxima menghasilkan beberapa angka nol yang rumit, yang kita abaikan.

\textgreater\&remvalue(a,b,c,r); // hapus nilai-nilai sebelumnya untuk perhitungan baru

\textgreater rabc \&= rhs(solve(triplespread(spread(b,r,r),spread(a,r,r),spread(c,r,r)),r){[}4{]}); \$rabc

\[-\frac{a\,b\,c}{c^2-2\,b\,c+a\,\left(-2\,c-2\,b\right)+b^2+a^2}\]Kita dapat menjadikannya sebuah fungsi Euler.

\textgreater function periradius(a,b,c) \&= rabc;

Mari kita periksa hasilnya untuk poin A, B, C.

\textgreater a:=quadrance(B,C); b:=quadrance(A,C); c:=quadrance(A,B);

Radiusnya memang 1.

\textgreater periradius(a,b,c)

\begin{verbatim}
1
\end{verbatim}

Faktanya adalah, bahwa penyebaran CBA hanya bergantung pada b dan c. Ini adalah teorema sudut akor.

\textgreater\$spread(b,a,c)*rabc \textbar{} ratsimp

\[\frac{b}{4}\]Faktanya, penyebarannya adalah b/(4r), dan kita melihat bahwa sudut chord b adalah setengah dari sudut tengah.

\textgreater\$doublespread(b/(4*r))-spread(b,r,r) \textbar{} ratsimp

\[0\]\# Contoh 6: Jarak Minimal pada Bidang

\section{Preliminary remark}\label{preliminary-remark}

Fungsi yang, pada sebuah titik M pada bidang, menetapkan jarak AM antara titik tetap A dan M, memiliki garis-garis tingkat yang cukup sederhana: lingkaran yang berpusat di A.

\textgreater\&remvalue();

\textgreater A={[}-1,-1{]};

\textgreater function d1(x,y):=sqrt((x-A{[}1{]})\textsuperscript{2+(y-A{[}2{]})}2)

\textgreater fcontour(``d1'',xmin=-2,xmax=0,ymin=-2,ymax=0,hue=1, \ldots{}\\
\textgreater{} title=``If you see ellipses, please set your window square''):

\begin{figure}
\centering
\pandocbounded{\includegraphics[keepaspectratio]{images/Nazwa Yuan Adelia Putri_23030630095_Geometri-130.png}}
\caption{images/Nazwa\%20Yuan\%20Adelia\%20Putri\_23030630095\_Geometri-130.png}
\end{figure}

dan grafiknya juga cukup sederhana: bagian atas kerucut:

\textgreater plot3d(``d1'',xmin=-2,xmax=0,ymin=-2,ymax=0):

\begin{figure}
\centering
\pandocbounded{\includegraphics[keepaspectratio]{images/Nazwa Yuan Adelia Putri_23030630095_Geometri-131.png}}
\caption{images/Nazwa\%20Yuan\%20Adelia\%20Putri\_23030630095\_Geometri-131.png}
\end{figure}

Tentu saja nilai minimum 0 diperoleh dalam A.

\section{Dua titik}\label{dua-titik}

Sekarang kita lihat fungsi MA+MB di mana A dan B adalah dua titik (tetap). Ini adalah ``fakta yang terkenal'' bahwa kurva level adalah elips, titik fokusnya adalah A dan B; kecuali AB minimum yang konstan pada segmen {[}AB{]}:

\textgreater B={[}1,-1{]};

\textgreater function d2(x,y):=d1(x,y)+sqrt((x-B{[}1{]})\textsuperscript{2+(y-B{[}2{]})}2)

\textgreater fcontour(``d2'',xmin=-2,xmax=2,ymin=-3,ymax=1,hue=1):

\begin{figure}
\centering
\pandocbounded{\includegraphics[keepaspectratio]{images/Nazwa Yuan Adelia Putri_23030630095_Geometri-132.png}}
\caption{images/Nazwa\%20Yuan\%20Adelia\%20Putri\_23030630095\_Geometri-132.png}
\end{figure}

Grafiknya lebih menarik:

\textgreater plot3d(``d2'',xmin=-2,xmax=2,ymin=-3,ymax=1):

\begin{figure}
\centering
\pandocbounded{\includegraphics[keepaspectratio]{images/Nazwa Yuan Adelia Putri_23030630095_Geometri-133.png}}
\caption{images/Nazwa\%20Yuan\%20Adelia\%20Putri\_23030630095\_Geometri-133.png}
\end{figure}

Pembatasan pada garis (AB) lebih terkenal:

\textgreater plot2d(``abs(x+1)+abs(x-1)'',xmin=-3,xmax=3):

\begin{figure}
\centering
\pandocbounded{\includegraphics[keepaspectratio]{images/Nazwa Yuan Adelia Putri_23030630095_Geometri-134.png}}
\caption{images/Nazwa\%20Yuan\%20Adelia\%20Putri\_23030630095\_Geometri-134.png}
\end{figure}

\section{Three points}\label{three-points}

Sekarang, hal-hal menjadi kurang sederhana: Hal ini sedikit kurang dikenal bahwa MA+MB+MC mencapai minimumnya pada satu titik di bidang, tetapi untuk menentukannya tidak sesederhana itu:

\begin{enumerate}
\def\labelenumi{\arabic{enumi})}
\tightlist
\item
  Jika salah satu sudut segitiga ABC lebih dari 120° (katakanlah di A), maka minimum dicapai pada titik ini (katakanlah AB+AC).
\end{enumerate}

Contoh:

\textgreater C={[}-4,1{]};

\textgreater function d3(x,y):=d2(x,y)+sqrt((x-C{[}1{]})\textsuperscript{2+(y-C{[}2{]})}2)

\textgreater plot3d(``d3'',xmin=-5,xmax=3,ymin=-4,ymax=4);

\textgreater insimg;

\begin{figure}
\centering
\pandocbounded{\includegraphics[keepaspectratio]{images/Nazwa Yuan Adelia Putri_23030630095_Geometri-135.png}}
\caption{images/Nazwa\%20Yuan\%20Adelia\%20Putri\_23030630095\_Geometri-135.png}
\end{figure}

\textgreater fcontour(``d3'',xmin=-4,xmax=1,ymin=-2,ymax=2,hue=1,title=``The minimum is on A'');

\textgreater P=(A\_B\_C\_A)'; plot2d(P{[}1{]},P{[}2{]},add=1,color=12);

\textgreater insimg;

\begin{figure}
\centering
\pandocbounded{\includegraphics[keepaspectratio]{images/Nazwa Yuan Adelia Putri_23030630095_Geometri-136.png}}
\caption{images/Nazwa\%20Yuan\%20Adelia\%20Putri\_23030630095\_Geometri-136.png}
\end{figure}

\begin{enumerate}
\def\labelenumi{\arabic{enumi})}
\setcounter{enumi}{1}
\tightlist
\item
  Tetapi jika semua sudut segitiga ABC kurang dari 120°, minimumnya adalah pada titik F di bagian dalam segitiga, yang merupakan satu-satunya titik yang melihat sisi-sisi ABC dengan sudut yang sama (masing-masing 120°):
\end{enumerate}

\textgreater C={[}-0.5,1{]};

\textgreater plot3d(``d3'',xmin=-2,xmax=2,ymin=-2,ymax=2):

\begin{figure}
\centering
\pandocbounded{\includegraphics[keepaspectratio]{images/Nazwa Yuan Adelia Putri_23030630095_Geometri-137.png}}
\caption{images/Nazwa\%20Yuan\%20Adelia\%20Putri\_23030630095\_Geometri-137.png}
\end{figure}

\textgreater fcontour(``d3'',xmin=-2,xmax=2,ymin=-2,ymax=2,hue=1,title=``The Fermat point'');

\textgreater P=(A\_B\_C\_A)'; plot2d(P{[}1{]},P{[}2{]},add=1,color=12);

\textgreater insimg;

\begin{figure}
\centering
\pandocbounded{\includegraphics[keepaspectratio]{images/Nazwa Yuan Adelia Putri_23030630095_Geometri-138.png}}
\caption{images/Nazwa\%20Yuan\%20Adelia\%20Putri\_23030630095\_Geometri-138.png}
\end{figure}

Merupakan kegiatan yang menarik untuk merealisasikan gambar di atas dengan perangkat lunak geometri; sebagai contoh, saya tahu sebuah perangkat lunak yang ditulis dalam bahasa Java yang memiliki instruksi ``garis kontur''\ldots{}

Semua hal di atas telah ditemukan oleh seorang hakim Perancis bernama Pierre de Fermat; dia menulis surat kepada para ahli lain seperti pendeta Marin Mersenne dan Blaise Pascal yang bekerja di bagian pajak penghasilan. Jadi titik unik F sedemikian rupa sehingga FA+FB+FC minimal, disebut titik Fermat dari segitiga. Namun tampaknya beberapa tahun sebelumnya, Torriccelli dari Italia telah menemukan titik ini sebelum Fermat menemukannya! Bagaimanapun juga, tradisinya adalah mencatat titik F ini\ldots{}

\section{Four points}\label{four-points}

Langkah selanjutnya adalah menambahkan titik ke-4 D dan mencoba meminimumkan MA+MB+MC+MD; misalkan Anda adalah operator TV kabel dan ingin menemukan di bidang mana Anda harus meletakkan antena sehingga Anda dapat memberi makan empat desa dan menggunakan panjang kabel sesedikit mungkin!

\textgreater D={[}1,1{]};

\textgreater function d4(x,y):=d3(x,y)+sqrt((x-D{[}1{]})\textsuperscript{2+(y-D{[}2{]})}2)

\textgreater plot3d(``d4'',xmin=-1.5,xmax=1.5,ymin=-1.5,ymax=1.5):

\begin{figure}
\centering
\pandocbounded{\includegraphics[keepaspectratio]{images/Nazwa Yuan Adelia Putri_23030630095_Geometri-139.png}}
\caption{images/Nazwa\%20Yuan\%20Adelia\%20Putri\_23030630095\_Geometri-139.png}
\end{figure}

\textgreater fcontour(``d4'',xmin=-1.5,xmax=1.5,ymin=-1.5,ymax=1.5,hue=1);

\textgreater P=(A\_B\_C\_D)'; plot2d(P{[}1{]},P{[}2{]},points=1,add=1,color=12);

\textgreater insimg;

\begin{figure}
\centering
\pandocbounded{\includegraphics[keepaspectratio]{images/Nazwa Yuan Adelia Putri_23030630095_Geometri-140.png}}
\caption{images/Nazwa\%20Yuan\%20Adelia\%20Putri\_23030630095\_Geometri-140.png}
\end{figure}

Masih ada nilai minimum dan tidak ada simpul A, B, C, maupun D:

\textgreater function f(x):=d4(x{[}1{]},x{[}2{]})

\textgreater neldermin(``f'',{[}0.2,0.2{]})

\begin{verbatim}
[0.142858,  0.142857]
\end{verbatim}

Tampaknya dalam kasus ini, koordinat titik optimal adalah rasional atau mendekati rasional\ldots{}

Karena ABCD adalah sebuah bujur sangkar, kita berharap bahwa titik optimalnya adalah pusat dari ABCD:

\textgreater C={[}-1,1{]};

\textgreater plot3d(``d4'',xmin=-1,xmax=1,ymin=-1,ymax=1):

\begin{figure}
\centering
\pandocbounded{\includegraphics[keepaspectratio]{images/Nazwa Yuan Adelia Putri_23030630095_Geometri-141.png}}
\caption{images/Nazwa\%20Yuan\%20Adelia\%20Putri\_23030630095\_Geometri-141.png}
\end{figure}

\textgreater fcontour(``d4'',xmin=-1.5,xmax=1.5,ymin=-1.5,ymax=1.5,hue=1);

\textgreater P=(A\_B\_C\_D)'; plot2d(P{[}1{]},P{[}2{]},add=1,color=12,points=1);

\textgreater insimg;

\begin{figure}
\centering
\pandocbounded{\includegraphics[keepaspectratio]{images/Nazwa Yuan Adelia Putri_23030630095_Geometri-142.png}}
\caption{images/Nazwa\%20Yuan\%20Adelia\%20Putri\_23030630095\_Geometri-142.png}
\end{figure}

\chapter{Contoh 7: Bola Dandelin dengan Povray}\label{contoh-7-bola-dandelin-dengan-povray}

Anda dapat menjalankan demonstrasi ini, jika Anda telah menginstal Povray, dan pvengine.exe pada jalur program.

Pertama, kita menghitung jari-jari bola.

Jika Anda melihat gambar di bawah ini, Anda dapat melihat bahwa kita membutuhkan dua lingkaran yang menyentuh dua garis yang membentuk kerucut, dan satu garis yang membentuk bidang yang memotong kerucut.

Kita menggunakan file geometri.e dari Euler untuk hal ini.

\textgreater load geometry;

Pertama, dua garis yang membentuk kerucut.

\textgreater g1 \&= lineThrough({[}0,0{]},{[}1,a{]})

\begin{verbatim}
                             [- a, 1, 0]
\end{verbatim}

\textgreater g2 \&= lineThrough({[}0,0{]},{[}-1,a{]})

\begin{verbatim}
                            [- a, - 1, 0]
\end{verbatim}

Kemudian baris ketiga.

\textgreater g \&= lineThrough({[}-1,0{]},{[}1,1{]})

\begin{verbatim}
                             [- 1, 2, 1]
\end{verbatim}

Kami merencanakan semuanya sejauh ini.

\textgreater setPlotRange(-1,1,0,2);

\textgreater color(black); plotLine(g(),``\,``)

\textgreater a:=2; color(blue); plotLine(g1(),``\,``), plotLine(g2(),''\,``):

\begin{figure}
\centering
\pandocbounded{\includegraphics[keepaspectratio]{images/Nazwa Yuan Adelia Putri_23030630095_Geometri-143.png}}
\caption{images/Nazwa\%20Yuan\%20Adelia\%20Putri\_23030630095\_Geometri-143.png}
\end{figure}

Sekarang, kita ambil titik umum pada sumbu y.

\textgreater P \&= {[}0,u{]}

\begin{verbatim}
                                [0, u]
\end{verbatim}

Hitung jarak ke g1.

\textgreater d1 \&= distance(P,projectToLine(P,g1)); \$d1

\[\sqrt{\left(\frac{a^2\,u}{a^2+1}-u\right)^2+\frac{a^2\,u^2}{\left(a
 ^2+1\right)^2}}\]Hitung jarak ke g.

\textgreater d \&= distance(P,projectToLine(P,g)); \$d

\[\sqrt{\left(\frac{u+2}{5}-u\right)^2+\frac{\left(2\,u-1\right)^2}{
 25}}\]Dan temukan pusat kedua lingkaran, yang jaraknya sama.

\textgreater sol \&= solve(d1\textsuperscript{2=d}2,u); \$sol

\[\left[ u=\frac{-\sqrt{5}\,\sqrt{a^2+1}+2\,a^2+2}{4\,a^2-1} , u=
 \frac{\sqrt{5}\,\sqrt{a^2+1}+2\,a^2+2}{4\,a^2-1} \right] \]Ada dua solusi.

Kami mengevaluasi solusi simbolis, dan menemukan kedua pusat, dan kedua jarak.

\textgreater u := sol()

\begin{verbatim}
[0.333333,  1]
\end{verbatim}

\textgreater dd := d()

\begin{verbatim}
[0.149071,  0.447214]
\end{verbatim}

Plot lingkaran-lingkaran tersebut ke dalam gambar.

\textgreater color(red);

\textgreater plotCircle(circleWithCenter({[}0,u{[}1{]}{]},dd{[}1{]}),``\,``);

\textgreater plotCircle(circleWithCenter({[}0,u{[}2{]}{]},dd{[}2{]}),``\,``);

\textgreater insimg;

\begin{figure}
\centering
\pandocbounded{\includegraphics[keepaspectratio]{images/Nazwa Yuan Adelia Putri_23030630095_Geometri-147.png}}
\caption{images/Nazwa\%20Yuan\%20Adelia\%20Putri\_23030630095\_Geometri-147.png}
\end{figure}

\section{Plot with Povray}\label{plot-with-povray}

Selanjutnya kita plot semuanya dengan Povray. Perhatikan bahwa Anda mengubah perintah apapun pada urutan perintah Povray berikut ini, dan jalankan kembali semua perintah dengan Shift-Return.

Pertama kita memuat fungsi povray.

\textgreater load povray;

\textgreater defaultpovray=``C:\textbackslash Program Files\textbackslash POV-Ray\textbackslash v3.7\textbackslash bin\textbackslash pvengine.exe''

\begin{verbatim}
C:\Program Files\POV-Ray\v3.7\bin\pvengine.exe
\end{verbatim}

Kami menyiapkan pemandangan dengan tepat.

\textgreater povstart(zoom=11,center={[}0,0,0.5{]},height=10°,angle=140°);

Selanjutnya kita tulis kedua bola tersebut ke file Povray.

\textgreater writeln(povsphere({[}0,0,u{[}1{]}{]},dd{[}1{]},povlook(red)));

\textgreater writeln(povsphere({[}0,0,u{[}2{]}{]},dd{[}2{]},povlook(red)));

Dan kerucutnya, transparan.

\textgreater writeln(povcone({[}0,0,0{]},0,{[}0,0,a{]},1,povlook(lightgray,1)));

Kami menghasilkan bidang yang terbatas pada kerucut.

\textgreater gp=g();

\textgreater pc=povcone({[}0,0,0{]},0,{[}0,0,a{]},1,``\,``);

\textgreater vp={[}gp{[}1{]},0,gp{[}2{]}{]}; dp=gp{[}3{]};

\textgreater writeln(povplane(vp,dp,povlook(blue,0.5),pc));

Sekarang kita menghasilkan dua titik pada lingkaran, di mana bola menyentuh kerucut.

\textgreater function turnz(v) := return

\textgreater P1=projectToLine({[}0,u{[}1{]}{]},g1()); P1=turnz({[}P1{[}1{]},0,P1{[}2{]}{]});

\textgreater writeln(povpoint(P1,povlook(yellow)));

\textgreater P2=projectToLine({[}0,u{[}2{]}{]},g1()); P2=turnz({[}P2{[}1{]},0,P2{[}2{]}{]});

\textgreater writeln(povpoint(P2,povlook(yellow)));

Kemudian, kita menghasilkan dua titik di mana bola-bola tersebut menyentuh bidang. Ini adalah fokus elips.

\textgreater P3=projectToLine({[}0,u{[}1{]}{]},g()); P3={[}P3{[}1{]},0,P3{[}2{]}{]};

\textgreater writeln(povpoint(P3,povlook(yellow)));

\textgreater P4=projectToLine({[}0,u{[}2{]}{]},g()); P4={[}P4{[}1{]},0,P4{[}2{]}{]};

\textgreater writeln(povpoint(P4,povlook(yellow)));

Selanjutnya kita menghitung perpotongan P1P2 dengan bidang.

\textgreater t1=scalp(vp,P1)-dp; t2=scalp(vp,P2)-dp; P5=P1+t1/(t1-t2)*(P2-P1);

\textgreater writeln(povpoint(P5,povlook(yellow)));

Kami menghubungkan titik-titik dengan segmen garis.

\textgreater writeln(povsegment(P1,P2,povlook(yellow)));

\textgreater writeln(povsegment(P5,P3,povlook(yellow)));

\textgreater writeln(povsegment(P5,P4,povlook(yellow)));

Sekarang, kita menghasilkan pita abu-abu, di mana bola-bola menyentuh kerucut.

\textgreater pcw=povcone({[}0,0,0{]},0,{[}0,0,a{]},1.01);

\textgreater pc1=povcylinder({[}0,0,P1{[}3{]}-defaultpointsize/2{]},{[}0,0,P1{[}3{]}+defaultpointsize/2{]},1);

\textgreater writeln(povintersection({[}pcw,pc1{]},povlook(gray)));

\textgreater pc2=povcylinder({[}0,0,P2{[}3{]}-defaultpointsize/2{]},{[}0,0,P2{[}3{]}+defaultpointsize/2{]},1);

\textgreater writeln(povintersection({[}pcw,pc2{]},povlook(gray)));

Mulai program Povray.

\textgreater povend();

\begin{figure}
\centering
\pandocbounded{\includegraphics[keepaspectratio]{images/Nazwa Yuan Adelia Putri_23030630095_Geometri-148.png}}
\caption{images/Nazwa\%20Yuan\%20Adelia\%20Putri\_23030630095\_Geometri-148.png}
\end{figure}

Untuk mendapatkan Anaglyph ini, kita harus memasukkan semuanya ke dalam fungsi scene. Fungsi ini akan digunakan dua kali nanti.

\textgreater function scene () \ldots{}

\begin{verbatim}
global a,u,dd,g,g1,defaultpointsize;
writeln(povsphere([0,0,u[1]],dd[1],povlook(red)));
writeln(povsphere([0,0,u[2]],dd[2],povlook(red)));
writeln(povcone([0,0,0],0,[0,0,a],1,povlook(lightgray,1)));
gp=g();
pc=povcone([0,0,0],0,[0,0,a],1,"");
vp=[gp[1],0,gp[2]]; dp=gp[3];
writeln(povplane(vp,dp,povlook(blue,0.5),pc));
P1=projectToLine([0,u[1]],g1()); P1=turnz([P1[1],0,P1[2]]);
writeln(povpoint(P1,povlook(yellow)));
P2=projectToLine([0,u[2]],g1()); P2=turnz([P2[1],0,P2[2]]);
writeln(povpoint(P2,povlook(yellow)));
P3=projectToLine([0,u[1]],g()); P3=[P3[1],0,P3[2]];
writeln(povpoint(P3,povlook(yellow)));
P4=projectToLine([0,u[2]],g()); P4=[P4[1],0,P4[2]];
writeln(povpoint(P4,povlook(yellow)));
t1=scalp(vp,P1)-dp; t2=scalp(vp,P2)-dp; P5=P1+t1/(t1-t2)*(P2-P1);
writeln(povpoint(P5,povlook(yellow)));
writeln(povsegment(P1,P2,povlook(yellow)));
writeln(povsegment(P5,P3,povlook(yellow)));
writeln(povsegment(P5,P4,povlook(yellow)));
pcw=povcone([0,0,0],0,[0,0,a],1.01);
pc1=povcylinder([0,0,P1[3]-defaultpointsize/2],[0,0,P1[3]+defaultpointsize/2],1);
writeln(povintersection([pcw,pc1],povlook(gray)));
pc2=povcylinder([0,0,P2[3]-defaultpointsize/2],[0,0,P2[3]+defaultpointsize/2],1);
writeln(povintersection([pcw,pc2],povlook(gray)));
endfunction
\end{verbatim}

Anda memerlukan kacamata merah/sian untuk mengapresiasi efek berikut ini.

\textgreater povanaglyph(``scene'',zoom=11,center={[}0,0,0.5{]},height=10°,angle=140°);

\begin{figure}
\centering
\pandocbounded{\includegraphics[keepaspectratio]{images/Nazwa Yuan Adelia Putri_23030630095_Geometri-149.png}}
\caption{images/Nazwa\%20Yuan\%20Adelia\%20Putri\_23030630095\_Geometri-149.png}
\end{figure}

\chapter{Contoh 8: Geometri Bumi}\label{contoh-8-geometri-bumi}

Pada buku catatan ini, kita ingin melakukan beberapa komputasi bola. Fungsi-fungsi tersebut terdapat pada file ``spherical.e'' pada folder contoh. Kita perlu memuat file tersebut terlebih dahulu.

\textgreater load ``spherical.e'';

Untuk memasukkan posisi geografis, kita menggunakan vektor dengan dua koordinat dalam radian (utara dan timur, nilai negatif untuk selatan dan barat). Berikut ini adalah koordinat untuk Kampus FMIPA UNY.

\textgreater FMIPA={[}rad(-7,-46.467),rad(110,23.05){]}

\begin{verbatim}
[-0.13569,  1.92657]
\end{verbatim}

Anda dapat mencetak posisi ini dengan sposprint (cetak posisi bola).

\textgreater sposprint(FMIPA) // posisi garis lintang dan garis bujur FMIPA UNY

\begin{verbatim}
S 7°46.467' E 110°23.050'
\end{verbatim}

Mari kita tambahkan dua kota lagi, Solo dan Semarang.

\textgreater Solo={[}rad(-7,-34.333),rad(110,49.683){]}; Semarang={[}rad(-6,-59.05),rad(110,24.533){]};

\textgreater sposprint(Solo), sposprint(Semarang),

\begin{verbatim}
S 7°34.333' E 110°49.683'
S 6°59.050' E 110°24.533'
\end{verbatim}

Pertama, kita menghitung vektor dari satu titik ke titik lainnya pada bola ideal. Vektor ini adalah {[}heading, jarak{]} dalam radian. Untuk menghitung jarak di bumi, kita kalikan dengan jari-jari bumi pada garis lintang 7°.

\textgreater br=svector(FMIPA,Solo); degprint(br{[}1{]}), br{[}2{]}*rearth(7°)-\textgreater km // perkiraan jarak FMIPA-Solo

\begin{verbatim}
65°20'26.60''
53.8945384608
\end{verbatim}

Ini adalah perkiraan yang baik. Rutinitas berikut ini menggunakan perkiraan yang lebih baik lagi. Pada jarak yang begitu pendek, hasilnya hampir sama.

\textgreater esdist(FMIPA,Semarang)-\textgreater'' km'' // perkiraan jarak FMIPA-Semarang

\begin{verbatim}
Commands must be separated by semicolon or comma!
Found:  // perkiraan jarak FMIPA-Semarang (character 32)
You can disable this in the Options menu.
Error in:
esdist(FMIPA,Semarang)-&gt;" km" // perkiraan jarak FMIPA-Semaran ...
                             ^
\end{verbatim}

Ada fungsi untuk judul, dengan mempertimbangkan bentuk elips bumi. Sekali lagi, kami mencetak dengan cara yang canggih.

\textgreater sdegprint(esdir(FMIPA,Solo))

\begin{verbatim}
     65.34°
\end{verbatim}

Sudut segitiga melebihi 180° pada bola.

\textgreater asum=sangle(Solo,FMIPA,Semarang)+sangle(FMIPA,Solo,Semarang)+sangle(FMIPA,Semarang,Solo); degprint(asum)

\begin{verbatim}
180°0'10.77''
\end{verbatim}

Ini bisa digunakan untuk menghitung luas segitiga. Catatan: Untuk segitiga kecil, cara ini tidak akurat karena kesalahan pengurangan dalam asum-pi.

\textgreater(asum-pi)*rearth(48°)\^{}2-\textgreater'' km\^{}2'' // perkiraan luas segitiga FMIPA-Solo-Semarang

\begin{verbatim}
Commands must be separated by semicolon or comma!
Found:  // perkiraan luas segitiga FMIPA-Solo-Semarang (character 32)
You can disable this in the Options menu.
Error in:
(asum-pi)*rearth(48°)^2-&gt;" km^2" // perkiraan luas segitiga FM ...
                                ^
\end{verbatim}

Ada sebuah fungsi untuk hal ini, yang menggunakan garis lintang rata-rata segitiga untuk menghitung radius bumi, dan menangani kesalahan pembulatan untuk segitiga yang sangat kecil.

\textgreater esarea(Solo,FMIPA,Semarang)-\textgreater'' km\^{}2'', //perkiraan yang sama dengan fungsi esarea()

\begin{verbatim}
2123.64310526 km^2
\end{verbatim}

Kita juga dapat menambahkan vektor ke posisi. Sebuah vektor berisi arah dan jarak, keduanya dalam radian. Untuk mendapatkan sebuah vektor, kita menggunakan svector. Untuk menambahkan sebuah vektor ke sebuah posisi, kita menggunakan saddvector.

\textgreater v=svector(FMIPA,Solo); sposprint(saddvector(FMIPA,v)), sposprint(Solo),

\begin{verbatim}
S 7°34.333' E 110°49.683'
S 7°34.333' E 110°49.683'
\end{verbatim}

Fungsi-fungsi ini mengasumsikan bola yang ideal. Hal yang sama di bumi.

\textgreater sposprint(esadd(FMIPA,esdir(FMIPA,Solo),esdist(FMIPA,Solo))), sposprint(Solo),

\begin{verbatim}
S 7°34.333' E 110°49.683'
S 7°34.333' E 110°49.683'
\end{verbatim}

Mari kita beralih ke contoh yang lebih besar, Tugu Jogja dan Monas Jakarta (menggunakan Google Earth untuk menemukan koordinatnya).

\textgreater Tugu={[}-7.7833°,110.3661°{]}; Monas={[}-6.175°,106.811944°{]};

\textgreater sposprint(Tugu), sposprint(Monas)

\begin{verbatim}
S 7°46.998' E 110°21.966'
S 6°10.500' E 106°48.717'
\end{verbatim}

Menurut Google Earth, jaraknya adalah 429,66 km. Kami mendapatkan perkiraan yang bagus.

\textgreater esdist(Tugu,Monas)-\textgreater'' km'' // perkiraan jarak Tugu Jogja - Monas Jakarta

\begin{verbatim}
Commands must be separated by semicolon or comma!
Found:  // perkiraan jarak Tugu Jogja - Monas Jakarta (character 32)
You can disable this in the Options menu.
Error in:
esdist(Tugu,Monas)-&gt;" km" // perkiraan jarak Tugu Jogja - Mona ...
                         ^
\end{verbatim}

Judulnya sama dengan yang dihitung di Google Earth.

\textgreater degprint(esdir(Tugu,Monas))

\begin{verbatim}
294°17'2.85''
\end{verbatim}

Namun demikian, kita tidak lagi mendapatkan posisi target yang tepat, jika kita menambahkan arah dan jarak ke posisi semula. Hal ini terjadi, karena kita tidak menghitung fungsi inversi secara tepat, tetapi mengambil perkiraan radius bumi di sepanjang jalur.

\textgreater sposprint(esadd(Tugu,esdir(Tugu,Monas),esdist(Tugu,Monas)))

\begin{verbatim}
S 6°10.500' E 106°48.717'
\end{verbatim}

Namun demikian, kesalahannya tidak besar.

\textgreater sposprint(Monas),

\begin{verbatim}
S 6°10.500' E 106°48.717'
\end{verbatim}

Tentu saja, kita tidak bisa berlayar dengan arah yang sama dari satu tujuan ke tujuan lainnya, jika kita ingin mengambil jalur terpendek. Bayangkan, Anda terbang ke arah NE mulai dari titik mana pun di bumi. Kemudian Anda akan berputar ke kutub utara. Lingkaran besar tidak mengikuti arah yang konstan!

Perhitungan berikut ini menunjukkan bahwa kita akan melenceng dari tujuan yang benar, jika kita menggunakan arah yang sama selama perjalanan.

\textgreater dist=esdist(Tugu,Monas); hd=esdir(Tugu,Monas);

Sekarang kita tambahkan 10 kali sepersepuluh dari jarak tersebut, dengan menggunakan arah menuju Monas, kita akan sampai di Tugu.

\textgreater p=Tugu; loop 1 to 10; p=esadd(p,hd,dist/10); end;

Hasilnya jauh berbeda.

\textgreater sposprint(p), skmprint(esdist(p,Monas))

\begin{verbatim}
S 6°11.250' E 106°48.372'
     1.529km
\end{verbatim}

Sebagai contoh lain, mari kita ambil dua titik di bumi pada garis lintang yang sama.

\textgreater P1={[}30°,10°{]}; P2={[}30°,50°{]};

Jalur terpendek dari P1 ke P2 bukanlah lingkaran lintang 30°, tetapi jalur yang lebih pendek yang dimulai 10° lebih jauh ke utara di P1.

\textgreater sdegprint(esdir(P1,P2))

\begin{verbatim}
     79.69°
\end{verbatim}

Namun, jika kita mengikuti pembacaan kompas ini, kita akan berputar ke kutub utara! Jadi, kita harus menyesuaikan arah kita di sepanjang jalan. Untuk tujuan kasar, kita menyesuaikannya pada 1/10 dari jarak total.

\textgreater p=P1; dist=esdist(P1,P2); \ldots{}\\
\textgreater{} loop 1 to 10; dir=esdir(p,P2); sdegprint(dir), p=esadd(p,dir,dist/10); end;

\begin{verbatim}
     79.69°
     81.67°
     83.71°
     85.78°
     87.89°
     90.00°
     92.12°
     94.22°
     96.29°
     98.33°
\end{verbatim}

Jaraknya tidak tepat, karena kita akan menambahkan sedikit kesalahan, jika kita mengikuti judul yang sama terlalu lama.

\textgreater skmprint(esdist(p,P2))

\begin{verbatim}
     0.203km
\end{verbatim}

Kita akan mendapatkan perkiraan yang baik, jika kita menyesuaikan arah setiap 1/100 dari total jarak dari Tugu ke Monas.

\textgreater p=Tugu; dist=esdist(Tugu,Monas); \ldots{}\\
\textgreater{} loop 1 to 100; p=esadd(p,esdir(p,Monas),dist/100); end;

\textgreater skmprint(esdist(p,Monas))

\begin{verbatim}
     0.000km
\end{verbatim}

Untuk keperluan navigasi, kita bisa mendapatkan urutan posisi GPS di sepanjang Bundaran Hotel Indonesia menuju Monas dengan fungsi navigate.

\textgreater load spherical; v=navigate(Tugu,Monas,10); \ldots{}\\
\textgreater{} loop 1 to rows(v); sposprint(v{[}\#{]}), end;

\begin{verbatim}
S 7°46.998' E 110°21.966'
S 7°37.422' E 110°0.573'
S 7°27.829' E 109°39.196'
S 7°18.219' E 109°17.834'
S 7°8.592' E 108°56.488'
S 6°58.948' E 108°35.157'
S 6°49.289' E 108°13.841'
S 6°39.614' E 107°52.539'
S 6°29.924' E 107°31.251'
S 6°20.219' E 107°9.977'
S 6°10.500' E 106°48.717'
\end{verbatim}

Kami menulis sebuah fungsi, yang memplot bumi, dua posisi, dan posisi di antaranya.

\textgreater function testplot \ldots{}

\begin{verbatim}
useglobal;
plotearth;
plotpos(Tugu,"Tugu Jogja"); plotpos(Monas,"Tugu Monas");
plotposline(v);
endfunction
\end{verbatim}

Sekarang rencanakan semuanya.

\textgreater plot3d(``testplot'',angle=25, height=6,\textgreater own,\textgreater user,zoom=4):

\begin{figure}
\centering
\pandocbounded{\includegraphics[keepaspectratio]{images/Nazwa Yuan Adelia Putri_23030630095_Geometri-150.png}}
\caption{images/Nazwa\%20Yuan\%20Adelia\%20Putri\_23030630095\_Geometri-150.png}
\end{figure}

Atau gunakan plot3d untuk mendapatkan tampilan anaglyph. Ini terlihat sangat bagus dengan kacamata merah/cyan.

\textgreater plot3d(``testplot'',angle=25,height=6,distance=5,own=1,anaglyph=1,zoom=4):

\begin{figure}
\centering
\pandocbounded{\includegraphics[keepaspectratio]{images/Nazwa Yuan Adelia Putri_23030630095_Geometri-151.png}}
\caption{images/Nazwa\%20Yuan\%20Adelia\%20Putri\_23030630095\_Geometri-151.png}
\end{figure}

\chapter{Latihan}\label{latihan-1}

\begin{enumerate}
\def\labelenumi{\arabic{enumi}.}
\tightlist
\item
  Gambarlah segi-n beraturan jika diketahui titik pusat O, n, dan jarak titik pusat ke titik-titik sudut segi-n tersebut (jari-jari lingkaran luar segi-n), r.
\end{enumerate}

Petunjuk:

\begin{itemize}
\item
  Besar sudut pusat yang menghadap masing-masing sisi segi-n adalah
\item
  (360/n).
\item
  Titik-titik sudut segi-n merupakan perpotongan lingkaran luar segi-n
\item
  dan garis-garis yang melalui pusat dan saling membentuk sudut sebesar
\item
  kelipatan (360/n).
\item
  Untuk n ganjil, pilih salah satu titik sudut adalah di atas.
\item
  Untuk n genap, pilih 2 titik di kanan dan kiri lurus dengan titik
\item
  pusat.
\item
  Anda dapat menggambar segi-3, 4, 5, 6, 7, dst beraturan.
\end{itemize}

Penyelesaian :

\textgreater load geometry

\begin{verbatim}
Numerical and symbolic geometry.
\end{verbatim}

\textgreater setPlotRange(-3.5,3.5,-3.5,3.5);

\textgreater A={[}-2,-2{]}; plotPoint(A,``A'');

\textgreater B={[}2,-2{]}; plotPoint(B,``B'');

\textgreater C={[}0,3{]}; plotPoint(C,``C'');

\textgreater plotSegment(A,B,``c'');

\textgreater plotSegment(B,C,``a'');

\textgreater plotSegment(A,C,``b'');

\textgreater aspect(1):

\begin{figure}
\centering
\pandocbounded{\includegraphics[keepaspectratio]{images/Nazwa Yuan Adelia Putri_23030630095_Geometri-152.png}}
\caption{images/Nazwa\%20Yuan\%20Adelia\%20Putri\_23030630095\_Geometri-152.png}
\end{figure}

\textgreater c=circleThrough(A,B,C);

\textgreater R=getCircleRadius(c);

\textgreater O=getCircleCenter(c);

\textgreater plotPoint(O,``O'');

\textgreater l=angleBisector(A,C,B);

\textgreater color(2); plotLine(l); color(1);

\textgreater plotCircle(c,``Lingkaran luar segitiga ABC''):

\begin{figure}
\centering
\pandocbounded{\includegraphics[keepaspectratio]{images/Nazwa Yuan Adelia Putri_23030630095_Geometri-153.png}}
\caption{images/Nazwa\%20Yuan\%20Adelia\%20Putri\_23030630095\_Geometri-153.png}
\end{figure}

\begin{enumerate}
\def\labelenumi{\arabic{enumi}.}
\setcounter{enumi}{1}
\tightlist
\item
  Gambarlah suatu parabola yang melalui 3 titik yang diketahui.
\end{enumerate}

Petunjuk:

\begin{itemize}
\item
  Misalkan persamaan parabolanya y= ax\^{}2+bx+c.
\item
  Substitusikan koordinat titik-titik yang diketahui ke persamaan tersebut.
\item
  Selesaikan SPL yang terbentuk untuk mendapatkan nilai-nilai a, b, c.
\end{itemize}

Penyelesaian :

\textgreater load geometry;

\textgreater setPlotRange(5); P={[}2,0{]}; Q={[}4,0{]}; R={[}0,-4{]};

\textgreater plotPoint(P,``P''); plotPoint(Q,``Q''); plotPoint(R,``R''):

\begin{figure}
\centering
\pandocbounded{\includegraphics[keepaspectratio]{images/Nazwa Yuan Adelia Putri_23030630095_Geometri-154.png}}
\caption{images/Nazwa\%20Yuan\%20Adelia\%20Putri\_23030630095\_Geometri-154.png}
\end{figure}

\textgreater sol \&= solve({[}a+b=-c,16*a+4*b=-c,c=-4{]},{[}a,b,c{]})

\begin{verbatim}
                     [[a = - 1, b = 5, c = - 4]]
\end{verbatim}

Sehingga didapat nilai a = -1, b = 5 dan c = -4

\textgreater function y\&=-x\^{}2+5*x-4

\begin{verbatim}
                               2
                            - x  + 5 x - 4
\end{verbatim}

\textgreater plot2d(``-x\^{}2+5*x-4'',-5,5,-5,5):

\begin{figure}
\centering
\pandocbounded{\includegraphics[keepaspectratio]{images/Nazwa Yuan Adelia Putri_23030630095_Geometri-155.png}}
\caption{images/Nazwa\%20Yuan\%20Adelia\%20Putri\_23030630095\_Geometri-155.png}
\end{figure}

\begin{enumerate}
\def\labelenumi{\arabic{enumi}.}
\setcounter{enumi}{2}
\item
  Gambarlah suatu segi-4 yang diketahui keempat titik sudutnya, misalnya A, B, C, D.

  \begin{itemize}
  \item
    Tentukan apakah segi-4 tersebut merupakan segi-4 garis singgung (sisinya-sisintya merupakan garis singgung lingkaran yang sama yakni lingkaran dalam segi-4 tersebut).
  \item
    Suatu segi-4 merupakan segi-4 garis singgung apabila keempat garis bagi sudutnya bertemu di satu titik.
  \item
    Jika segi-4 tersebut merupakan segi-4 garis singgung, gambar lingkaran dalamnya.
  \item
    Tunjukkan bahwa syarat suatu segi-4 merupakan segi-4 garis singgung apabila hasil kali panjang sisi-sisi yang berhadapan sama.
  \end{itemize}
\end{enumerate}

Penyelesaian :

\textgreater load geometry

\begin{verbatim}
Numerical and symbolic geometry.
\end{verbatim}

\textgreater setPlotRange(-4.5,4.5,-4.5,4.5);

\textgreater A={[}-3,-3{]}; plotPoint(A,``A'');

\textgreater B={[}3,-3{]}; plotPoint(B,``B'');

\textgreater C={[}3,3{]}; plotPoint(C,``C'');

\textgreater D={[}-3,3{]}; plotPoint(D,``D'');

\textgreater plotSegment(A,B,``\,``);

\textgreater plotSegment(B,C,``\,``);

\textgreater plotSegment(C,D,``\,``);

\textgreater plotSegment(A,D,``\,``);

\textgreater aspect(1):

\begin{figure}
\centering
\pandocbounded{\includegraphics[keepaspectratio]{images/Nazwa Yuan Adelia Putri_23030630095_Geometri-156.png}}
\caption{images/Nazwa\%20Yuan\%20Adelia\%20Putri\_23030630095\_Geometri-156.png}
\end{figure}

\textgreater l=angleBisector(A,B,C);

\textgreater m=angleBisector(B,C,D);

\textgreater P=lineIntersection(l,m);

\textgreater color(5); plotLine(l); plotLine(m); color(1);

\textgreater plotPoint(P,``P''):

\begin{figure}
\centering
\pandocbounded{\includegraphics[keepaspectratio]{images/Nazwa Yuan Adelia Putri_23030630095_Geometri-157.png}}
\caption{images/Nazwa\%20Yuan\%20Adelia\%20Putri\_23030630095\_Geometri-157.png}
\end{figure}

Dari gambar diatas terlihat bahwa keempat garis bagi sudutnya bertemu di satu titik yaitu titik P.

\textgreater r=norm(P-projectToLine(P,lineThrough(A,B)));

\textgreater plotCircle(circleWithCenter(P,r),``Lingkaran dalam segiempat ABCD''):

\begin{figure}
\centering
\pandocbounded{\includegraphics[keepaspectratio]{images/Nazwa Yuan Adelia Putri_23030630095_Geometri-158.png}}
\caption{images/Nazwa\%20Yuan\%20Adelia\%20Putri\_23030630095\_Geometri-158.png}
\end{figure}

Dari gambar diatas, terlihat bahwa sisi-sisinya merupakan garis singgung lingkaran yang sama yaitu lingkaran dalam segiempat.

Akan ditunjukkan bahwa hasil kali panjang sisi-sisi yang berhadapan sama.

\textgreater AB=norm(A-B) //panjang sisi AB

\begin{verbatim}
6
\end{verbatim}

\textgreater CD=norm(C-D) //panjang sisi CD

\begin{verbatim}
6
\end{verbatim}

\textgreater AD=norm(A-D) //panjang sisi AD

\begin{verbatim}
6
\end{verbatim}

\textgreater BC=norm(B-C) //panjang sisi BC

\begin{verbatim}
6
\end{verbatim}

\textgreater AB.CD

\begin{verbatim}
36
\end{verbatim}

\textgreater AD.BC

\begin{verbatim}
36
\end{verbatim}

\begin{enumerate}
\def\labelenumi{\arabic{enumi}.}
\setcounter{enumi}{3}
\tightlist
\item
  Gambarlah suatu ellips jika diketahui kedua titik fokusnya, misalnya P dan Q. Ingat ellips dengan fokus P dan Q adalah tempat kedudukan titik-titik yang jumlah jarak ke P dan ke Q selalu sama (konstan).
\end{enumerate}

Penyelesaian :

\textgreater P={[}-1,-1{]}; Q={[}1,-1{]};

\textgreater function d1(x,y):=sqrt((x-P{[}1{]})\textsuperscript{2+(y-P{[}2{]})}2)

\textgreater Q={[}1,-1{]}; function d2(x,y):=sqrt((x-P{[}1{]})\textsuperscript{2+(y-P{[}2{]})}2)+sqrt((x-Q{[}1{]})\textsuperscript{2+(y-Q{[}2{]})}2)

\textgreater fcontour(``d2'',xmin=-2,xmax=2,ymin=-3,ymax=1,hue=1):

\begin{figure}
\centering
\pandocbounded{\includegraphics[keepaspectratio]{images/Nazwa Yuan Adelia Putri_23030630095_Geometri-159.png}}
\caption{images/Nazwa\%20Yuan\%20Adelia\%20Putri\_23030630095\_Geometri-159.png}
\end{figure}

Grafik menjadi lebih menarik

\textgreater plot3d(``d2'',xmin=-2,xmax=2,ymin=-3,ymax=1):

\begin{figure}
\centering
\pandocbounded{\includegraphics[keepaspectratio]{images/Nazwa Yuan Adelia Putri_23030630095_Geometri-160.png}}
\caption{images/Nazwa\%20Yuan\%20Adelia\%20Putri\_23030630095\_Geometri-160.png}
\end{figure}

Batasan ke garis PQ

\textgreater plot2d(``abs(x+1)+abs(x-1)'',xmin=-3,xmax=3):

\begin{figure}
\centering
\pandocbounded{\includegraphics[keepaspectratio]{images/Nazwa Yuan Adelia Putri_23030630095_Geometri-161.png}}
\caption{images/Nazwa\%20Yuan\%20Adelia\%20Putri\_23030630095\_Geometri-161.png}
\end{figure}

\begin{enumerate}
\def\labelenumi{\arabic{enumi}.}
\setcounter{enumi}{4}
\tightlist
\item
  Gambarlah suatu hiperbola jika diketahui kedua titik fokusnya, misalnya P dan Q. Ingat ellips dengan fokus P dan Q adalah tempat kedudukan titik-titik yang selisih jarak ke P dan ke Q selalu sama (konstan).
\end{enumerate}

Penyelesaian :

\textgreater P={[}-1,-1{]}; Q={[}1,-1{]};

\textgreater function d1(x,y):=sqrt((x-p{[}1{]})\textsuperscript{2+(y-p{[}2{]})}2)

\textgreater Q={[}1,-1{]}; function d2(x,y):=sqrt((x-P{[}1{]})\textsuperscript{2+(y-P{[}2{]})}2)+sqrt((x+Q{[}1{]})\textsuperscript{2+(y+Q{[}2{]})}2)

\textgreater fcontour(``d2'',xmin=-2,xmax=2,ymin=-3,ymax=1,hue=1):

\begin{figure}
\centering
\pandocbounded{\includegraphics[keepaspectratio]{images/Nazwa Yuan Adelia Putri_23030630095_Geometri-162.png}}
\caption{images/Nazwa\%20Yuan\%20Adelia\%20Putri\_23030630095\_Geometri-162.png}
\end{figure}

Grafik yang lebih menarik

\textgreater plot3d(``d2'',xmin=-2,xmax=2,ymin=-3,ymax=1):

\begin{figure}
\centering
\pandocbounded{\includegraphics[keepaspectratio]{images/Nazwa Yuan Adelia Putri_23030630095_Geometri-163.png}}
\caption{images/Nazwa\%20Yuan\%20Adelia\%20Putri\_23030630095\_Geometri-163.png}
\end{figure}

\textgreater plot2d(``abs(x+1)+abs(x-1)'',xmin=-3,xmax=3):

\begin{figure}
\centering
\pandocbounded{\includegraphics[keepaspectratio]{images/Nazwa Yuan Adelia Putri_23030630095_Geometri-164.png}}
\caption{images/Nazwa\%20Yuan\%20Adelia\%20Putri\_23030630095\_Geometri-164.png}
\end{figure}

\backmatter
\end{document}
