% Options for packages loaded elsewhere
\PassOptionsToPackage{unicode}{hyperref}
\PassOptionsToPackage{hyphens}{url}
\documentclass[
]{book}
\usepackage{xcolor}
\usepackage{amsmath,amssymb}
\setcounter{secnumdepth}{-\maxdimen} % remove section numbering
\usepackage{iftex}
\ifPDFTeX
  \usepackage[T1]{fontenc}
  \usepackage[utf8]{inputenc}
  \usepackage{textcomp} % provide euro and other symbols
\else % if luatex or xetex
  \usepackage{unicode-math} % this also loads fontspec
  \defaultfontfeatures{Scale=MatchLowercase}
  \defaultfontfeatures[\rmfamily]{Ligatures=TeX,Scale=1}
\fi
\usepackage{lmodern}
\ifPDFTeX\else
  % xetex/luatex font selection
\fi
% Use upquote if available, for straight quotes in verbatim environments
\IfFileExists{upquote.sty}{\usepackage{upquote}}{}
\IfFileExists{microtype.sty}{% use microtype if available
  \usepackage[]{microtype}
  \UseMicrotypeSet[protrusion]{basicmath} % disable protrusion for tt fonts
}{}
\makeatletter
\@ifundefined{KOMAClassName}{% if non-KOMA class
  \IfFileExists{parskip.sty}{%
    \usepackage{parskip}
  }{% else
    \setlength{\parindent}{0pt}
    \setlength{\parskip}{6pt plus 2pt minus 1pt}}
}{% if KOMA class
  \KOMAoptions{parskip=half}}
\makeatother
\usepackage{graphicx}
\makeatletter
\newsavebox\pandoc@box
\newcommand*\pandocbounded[1]{% scales image to fit in text height/width
  \sbox\pandoc@box{#1}%
  \Gscale@div\@tempa{\textheight}{\dimexpr\ht\pandoc@box+\dp\pandoc@box\relax}%
  \Gscale@div\@tempb{\linewidth}{\wd\pandoc@box}%
  \ifdim\@tempb\p@<\@tempa\p@\let\@tempa\@tempb\fi% select the smaller of both
  \ifdim\@tempa\p@<\p@\scalebox{\@tempa}{\usebox\pandoc@box}%
  \else\usebox{\pandoc@box}%
  \fi%
}
% Set default figure placement to htbp
\def\fps@figure{htbp}
\makeatother
\setlength{\emergencystretch}{3em} % prevent overfull lines
\providecommand{\tightlist}{%
  \setlength{\itemsep}{0pt}\setlength{\parskip}{0pt}}
\usepackage{bookmark}
\IfFileExists{xurl.sty}{\usepackage{xurl}}{} % add URL line breaks if available
\urlstyle{same}
\hypersetup{
  hidelinks,
  pdfcreator={LaTeX via pandoc}}

\author{}
\date{}

\begin{document}
\frontmatter

\mainmatter
\chapter{Nazwa Yuan Adelia Putri\_23030630095\_EMT4Plot2D}\label{nazwa-yuan-adelia-putri_23030630095_emt4plot2d}

Nama : Nazwa Yuan Adelia Putri Kelas : Matematika B NIM : 23030630095

\chapter{Menggambar Grafik 2D dengan EMT}\label{menggambar-grafik-2d-dengan-emt}

Notebook ini menjelaskan tentang cara menggambar berbagaikurva dan grafik 2D dengan software EMT. EMT menyediakan fungsi plot2d() untuk menggambar berbagai kurva dan grafik dua dimensi (2D).

\section{Plot~Dasar}\label{plot-dasar}

Ada fungsi yang sangat mendasar dari plot. Ada koordinat layar, yang selalu berkisar dari 0 hingga 1024 di setiap sumbu, tidak peduli apakah layarnya persegi atau tidak. Semut ada koordinat plot, yang dapat diatur dengan setplot(). Pemetaan antara koordinat tergantung pada jendela plot saat ini. Misalnya, shrinkwindow() default menyisakan ruang untuk label sumbu dan judul plot.

Dalam contoh, kita hanya menggambar beberapa garis acak dalam berbagai warna. Untuk detail tentang fungsi ini, pelajari fungsi inti EMT.

\textgreater clg; // clear screen

\textgreater window(0,0,1024,1024); // use all of the window

\textgreater setplot(0,1,0,1); // set plot coordinates

\textgreater hold on; // start overwrite mode

\textgreater n=100; X=random(n,2); Y=random(n,2); // get random points

\textgreater colors=rgb(random(n),random(n),random(n)); // get random colors

\textgreater loop 1 to n; color(colors{[}\#{]}); plot(X{[}\#{]},Y{[}\#{]}); end; // plot

\textgreater hold off; // end overwrite mode

\textgreater insimg; // insert to notebook

\begin{figure}
\centering
\pandocbounded{\includegraphics[keepaspectratio]{images/Nazwa Yuan Adelia Putri_23030630095_EMT4Plot2D-001.png}}
\caption{images/Nazwa\%20Yuan\%20Adelia\%20Putri\_23030630095\_EMT4Plot2D-001.png}
\end{figure}

\textgreater reset;

Grafik perlu ditahan, karena perintah plot() akan menghapus jendela plot.

Untuk menghapus semua yang kami lakukan, kami menggunakan reset().

Untuk menampilkan gambar hasil plot di layar notebook, perintah plot2d() dapat diakhiri dengan titik dua (:). Cara lain adalah perintah plot2d() diakhiri dengan titik koma (;), kemudian menggunakan perintah insimg() untuk menampilkan gambar hasil plot.

Untuk contoh lain, kami menggambar plot sebagai sisipan di plot lain. Ini dilakukan dengan mendefinisikan jendela plot yang lebih kecil. Perhatikan bahwa jendela ini tidak menyediakan ruang untuk label sumbu di luar jendela plot. Kita harus menambahkan beberapa margin untuk ini sesuai kebutuhan. Perhatikan bahwa kami menyimpan dan memulihkan jendela penuh, dan menahan plot saat ini saat kami memplot inset.

\textgreater plot2d(``x\^{}3-x'');

\textgreater xw=200; yw=100; ww=300; hw=300;

\textgreater ow=window();

\textgreater window(xw,yw,xw+ww,yw+hw);

\textgreater hold on;

\textgreater barclear(xw-50,yw-10,ww+60,ww+60);

\textgreater plot2d(``x\^{}4-x'',grid=6):

\begin{figure}
\centering
\pandocbounded{\includegraphics[keepaspectratio]{images/Nazwa Yuan Adelia Putri_23030630095_EMT4Plot2D-002.png}}
\caption{images/Nazwa\%20Yuan\%20Adelia\%20Putri\_23030630095\_EMT4Plot2D-002.png}
\end{figure}

\textgreater hold off;

\textgreater window(ow);

Plot dengan banyak angka dicapai dengan cara yang sama. Ada fungsi figure() utilitas untuk ini.

\section{Aspek~Plot}\label{aspek-plot}

Plot default menggunakan jendela plot persegi. Anda dapat mengubah ini dengan fungsi aspek(). Jangan lupa untuk mengatur ulang aspek nanti. Anda juga dapat mengubah default ini di menu dengan ``Set Aspect'' ke rasio aspek tertentu atau ke ukuran jendela grafis saat ini.

Tetapi Anda juga dapat mengubahnya untuk satu plot. Untuk ini, ukuran area plot saat ini diubah, dan jendela diatur sehingga label memiliki cukup ruang.

\textgreater aspect(2); // rasio panjang dan lebar 2:1

\textgreater plot2d({[}``sin(x)'',``cos(x)''{]},0,2pi):

\begin{figure}
\centering
\pandocbounded{\includegraphics[keepaspectratio]{images/Nazwa Yuan Adelia Putri_23030630095_EMT4Plot2D-003.png}}
\caption{images/Nazwa\%20Yuan\%20Adelia\%20Putri\_23030630095\_EMT4Plot2D-003.png}
\end{figure}

\textgreater aspect();

\textgreater reset;

Fungsi reset() mengembalikan default plot termasuk rasio aspek.

\chapter{Plot 2D di Euler}\label{plot-2d-di-euler}

EMT Math Toolbox memiliki plot dalam 2D, baik untuk data maupun fungsi. EMT menggunakan fungsi plot2d. Fungsi ini dapat memplot fungsi dan data.

Dimungkinkan untuk membuat plot di Maxima menggunakan Gnuplot atau dengan Python menggunakan Math Plot Lib.

Euler dapat memplot plot 2D dari

\begin{itemize}
\item
  ekspresi
\item
  fungsi, variabel, atau kurva parameter,
\item
  vektor nilai x-y,
\item
  awan titik di pesawat,
\item
  kurva implisit dengan level atau wilayah level.
\item
  Fungsi kompleks
\end{itemize}

Gaya plot mencakup berbagai gaya untuk garis dan titik, plot batang dan plot berbayang.

\chapter{Plot Ekspresi atau Variabel}\label{plot-ekspresi-atau-variabel}

Ekspresi tunggal dalam ``x'' (mis. ``4*x\^{}2'') atau nama fungsi (mis. ``f'') menghasilkan grafik fungsi.

Berikut adalah contoh paling dasar, yang menggunakan rentang default dan menetapkan rentang y yang tepat agar sesuai dengan plot fungsi.

Catatan: Jika Anda mengakhiri baris perintah dengan titik dua ``:'', plot akan dimasukkan ke dalam jendela teks. Jika tidak, tekan TAB untuk melihat plot jika jendela plot tertutup.

\textgreater plot2d(``x\^{}2''):

\begin{figure}
\centering
\pandocbounded{\includegraphics[keepaspectratio]{images/Nazwa Yuan Adelia Putri_23030630095_EMT4Plot2D-004.png}}
\caption{images/Nazwa\%20Yuan\%20Adelia\%20Putri\_23030630095\_EMT4Plot2D-004.png}
\end{figure}

\textgreater aspect(1.5); plot2d(``x\^{}3-x''):

\begin{figure}
\centering
\pandocbounded{\includegraphics[keepaspectratio]{images/Nazwa Yuan Adelia Putri_23030630095_EMT4Plot2D-005.png}}
\caption{images/Nazwa\%20Yuan\%20Adelia\%20Putri\_23030630095\_EMT4Plot2D-005.png}
\end{figure}

\textgreater a:=5.6; plot2d(``exp(-a*x\^{}2)/a''); insimg(30); // menampilkan gambar hasil plot setinggi 25 baris

\begin{figure}
\centering
\pandocbounded{\includegraphics[keepaspectratio]{images/Nazwa Yuan Adelia Putri_23030630095_EMT4Plot2D-006.png}}
\caption{images/Nazwa\%20Yuan\%20Adelia\%20Putri\_23030630095\_EMT4Plot2D-006.png}
\end{figure}

Dari beberapa contoh sebelumnya Anda dapat melihat bahwa aslinya gambar plot menggunakan sumbu X dengan rentang nilai dari -2 sampai dengan 2. Untuk mengubah rentang nilai X dan Y, Anda dapat menambahkan nilai-nilai batas X (dan Y) di belakang ekspresi yang digambar.

The plot range is set with the following assigned parameters

\begin{itemize}
\item
  a,b: x-range (default -2,2)
\item
  c,d: y-range (default: scale with values)
\item
  r: alternatively a radius around the plot center
\item
  cx,cy: the coordinates of the plot center (default 0,0)
\end{itemize}

\textgreater plot2d(``x\^{}3-x'',-1,2):

\begin{figure}
\centering
\pandocbounded{\includegraphics[keepaspectratio]{images/Nazwa Yuan Adelia Putri_23030630095_EMT4Plot2D-007.png}}
\caption{images/Nazwa\%20Yuan\%20Adelia\%20Putri\_23030630095\_EMT4Plot2D-007.png}
\end{figure}

\textgreater plot2d(``sin(x)'',-2*pi,2*pi): // plot sin(x) pada interval {[}-2pi, 2pi{]}

\begin{figure}
\centering
\pandocbounded{\includegraphics[keepaspectratio]{images/Nazwa Yuan Adelia Putri_23030630095_EMT4Plot2D-008.png}}
\caption{images/Nazwa\%20Yuan\%20Adelia\%20Putri\_23030630095\_EMT4Plot2D-008.png}
\end{figure}

\textgreater plot2d(``cos(x)'',``sin(3*x)'',xmin=0,xmax=2pi):

\begin{figure}
\centering
\pandocbounded{\includegraphics[keepaspectratio]{images/Nazwa Yuan Adelia Putri_23030630095_EMT4Plot2D-009.png}}
\caption{images/Nazwa\%20Yuan\%20Adelia\%20Putri\_23030630095\_EMT4Plot2D-009.png}
\end{figure}

Alternatif untuk titik dua adalah perintah insimg(baris), yang menyisipkan plot yang menempati sejumlah baris teks tertentu.

Dalam opsi, plot dapat diatur untuk muncul

\begin{itemize}
\item
  di jendela terpisah yang dapat diubah ukurannya,
\item
  di jendela buku catatan.
\end{itemize}

Lebih banyak gaya dapat dicapai dengan perintah plot tertentu.

Bagaimanapun, tekan tombol tabulator untuk melihat plot, jika disembunyikan.

Untuk membagi jendela menjadi beberapa plot, gunakan perintah figure(). Dalam contoh, kami memplot x\^{}1 hingga x\^{}4 menjadi 4 bagian jendela. figure(0) mengatur ulang jendela default.

\textgreater reset;

\textgreater figure(2,2); \ldots{}\\
\textgreater{} for n=1 to 4; figure(n); plot2d(``x\^{}''+n); end; \ldots{}\\
\textgreater{} figure(0):

\begin{figure}
\centering
\pandocbounded{\includegraphics[keepaspectratio]{images/Nazwa Yuan Adelia Putri_23030630095_EMT4Plot2D-010.png}}
\caption{images/Nazwa\%20Yuan\%20Adelia\%20Putri\_23030630095\_EMT4Plot2D-010.png}
\end{figure}

Di plot2d(), ada gaya alternatif yang tersedia dengan grid=x. Untuk gambaran umum, kami menunjukkan berbagai gaya kisi dalam satu gambar (lihat di bawah untuk perintah figure()). Gaya kisi=0 tidak disertakan. Ini menunjukkan tidak ada grid dan tidak ada bingkai.

\textgreater figure(3,3); \ldots{}\\
\textgreater{} for k=1:9; figure(k); plot2d(``x\^{}3-x'',-2,1,grid=k); end; \ldots{}\\
\textgreater{} figure(0):

\begin{figure}
\centering
\pandocbounded{\includegraphics[keepaspectratio]{images/Nazwa Yuan Adelia Putri_23030630095_EMT4Plot2D-011.png}}
\caption{images/Nazwa\%20Yuan\%20Adelia\%20Putri\_23030630095\_EMT4Plot2D-011.png}
\end{figure}

Jika argumen ke plot2d() adalah ekspresi yang diikuti oleh empat angka, angka-angka ini adalah rentang x dan y untuk plot.

Atau, a, b, c, d dapat ditentukan sebagai parameter yang ditetapkan sebagai a=\ldots{} dll.

Dalam contoh berikut, kita mengubah gaya kisi, menambahkan label, dan menggunakan label vertikal untuk sumbu y.

\textgreater aspect(1.5); plot2d(``sin(x)'',0,2pi,-1.2,1.2,grid=3,xl=``x'',yl=``sin(x)''):

\begin{figure}
\centering
\pandocbounded{\includegraphics[keepaspectratio]{images/Nazwa Yuan Adelia Putri_23030630095_EMT4Plot2D-012.png}}
\caption{images/Nazwa\%20Yuan\%20Adelia\%20Putri\_23030630095\_EMT4Plot2D-012.png}
\end{figure}

\textgreater plot2d(``sin(x)+cos(2*x)'',0,4pi):

\begin{figure}
\centering
\pandocbounded{\includegraphics[keepaspectratio]{images/Nazwa Yuan Adelia Putri_23030630095_EMT4Plot2D-013.png}}
\caption{images/Nazwa\%20Yuan\%20Adelia\%20Putri\_23030630095\_EMT4Plot2D-013.png}
\end{figure}

Gambar yang dihasilkan dengan memasukkan plot ke dalam jendela teks disimpan di direktori yang sama dengan buku catatan, secara default di subdirektori bernama ``gambar''. Mereka juga digunakan oleh ekspor HTML.

Anda cukup menandai gambar apa saja dan menyalinnya ke clipboard dengan Ctrl-C. Tentu saja, Anda juga dapat mengekspor grafik saat ini dengan fungsi di menu File.

Fungsi atau ekspresi dalam plot2d dievaluasi secara adaptif. Untuk kecepatan lebih, matikan plot adaptif dengan \textless adaptive dan tentukan jumlah subinterval dengan n=\ldots{} Ini hanya diperlukan dalam kasus yang jarang terjadi.

\textgreater plot2d(``sign(x)*exp(-x\^{}2)'',-1,1,\textless adaptive,n=10000):

\begin{figure}
\centering
\pandocbounded{\includegraphics[keepaspectratio]{images/Nazwa Yuan Adelia Putri_23030630095_EMT4Plot2D-014.png}}
\caption{images/Nazwa\%20Yuan\%20Adelia\%20Putri\_23030630095\_EMT4Plot2D-014.png}
\end{figure}

\textgreater plot2d(``x\^{}x'',r=1.2,cx=1,cy=1):

\begin{figure}
\centering
\pandocbounded{\includegraphics[keepaspectratio]{images/Nazwa Yuan Adelia Putri_23030630095_EMT4Plot2D-015.png}}
\caption{images/Nazwa\%20Yuan\%20Adelia\%20Putri\_23030630095\_EMT4Plot2D-015.png}
\end{figure}

Perhatikan bahwa x\^{}x tidak didefinisikan untuk x\textless=0. Fungsi plot2d menangkap kesalahan ini, dan mulai merencanakan segera setelah fungsi didefinisikan. Ini berfungsi untuk semua fungsi yang mengembalikan NAN keluar dari jangkauan definisinya.

\textgreater plot2d(``log(x)'',-0.1,2):

\begin{figure}
\centering
\pandocbounded{\includegraphics[keepaspectratio]{images/Nazwa Yuan Adelia Putri_23030630095_EMT4Plot2D-016.png}}
\caption{images/Nazwa\%20Yuan\%20Adelia\%20Putri\_23030630095\_EMT4Plot2D-016.png}
\end{figure}

Parameter square=true (atau \textgreater square) memilih y-range secara otomatis sehingga hasilnya adalah jendela plot persegi. Perhatikan bahwa secara default, Euler menggunakan ruang persegi di dalam jendela plot.

\textgreater plot2d(``x\^{}3-x'',\textgreater square):

\begin{figure}
\centering
\pandocbounded{\includegraphics[keepaspectratio]{images/Nazwa Yuan Adelia Putri_23030630095_EMT4Plot2D-017.png}}
\caption{images/Nazwa\%20Yuan\%20Adelia\%20Putri\_23030630095\_EMT4Plot2D-017.png}
\end{figure}

\textgreater plot2d(`'integrate(``sin(x)*exp(-x\^{}2)'',0,x)'\,',0,2): // plot integral

\begin{figure}
\centering
\pandocbounded{\includegraphics[keepaspectratio]{images/Nazwa Yuan Adelia Putri_23030630095_EMT4Plot2D-018.png}}
\caption{images/Nazwa\%20Yuan\%20Adelia\%20Putri\_23030630095\_EMT4Plot2D-018.png}
\end{figure}

Jika Anda membutuhkan lebih banyak ruang untuk label-y, panggil shrinkwindow() dengan parameter yang lebih kecil, atau tetapkan nilai positif untuk ``lebih kecil'' di plot2d().

\textgreater plot2d(``gamma(x)'',1,10,yl=``y-values'',smaller=6,\textless vertical):

\begin{figure}
\centering
\pandocbounded{\includegraphics[keepaspectratio]{images/Nazwa Yuan Adelia Putri_23030630095_EMT4Plot2D-019.png}}
\caption{images/Nazwa\%20Yuan\%20Adelia\%20Putri\_23030630095\_EMT4Plot2D-019.png}
\end{figure}

Ekspresi simbolik juga dapat digunakan, karena disimpan sebagai ekspresi string sederhana.

\textgreater x=linspace(0,2pi,1000); plot2d(sin(5x),cos(7x)):

\begin{figure}
\centering
\pandocbounded{\includegraphics[keepaspectratio]{images/Nazwa Yuan Adelia Putri_23030630095_EMT4Plot2D-020.png}}
\caption{images/Nazwa\%20Yuan\%20Adelia\%20Putri\_23030630095\_EMT4Plot2D-020.png}
\end{figure}

\textgreater a:=5.6; expr \&= exp(-a*x\^{}2)/a; // define expression

\textgreater plot2d(expr,-2,2): // plot from -2 to 2

\begin{figure}
\centering
\pandocbounded{\includegraphics[keepaspectratio]{images/Nazwa Yuan Adelia Putri_23030630095_EMT4Plot2D-021.png}}
\caption{images/Nazwa\%20Yuan\%20Adelia\%20Putri\_23030630095\_EMT4Plot2D-021.png}
\end{figure}

\textgreater plot2d(expr,r=1,thickness=2): // plot in a square around (0,0)

\begin{figure}
\centering
\pandocbounded{\includegraphics[keepaspectratio]{images/Nazwa Yuan Adelia Putri_23030630095_EMT4Plot2D-022.png}}
\caption{images/Nazwa\%20Yuan\%20Adelia\%20Putri\_23030630095\_EMT4Plot2D-022.png}
\end{figure}

\textgreater plot2d(\&diff(expr,x),\textgreater add,style=``--'',color=red): // add another plot

\begin{figure}
\centering
\pandocbounded{\includegraphics[keepaspectratio]{images/Nazwa Yuan Adelia Putri_23030630095_EMT4Plot2D-023.png}}
\caption{images/Nazwa\%20Yuan\%20Adelia\%20Putri\_23030630095\_EMT4Plot2D-023.png}
\end{figure}

\textgreater plot2d(\&diff(expr,x,2),a=-2,b=2,c=-2,d=1): // plot in rectangle

\begin{figure}
\centering
\pandocbounded{\includegraphics[keepaspectratio]{images/Nazwa Yuan Adelia Putri_23030630095_EMT4Plot2D-024.png}}
\caption{images/Nazwa\%20Yuan\%20Adelia\%20Putri\_23030630095\_EMT4Plot2D-024.png}
\end{figure}

\textgreater plot2d(\&diff(expr,x),a=-2,b=2,\textgreater square): // keep plot square

\begin{figure}
\centering
\pandocbounded{\includegraphics[keepaspectratio]{images/Nazwa Yuan Adelia Putri_23030630095_EMT4Plot2D-025.png}}
\caption{images/Nazwa\%20Yuan\%20Adelia\%20Putri\_23030630095\_EMT4Plot2D-025.png}
\end{figure}

\textgreater plot2d(``x\^{}2'',0,1,steps=1,color=red,n=10):

\begin{figure}
\centering
\pandocbounded{\includegraphics[keepaspectratio]{images/Nazwa Yuan Adelia Putri_23030630095_EMT4Plot2D-026.png}}
\caption{images/Nazwa\%20Yuan\%20Adelia\%20Putri\_23030630095\_EMT4Plot2D-026.png}
\end{figure}

\textgreater plot2d(``x\^{}2'',\textgreater add,steps=2,color=blue,n=10):

\begin{figure}
\centering
\pandocbounded{\includegraphics[keepaspectratio]{images/Nazwa Yuan Adelia Putri_23030630095_EMT4Plot2D-027.png}}
\caption{images/Nazwa\%20Yuan\%20Adelia\%20Putri\_23030630095\_EMT4Plot2D-027.png}
\end{figure}

\chapter{Fungsi dalam satu Parameter}\label{fungsi-dalam-satu-parameter}

Fungsi plot yang paling penting untuk plot planar adalah plot2d(). Fungsi ini diimplementasikan dalam bahasa Euler dalam file ``plot.e'', yang dimuat di awal program.

Berikut adalah beberapa contoh menggunakan fungsi. Seperti biasa di EMT, fungsi yang berfungsi untuk fungsi atau ekspresi lain, Anda dapat meneruskan parameter tambahan (selain x) yang bukan variabel global ke fungsi dengan parameter titik koma atau dengan koleksi panggilan.

\textgreater function f(x,a) := x\textsuperscript{2/a+a*x}2-x; // define a function

\textgreater a=0.3; plot2d(``f'',0,1;a): // plot with a=0.3

\begin{figure}
\centering
\pandocbounded{\includegraphics[keepaspectratio]{images/Nazwa Yuan Adelia Putri_23030630095_EMT4Plot2D-028.png}}
\caption{images/Nazwa\%20Yuan\%20Adelia\%20Putri\_23030630095\_EMT4Plot2D-028.png}
\end{figure}

\textgreater plot2d(``f'',0,1;0.4): // plot with a=0.4

\begin{figure}
\centering
\pandocbounded{\includegraphics[keepaspectratio]{images/Nazwa Yuan Adelia Putri_23030630095_EMT4Plot2D-029.png}}
\caption{images/Nazwa\%20Yuan\%20Adelia\%20Putri\_23030630095\_EMT4Plot2D-029.png}
\end{figure}

\textgreater plot2d(\{\{``f'',0.2\}\},0,1): // plot with a=0.2

\begin{figure}
\centering
\pandocbounded{\includegraphics[keepaspectratio]{images/Nazwa Yuan Adelia Putri_23030630095_EMT4Plot2D-030.png}}
\caption{images/Nazwa\%20Yuan\%20Adelia\%20Putri\_23030630095\_EMT4Plot2D-030.png}
\end{figure}

\textgreater plot2d(\{\{``f(x,b)'',b=0.1\}\},0,1): // plot with 0.1

\begin{figure}
\centering
\pandocbounded{\includegraphics[keepaspectratio]{images/Nazwa Yuan Adelia Putri_23030630095_EMT4Plot2D-031.png}}
\caption{images/Nazwa\%20Yuan\%20Adelia\%20Putri\_23030630095\_EMT4Plot2D-031.png}
\end{figure}

\textgreater function f(x) := x\^{}3-x; \ldots{}\\
\textgreater{} plot2d(``f'',r=1):

\begin{figure}
\centering
\pandocbounded{\includegraphics[keepaspectratio]{images/Nazwa Yuan Adelia Putri_23030630095_EMT4Plot2D-032.png}}
\caption{images/Nazwa\%20Yuan\%20Adelia\%20Putri\_23030630095\_EMT4Plot2D-032.png}
\end{figure}

Berikut adalah ringkasan dari fungsi yang diterima

\begin{itemize}
\item
  ekspresi atau ekspresi simbolik dalam x
\item
  fungsi atau fungsi simbolis dengan nama sebagai ``f''
\item
  fungsi simbolis hanya dengan nama f
\end{itemize}

Fungsi plot2d() juga menerima fungsi simbolis. Untuk fungsi simbolis, nama saja yang berfungsi.

\textgreater function f(x) \&= diff(x\^{}x,x)

\begin{verbatim}
                            x
                           x  (log(x) + 1)
\end{verbatim}

\textgreater plot2d(f,0,2):

\begin{figure}
\centering
\pandocbounded{\includegraphics[keepaspectratio]{images/Nazwa Yuan Adelia Putri_23030630095_EMT4Plot2D-033.png}}
\caption{images/Nazwa\%20Yuan\%20Adelia\%20Putri\_23030630095\_EMT4Plot2D-033.png}
\end{figure}

Tentu saja, untuk ekspresi atau ekspresi simbolik, nama variabel sudah cukup untuk memplotnya.

\textgreater expr \&= sin(x)*exp(-x)

\begin{verbatim}
                              - x
                             E    sin(x)
\end{verbatim}

\textgreater plot2d(expr,0,3pi):

\begin{figure}
\centering
\pandocbounded{\includegraphics[keepaspectratio]{images/Nazwa Yuan Adelia Putri_23030630095_EMT4Plot2D-034.png}}
\caption{images/Nazwa\%20Yuan\%20Adelia\%20Putri\_23030630095\_EMT4Plot2D-034.png}
\end{figure}

\textgreater function f(x) \&= x\^{}x;

\textgreater plot2d(f,r=1,cx=1,cy=1,color=blue,thickness=2);

\textgreater plot2d(\&diff(f(x),x),\textgreater add,color=red,style=``-.-''):

\begin{figure}
\centering
\pandocbounded{\includegraphics[keepaspectratio]{images/Nazwa Yuan Adelia Putri_23030630095_EMT4Plot2D-035.png}}
\caption{images/Nazwa\%20Yuan\%20Adelia\%20Putri\_23030630095\_EMT4Plot2D-035.png}
\end{figure}

Untuk gaya garis ada berbagai pilihan.

\begin{itemize}
\item
  gaya=``\ldots{}''. Pilih dari ``-'', ``--'', ``-.'', ``.'', ``.-.'', ``-.-''.
\item
  warna: Lihat di bawah untuk warna.
\item
  ketebalan: Default adalah 1.
\end{itemize}

Warna dapat dipilih sebagai salah satu warna default, atau sebagai warna RGB.

\begin{itemize}
\item
  0.15: indeks warna default.
\item
  konstanta warna: putih, hitam, merah, hijau, biru, cyan, zaitun,
\item
  abu-abu muda, abu-abu, abu-abu tua, oranye, hijau muda, pirus, biru
\item
  muda, oranye terang, kuning
\item
  rgb(merah, hijau, biru): parameter adalah real dalam {[}0,1{]}.
\end{itemize}

\textgreater plot2d(``exp(-x\^{}2)'',r=2,color=red,thickness=3,style=``--''):

\begin{figure}
\centering
\pandocbounded{\includegraphics[keepaspectratio]{images/Nazwa Yuan Adelia Putri_23030630095_EMT4Plot2D-036.png}}
\caption{images/Nazwa\%20Yuan\%20Adelia\%20Putri\_23030630095\_EMT4Plot2D-036.png}
\end{figure}

Berikut adalah tampilan warna EMT yang telah ditentukan sebelumnya.

\textgreater aspect(2); columnsplot(ones(1,16),lab=0:15,grid=0,color=0:15):

\begin{figure}
\centering
\pandocbounded{\includegraphics[keepaspectratio]{images/Nazwa Yuan Adelia Putri_23030630095_EMT4Plot2D-037.png}}
\caption{images/Nazwa\%20Yuan\%20Adelia\%20Putri\_23030630095\_EMT4Plot2D-037.png}
\end{figure}

Tapi Anda bisa menggunakan warna apa saja.

\textgreater columnsplot(ones(1,16),grid=0,color=rgb(0,0,linspace(0,1,15))):

\begin{figure}
\centering
\pandocbounded{\includegraphics[keepaspectratio]{images/Nazwa Yuan Adelia Putri_23030630095_EMT4Plot2D-038.png}}
\caption{images/Nazwa\%20Yuan\%20Adelia\%20Putri\_23030630095\_EMT4Plot2D-038.png}
\end{figure}

\chapter{Menggambar Beberapa Kurva pada bidang koordinat yang sama}\label{menggambar-beberapa-kurva-pada-bidang-koordinat-yang-sama}

Plot lebih dari satu fungsi (multiple function) ke dalam satu jendela dapat dilakukan dengan berbagai cara. Salah satu metode menggunakan \textgreater add untuk beberapa panggilan ke plot2d secara keseluruhan, tetapi panggilan pertama. Kami telah menggunakan fitur ini dalam contoh di atas.

\textgreater aspect(); plot2d(``cos(x)'',r=2,grid=6); plot2d(``x'',style=``.'',\textgreater add):

\begin{figure}
\centering
\pandocbounded{\includegraphics[keepaspectratio]{images/Nazwa Yuan Adelia Putri_23030630095_EMT4Plot2D-039.png}}
\caption{images/Nazwa\%20Yuan\%20Adelia\%20Putri\_23030630095\_EMT4Plot2D-039.png}
\end{figure}

\textgreater aspect(1.5); plot2d(``sin(x)'',0,2pi); plot2d(``cos(x)'',color=blue,style=``--'',\textgreater add):

\begin{figure}
\centering
\pandocbounded{\includegraphics[keepaspectratio]{images/Nazwa Yuan Adelia Putri_23030630095_EMT4Plot2D-040.png}}
\caption{images/Nazwa\%20Yuan\%20Adelia\%20Putri\_23030630095\_EMT4Plot2D-040.png}
\end{figure}

Salah satu kegunaan \textgreater add adalah untuk menambahkan titik pada kurva.

\textgreater plot2d(``sin(x)'',0,pi); plot2d(2,sin(2),\textgreater points,\textgreater add):

\begin{figure}
\centering
\pandocbounded{\includegraphics[keepaspectratio]{images/Nazwa Yuan Adelia Putri_23030630095_EMT4Plot2D-041.png}}
\caption{images/Nazwa\%20Yuan\%20Adelia\%20Putri\_23030630095\_EMT4Plot2D-041.png}
\end{figure}

Kami menambahkan titik persimpangan dengan label (pada posisi ``cl'' untuk kiri tengah), dan memasukkan hasilnya ke dalam notebook. Kami juga menambahkan judul ke plot.

\textgreater plot2d({[}``cos(x)'',``x''{]},r=1.1,cx=0.5,cy=0.5, \ldots{}\\
\textgreater{} color={[}black,blue{]},style={[}``-'',``.''{]}, \ldots{}\\
\textgreater{} grid=1);

\textgreater x0=solve(``cos(x)-x'',1); \ldots{}\\
\textgreater{} plot2d(x0,x0,\textgreater points,\textgreater add,title=``Intersection Demo''); \ldots{}\\
\textgreater{} label(``cos(x) = x'',x0,x0,pos=``cl'',offset=20):

\begin{figure}
\centering
\pandocbounded{\includegraphics[keepaspectratio]{images/Nazwa Yuan Adelia Putri_23030630095_EMT4Plot2D-042.png}}
\caption{images/Nazwa\%20Yuan\%20Adelia\%20Putri\_23030630095\_EMT4Plot2D-042.png}
\end{figure}

Dalam demo berikut, kami memplot fungsi sinc(x)=sin(x)/x dan ekspansi Taylor ke-8 dan ke-16. Kami menghitung ekspansi ini menggunakan Maxima melalui ekspresi simbolis.

Plot ini dilakukan dalam perintah multi-baris berikut dengan tiga panggilan ke plot2d(). Yang kedua dan yang ketiga memiliki set flag \textgreater add, yang membuat plot menggunakan rentang sebelumnya.

Kami menambahkan kotak label yang menjelaskan fungsi.

\textgreater\$taylor(sin(x)/x,x,0,4)

\[\frac{x^4}{120}-\frac{x^2}{6}+1\]\textgreater plot2d(``sinc(x)'',0,4pi,color=green,thickness=2); \ldots{}\\
\textgreater{} plot2d(\&taylor(sin(x)/x,x,0,8),\textgreater add,color=blue,style=``--''); \ldots{}\\
\textgreater{} plot2d(\&taylor(sin(x)/x,x,0,16),\textgreater add,color=red,style=``-.-''); \ldots{}\\
\textgreater{} labelbox({[}``sinc'',``T8'',``T16''{]},styles={[}``-'',``--'',``-.-''{]}, \ldots{}\\
\textgreater{} colors={[}black,blue,red{]}):

\begin{figure}
\centering
\pandocbounded{\includegraphics[keepaspectratio]{images/Nazwa Yuan Adelia Putri_23030630095_EMT4Plot2D-044.png}}
\caption{images/Nazwa\%20Yuan\%20Adelia\%20Putri\_23030630095\_EMT4Plot2D-044.png}
\end{figure}

Dalam contoh berikut, kami menghasilkan Bernstein-Polinomial.

\[B_i(x) = \binom{n}{i} x^i (1-x)^{n-i}\]\textgreater plot2d(``(1-x)\^{}10'',0,1); // plot first function

\textgreater for i=1 to 10; plot2d(``bin(10,i)*x\textsuperscript{i*(1-x)}(10-i)'',\textgreater add); end;

\textgreater insimg;

\begin{figure}
\centering
\pandocbounded{\includegraphics[keepaspectratio]{images/Nazwa Yuan Adelia Putri_23030630095_EMT4Plot2D-046.png}}
\caption{images/Nazwa\%20Yuan\%20Adelia\%20Putri\_23030630095\_EMT4Plot2D-046.png}
\end{figure}

Metode kedua menggunakan pasangan matriks nilai-x dan matriks nilai-y yang berukuran sama.

Kami menghasilkan matriks nilai dengan satu Polinomial Bernstein di setiap baris. Untuk ini, kita cukup menggunakan vektor kolom i. Lihat pengantar tentang bahasa matriks untuk mempelajari lebih detail.

\textgreater x=linspace(0,1,500);

\textgreater n=10; k=(0:n)'; // n is row vector, k is column vector

\textgreater y=bin(n,k)*x\textsuperscript{k*(1-x)}(n-k); // y is a matrix then

\textgreater plot2d(x,y):

\begin{figure}
\centering
\pandocbounded{\includegraphics[keepaspectratio]{images/Nazwa Yuan Adelia Putri_23030630095_EMT4Plot2D-047.png}}
\caption{images/Nazwa\%20Yuan\%20Adelia\%20Putri\_23030630095\_EMT4Plot2D-047.png}
\end{figure}

Perhatikan bahwa parameter warna dapat berupa vektor. Kemudian setiap warna digunakan untuk setiap baris matriks.

\textgreater x=linspace(0,1,200); y=x\^{}(1:10)'; plot2d(x,y,color=1:10):

\begin{figure}
\centering
\pandocbounded{\includegraphics[keepaspectratio]{images/Nazwa Yuan Adelia Putri_23030630095_EMT4Plot2D-048.png}}
\caption{images/Nazwa\%20Yuan\%20Adelia\%20Putri\_23030630095\_EMT4Plot2D-048.png}
\end{figure}

Metode lain adalah menggunakan vektor ekspresi (string). Anda kemudian dapat menggunakan larik warna, larik gaya, dan larik ketebalan dengan panjang yang sama.

\textgreater plot2d({[}``sin(x)'',``cos(x)''{]},0,2pi,color=4:5):

\begin{figure}
\centering
\pandocbounded{\includegraphics[keepaspectratio]{images/Nazwa Yuan Adelia Putri_23030630095_EMT4Plot2D-049.png}}
\caption{images/Nazwa\%20Yuan\%20Adelia\%20Putri\_23030630095\_EMT4Plot2D-049.png}
\end{figure}

\textgreater plot2d({[}``sin(x)'',``cos(x)''{]},0,2pi): // plot vector of expressions

\begin{figure}
\centering
\pandocbounded{\includegraphics[keepaspectratio]{images/Nazwa Yuan Adelia Putri_23030630095_EMT4Plot2D-050.png}}
\caption{images/Nazwa\%20Yuan\%20Adelia\%20Putri\_23030630095\_EMT4Plot2D-050.png}
\end{figure}

Kita bisa mendapatkan vektor seperti itu dari Maxima menggunakan makelist() dan mxm2str().

\textgreater v \&= makelist(binomial(10,i)*x\textsuperscript{i*(1-x)}(10-i),i,0,10) // make list

\begin{verbatim}
               10            9              8  2             7  3
       [(1 - x)  , 10 (1 - x)  x, 45 (1 - x)  x , 120 (1 - x)  x , 
           6  4             5  5             4  6             3  7
210 (1 - x)  x , 252 (1 - x)  x , 210 (1 - x)  x , 120 (1 - x)  x , 
          2  8              9   10
45 (1 - x)  x , 10 (1 - x) x , x  ]
\end{verbatim}

\textgreater mxm2str(v) // get a vector of strings from the symbolic vector

\begin{verbatim}
(1-x)^10
10*(1-x)^9*x
45*(1-x)^8*x^2
120*(1-x)^7*x^3
210*(1-x)^6*x^4
252*(1-x)^5*x^5
210*(1-x)^4*x^6
120*(1-x)^3*x^7
45*(1-x)^2*x^8
10*(1-x)*x^9
x^10
\end{verbatim}

\textgreater plot2d(mxm2str(v),0,1): // plot functions

\begin{figure}
\centering
\pandocbounded{\includegraphics[keepaspectratio]{images/Nazwa Yuan Adelia Putri_23030630095_EMT4Plot2D-051.png}}
\caption{images/Nazwa\%20Yuan\%20Adelia\%20Putri\_23030630095\_EMT4Plot2D-051.png}
\end{figure}

Alternatif lain adalah dengan menggunakan bahasa matriks Euler.

Jika ekspresi menghasilkan matriks fungsi, dengan satu fungsi di setiap baris, semua fungsi ini akan diplot ke dalam satu plot.

Untuk ini, gunakan vektor parameter dalam bentuk vektor kolom. Jika array warna ditambahkan, itu akan digunakan untuk setiap baris plot.

\textgreater n=(1:10)'; plot2d(``x\^{}n'',0,1,color=1:10):

\begin{figure}
\centering
\pandocbounded{\includegraphics[keepaspectratio]{images/Nazwa Yuan Adelia Putri_23030630095_EMT4Plot2D-052.png}}
\caption{images/Nazwa\%20Yuan\%20Adelia\%20Putri\_23030630095\_EMT4Plot2D-052.png}
\end{figure}

Ekspresi dan fungsi satu baris dapat melihat variabel global.

Jika Anda tidak dapat menggunakan variabel global, Anda perlu menggunakan fungsi dengan parameter tambahan, dan meneruskan parameter ini sebagai parameter titik koma.

Berhati-hatilah, untuk meletakkan semua parameter yang ditetapkan di akhir perintah plot2d. Dalam contoh kita meneruskan a=5 ke fungsi f, yang kita plot dari -10 hingga 10.

\textgreater function f(x,a) := 1/a*exp(-x\^{}2/a); \ldots{}\\
\textgreater{} plot2d(``f'',-10,10;5,thickness=2,title=``a=5''):

\begin{figure}
\centering
\pandocbounded{\includegraphics[keepaspectratio]{images/Nazwa Yuan Adelia Putri_23030630095_EMT4Plot2D-053.png}}
\caption{images/Nazwa\%20Yuan\%20Adelia\%20Putri\_23030630095\_EMT4Plot2D-053.png}
\end{figure}

Atau, gunakan koleksi dengan nama fungsi dan semua parameter tambahan. Daftar khusus ini disebut koleksi panggilan, dan itu adalah cara yang lebih disukai untuk meneruskan argumen ke fungsi yang dengan sendirinya diteruskan sebagai argumen ke fungsi lain.

Dalam contoh berikut, kami menggunakan loop untuk memplot beberapa fungsi (lihat tutorial tentang pemrograman untuk loop).

\textgreater plot2d(\{\{``f'',1\}\},-10,10); \ldots{}\\
\textgreater{} for a=2:10; plot2d(\{\{``f'',a\}\},\textgreater add); end:

\begin{figure}
\centering
\pandocbounded{\includegraphics[keepaspectratio]{images/Nazwa Yuan Adelia Putri_23030630095_EMT4Plot2D-054.png}}
\caption{images/Nazwa\%20Yuan\%20Adelia\%20Putri\_23030630095\_EMT4Plot2D-054.png}
\end{figure}

Kami dapat mencapai hasil yang sama dengan cara berikut menggunakan bahasa matriks EMT. Setiap baris matriks f(x,a) adalah satu fungsi. Selain itu, kita dapat mengatur warna untuk setiap baris matriks. Klik dua kali pada fungsi getspectral() untuk penjelasannya.

\textgreater x=-10:0.01:10; a=(1:10)'; plot2d(x,f(x,a),color=getspectral(a/10)):

\begin{figure}
\centering
\pandocbounded{\includegraphics[keepaspectratio]{images/Nazwa Yuan Adelia Putri_23030630095_EMT4Plot2D-055.png}}
\caption{images/Nazwa\%20Yuan\%20Adelia\%20Putri\_23030630095\_EMT4Plot2D-055.png}
\end{figure}

\section{Label Teks}\label{label-teks}

Dekorasi sederhana bisa

\begin{itemize}
\item
  judul dengan judul=``\ldots{}''
\item
  x- dan y-label dengan xl=``\ldots{}'', yl=``\ldots{}''
\item
  label teks lain dengan label(``\ldots{}'',x,y)
\end{itemize}

Perintah label akan memplot ke dalam plot saat ini pada koordinat plot (x,y). Itu bisa mengambil argumen posisi.

\textgreater plot2d(``x\textsuperscript{3-x'',-1,2,title=''y=x}3-x'',yl=``y'',xl=``x''):

\begin{figure}
\centering
\pandocbounded{\includegraphics[keepaspectratio]{images/Nazwa Yuan Adelia Putri_23030630095_EMT4Plot2D-056.png}}
\caption{images/Nazwa\%20Yuan\%20Adelia\%20Putri\_23030630095\_EMT4Plot2D-056.png}
\end{figure}

\textgreater expr := ``log(x)/x''; \ldots{}\\
\textgreater{} plot2d(expr,0.5,5,title=``y=''+expr,xl=``x'',yl=``y''); \ldots{}\\
\textgreater{} label(``(1,0)'',1,0); label(``Max'',E,expr(E),pos=``lc''):

\begin{figure}
\centering
\pandocbounded{\includegraphics[keepaspectratio]{images/Nazwa Yuan Adelia Putri_23030630095_EMT4Plot2D-057.png}}
\caption{images/Nazwa\%20Yuan\%20Adelia\%20Putri\_23030630095\_EMT4Plot2D-057.png}
\end{figure}

Ada juga fungsi labelbox(), yang dapat menampilkan fungsi dan teks. Dibutuhkan vektor string dan warna, satu item untuk setiap fungsi.

\textgreater function f(x) \&= x\textsuperscript{2*exp(-x}2); \ldots{}\\
\textgreater{} plot2d(\&f(x),a=-3,b=3,c=-1,d=1); \ldots{}\\
\textgreater{} plot2d(\&diff(f(x),x),\textgreater add,color=blue,style=``--''); \ldots{}\\
\textgreater{} labelbox({[}``function'',``derivative''{]},styles={[}``-'',``--''{]}, \ldots{}\\
\textgreater{} colors={[}black,blue{]},w=0.4):

\begin{figure}
\centering
\pandocbounded{\includegraphics[keepaspectratio]{images/Nazwa Yuan Adelia Putri_23030630095_EMT4Plot2D-058.png}}
\caption{images/Nazwa\%20Yuan\%20Adelia\%20Putri\_23030630095\_EMT4Plot2D-058.png}
\end{figure}

Kotak ditambatkan di kanan atas secara default, tetapi \textgreater{} kiri menambatkannya di kiri atas. Anda dapat memindahkannya ke tempat yang Anda suka. Posisi jangkar adalah sudut kanan atas kotak, dan angkanya adalah pecahan dari ukuran jendela grafik. Lebarnya otomatis.

Untuk plot titik, kotak label juga berfungsi. Tambahkan parameter \textgreater points, atau vektor flag, satu untuk setiap label.

Dalam contoh berikut, hanya ada satu fungsi. Jadi kita bisa menggunakan string sebagai pengganti vektor string. Kami mengatur warna teks menjadi hitam untuk contoh ini.

\textgreater n=10; plot2d(0:n,bin(n,0:n),\textgreater addpoints); \ldots{}\\
\textgreater{} labelbox(``Binomials'',styles=``{[}{]}'',\textgreater points,x=0.1,y=0.1, \ldots{}\\
\textgreater{} tcolor=black,\textgreater left):

\begin{figure}
\centering
\pandocbounded{\includegraphics[keepaspectratio]{images/Nazwa Yuan Adelia Putri_23030630095_EMT4Plot2D-059.png}}
\caption{images/Nazwa\%20Yuan\%20Adelia\%20Putri\_23030630095\_EMT4Plot2D-059.png}
\end{figure}

Gaya plot ini juga tersedia di statplot(). Seperti di plot2d() warna dapat diatur untuk setiap baris plot. Ada lebih banyak plot khusus untuk keperluan statistik (lihat tutorial tentang statistik).

\textgreater statplot(1:10,random(2,10),color={[}red,blue{]}):

\begin{figure}
\centering
\pandocbounded{\includegraphics[keepaspectratio]{images/Nazwa Yuan Adelia Putri_23030630095_EMT4Plot2D-060.png}}
\caption{images/Nazwa\%20Yuan\%20Adelia\%20Putri\_23030630095\_EMT4Plot2D-060.png}
\end{figure}

Fitur serupa adalah fungsi textbox().

Lebar secara default adalah lebar maksimal dari baris teks. Tapi itu bisa diatur oleh pengguna juga.

\textgreater function f(x) \&= exp(-x)*sin(2*pi*x); \ldots{}\\
\textgreater{} plot2d(``f(x)'',0,2pi); \ldots{}\\
\textgreater{} textbox(latex(``\textbackslash text\{Example of a damped oscillation\}\textbackslash{} f(x)=e\^{}\{-x\}sin(2\textbackslash pi x)''),w=0.85):

\begin{figure}
\centering
\pandocbounded{\includegraphics[keepaspectratio]{images/Nazwa Yuan Adelia Putri_23030630095_EMT4Plot2D-061.png}}
\caption{images/Nazwa\%20Yuan\%20Adelia\%20Putri\_23030630095\_EMT4Plot2D-061.png}
\end{figure}

Label teks, judul, kotak label, dan teks lainnya dapat berisi string Unicode (lihat sintaks EMT untuk mengetahui lebih lanjut tentang string Unicode).

\textgreater plot2d(``x\^{}3-x'',title=u''x → x³ - x''):

\begin{figure}
\centering
\pandocbounded{\includegraphics[keepaspectratio]{images/Nazwa Yuan Adelia Putri_23030630095_EMT4Plot2D-062.png}}
\caption{images/Nazwa\%20Yuan\%20Adelia\%20Putri\_23030630095\_EMT4Plot2D-062.png}
\end{figure}

Label pada sumbu x dan y bisa vertikal, begitu juga sumbunya.

\textgreater plot2d(``sinc(x)'',0,2pi,xl=``x'',yl=u''x → sinc(x)``,\textgreater vertical):

\begin{figure}
\centering
\pandocbounded{\includegraphics[keepaspectratio]{images/Nazwa Yuan Adelia Putri_23030630095_EMT4Plot2D-063.png}}
\caption{images/Nazwa\%20Yuan\%20Adelia\%20Putri\_23030630095\_EMT4Plot2D-063.png}
\end{figure}

\section{LaTeX}\label{latex}

Anda juga dapat memplot rumus LaTeX jika Anda telah menginstal sistem LaTeX. Saya merekomendasikan MiKTeX. Jalur ke biner ``lateks'' dan ``dvipng'' harus berada di jalur sistem, atau Anda harus mengatur LaTeX di menu opsi.

Perhatikan, bahwa penguraian LaTeX lambat. Jika Anda ingin menggunakan LaTeX dalam plot animasi, Anda harus memanggil latex() sebelum loop sekali dan menggunakan hasilnya (gambar dalam matriks RGB).

Dalam plot berikut, kami menggunakan LaTeX untuk label x dan y, label, kotak label, dan judul plot.

\textgreater plot2d(``exp(-x)*sin(x)/x'',a=0,b=2pi,c=0,d=1,grid=6,color=blue, \ldots{}\\
\textgreater{} title=latex(``\textbackslash text\{Function \(\\Phi\)\}''), \ldots{}\\
\textgreater{} xl=latex(``\textbackslash phi''),yl=latex(``\textbackslash Phi(\textbackslash phi)'')); \ldots{}\\
\textgreater{} textbox( \ldots{}\\
\textgreater{} latex(``\textbackslash Phi(\textbackslash phi) = e\^{}\{-\textbackslash phi\} \textbackslash frac\{\textbackslash sin(\textbackslash phi)\}\{\textbackslash phi\}''),x=0.8,y=0.5); \ldots{}\\
\textgreater{} label(latex(``\textbackslash Phi'',color=blue),1,0.4):

\begin{figure}
\centering
\pandocbounded{\includegraphics[keepaspectratio]{images/Nazwa Yuan Adelia Putri_23030630095_EMT4Plot2D-064.png}}
\caption{images/Nazwa\%20Yuan\%20Adelia\%20Putri\_23030630095\_EMT4Plot2D-064.png}
\end{figure}

Seringkali, kami menginginkan spasi dan label teks non-konformal pada sumbu x. Kita dapat menggunakan xaxis() dan yaxis() seperti yang akan kita tunjukkan nanti.

Cara termudah adalah dengan membuat plot kosong dengan bingkai menggunakan grid=4, lalu menambahkan grid dengan ygrid() dan xgrid(). Dalam contoh berikut, kami menggunakan tiga string LaTeX untuk label pada sumbu x dengan xtick().

\textgreater plot2d(``sinc(x)'',0,2pi,grid=4,\textless ticks); \ldots{}\\
\textgreater{} ygrid(-2:0.5:2,grid=6); \ldots{}\\
\textgreater{} xgrid({[}0:2{]}*pi,\textless ticks,grid=6); \ldots{}\\
\textgreater{} xtick({[}0,pi,2pi{]},{[}``0'',``\textbackslash pi'',``2\textbackslash pi''{]},\textgreater latex):

\begin{figure}
\centering
\pandocbounded{\includegraphics[keepaspectratio]{images/Nazwa Yuan Adelia Putri_23030630095_EMT4Plot2D-065.png}}
\caption{images/Nazwa\%20Yuan\%20Adelia\%20Putri\_23030630095\_EMT4Plot2D-065.png}
\end{figure}

Tentu saja, fungsi juga dapat digunakan.

\textgreater function map f(x) \ldots{}

\begin{verbatim}
if x>0 then return x^4
else return x^2
endif
endfunction
\end{verbatim}

Parameter ``peta'' membantu menggunakan fungsi untuk vektor. Untuk

plot, itu tidak perlu. Tetapi untuk mendemonstrasikan vektorisasi itu

berguna, kami menambahkan beberapa poin kunci ke plot di x=-1, x=0 dan x=1.

Pada plot berikut, kami juga memasukkan beberapa kode LaTeX. Kami menggunakannya untuk

dua label dan kotak teks. Tentu saja, Anda hanya akan dapat menggunakan

LaTeX jika Anda telah menginstal LaTeX dengan benar.

\textgreater plot2d(``f'',-1,1,xl=``x'',yl=``f(x)'',grid=6); \ldots{}\\
\textgreater{} plot2d({[}-1,0,1{]},f({[}-1,0,1{]}),\textgreater points,\textgreater add); \ldots{}\\
\textgreater{} label(latex(``x\^{}3''),0.72,f(0.72)); \ldots{}\\
\textgreater{} label(latex(``x\^{}2''),-0.52,f(-0.52),pos=``ll''); \ldots{}\\
\textgreater{} textbox( \ldots{}\\
\textgreater{} latex(``f(x)=\textbackslash begin\{cases\} x\^{}3 \& x\textgreater0 \textbackslash\textbackslash{} x\^{}2 \& x \textbackslash le 0\textbackslash end\{cases\}''), \ldots{}\\
\textgreater{} x=0.7,y=0.2):

\begin{figure}
\centering
\pandocbounded{\includegraphics[keepaspectratio]{images/Nazwa Yuan Adelia Putri_23030630095_EMT4Plot2D-066.png}}
\caption{images/Nazwa\%20Yuan\%20Adelia\%20Putri\_23030630095\_EMT4Plot2D-066.png}
\end{figure}

\section{Interaksi pengguna}\label{interaksi-pengguna}

Saat memplot fungsi atau ekspresi, parameter \textgreater user memungkinkan pengguna untuk memperbesar dan menggeser plot dengan tombol kursor atau mouse. Pengguna dapat

\begin{itemize}
\item
  perbesar dengan + atau -
\item
  pindahkan plot dengan tombol kursor
\item
  pilih jendela plot dengan mouse
\item
  atur ulang tampilan dengan spasi
\item
  keluar dengan kembali
\end{itemize}

Tombol spasi akan mengatur ulang plot ke jendela plot asli.

Saat memplot data, flag \textgreater user hanya akan menunggu penekanan tombol.

\textgreater plot2d(\{\{``x\^{}3-a*x'',a=1\}\},\textgreater user,title=``Press any key!''):

\begin{figure}
\centering
\pandocbounded{\includegraphics[keepaspectratio]{images/Nazwa Yuan Adelia Putri_23030630095_EMT4Plot2D-067.png}}
\caption{images/Nazwa\%20Yuan\%20Adelia\%20Putri\_23030630095\_EMT4Plot2D-067.png}
\end{figure}

\textgreater plot2d(``exp(x)*sin(x)'',user=true, \ldots{}\\
\textgreater{} title=``+/- or cursor keys (return to exit)''):

\begin{figure}
\centering
\pandocbounded{\includegraphics[keepaspectratio]{images/Nazwa Yuan Adelia Putri_23030630095_EMT4Plot2D-068.png}}
\caption{images/Nazwa\%20Yuan\%20Adelia\%20Putri\_23030630095\_EMT4Plot2D-068.png}
\end{figure}

Berikut ini menunjukkan cara interaksi pengguna tingkat lanjut (lihat tutorial tentang pemrograman untuk detailnya).

Fungsi bawaan mousedrag() menunggu event mouse atau keyboard. Ini melaporkan mouse ke bawah, mouse dipindahkan atau mouse ke atas, dan penekanan tombol. Fungsi dragpoints() memanfaatkan ini, dan memungkinkan pengguna menyeret titik mana pun dalam plot.

Kita membutuhkan fungsi plot terlebih dahulu. Sebagai contoh, kita interpolasi dalam 5 titik dengan polinomial. Fungsi harus diplot ke area plot tetap.

\textgreater function plotf(xp,yp,select) \ldots{}

\begin{verbatim}
  d=interp(xp,yp);
  plot2d("interpval(xp,d,x)";d,xp,r=2);
  plot2d(xp,yp,>points,>add);
  if select>0 then
    plot2d(xp[select],yp[select],color=red,>points,>add);
  endif;
  title("Drag one point, or press space or return!");
endfunction
\end{verbatim}

Perhatikan parameter titik koma di plot2d (d dan xp), yang diteruskan ke evaluasi fungsi interp(). Tanpa ini, kita harus menulis fungsi plotinterp() terlebih dahulu, mengakses nilai secara global.

Sekarang kita menghasilkan beberapa nilai acak, dan membiarkan pengguna menyeret poin.

\textgreater t=-1:0.5:1; dragpoints(``plotf'',t,random(size(t))-0.5):

\begin{figure}
\centering
\pandocbounded{\includegraphics[keepaspectratio]{images/Nazwa Yuan Adelia Putri_23030630095_EMT4Plot2D-069.png}}
\caption{images/Nazwa\%20Yuan\%20Adelia\%20Putri\_23030630095\_EMT4Plot2D-069.png}
\end{figure}

Ada juga fungsi, yang memplot fungsi lain tergantung pada vektor parameter, dan memungkinkan pengguna menyesuaikan parameter ini.

Pertama kita membutuhkan fungsi plot.

\textgreater function plotf({[}a,b{]}) := plot2d(``exp(a*x)*cos(2pi*b*x)'',0,2pi;a,b);

Kemudian kita membutuhkan nama untuk parameter, nilai awal dan matriks rentang nx2, opsional baris judul.

Ada slider interaktif, yang dapat mengatur nilai oleh pengguna. Fungsi dragvalues() menyediakan ini.

\textgreater dragvalues(``plotf'',{[}``a'',``b''{]},{[}-1,2{]},{[}{[}-2,2{]};{[}1,10{]}{]}, \ldots{}\\
\textgreater{} heading=``Drag these values:'',hcolor=black):

\begin{figure}
\centering
\pandocbounded{\includegraphics[keepaspectratio]{images/Nazwa Yuan Adelia Putri_23030630095_EMT4Plot2D-070.png}}
\caption{images/Nazwa\%20Yuan\%20Adelia\%20Putri\_23030630095\_EMT4Plot2D-070.png}
\end{figure}

Dimungkinkan untuk membatasi nilai yang diseret ke bilangan bulat. Sebagai contoh, kita menulis fungsi plot, yang memplot polinomial Taylor derajat n ke fungsi kosinus.

\textgreater function plotf(n) \ldots{}

\begin{verbatim}
plot2d("cos(x)",0,2pi,>square,grid=6);
plot2d(&"taylor(cos(x),x,0,@n)",color=blue,>add);
textbox("Taylor polynomial of degree "+n,0.1,0.02,style="t",>left);
endfunction
\end{verbatim}

Sekarang kami mengizinkan derajat n bervariasi dari 0 hingga 20 dalam 20 pemberhentian. Hasil dragvalues() digunakan untuk memplot sketsa dengan n ini, dan untuk memasukkan plot ke dalam buku catatan.

\textgreater nd=dragvalues(``plotf'',``degree'',2,{[}0,20{]},20,y=0.8, \ldots{}\\
\textgreater{} heading=``Drag the value:''); \ldots{}\\
\textgreater{} plotf(nd):

\begin{figure}
\centering
\pandocbounded{\includegraphics[keepaspectratio]{images/Nazwa Yuan Adelia Putri_23030630095_EMT4Plot2D-071.png}}
\caption{images/Nazwa\%20Yuan\%20Adelia\%20Putri\_23030630095\_EMT4Plot2D-071.png}
\end{figure}

Berikut ini adalah demonstrasi sederhana dari fungsi tersebut. Pengguna dapat menggambar di atas jendela plot, meninggalkan jejak poin.

\textgreater function dragtest \ldots{}

\begin{verbatim}
  plot2d(none,r=1,title="Drag with the mouse, or press any key!");
  start=0;
  repeat
    {flag,m,time}=mousedrag();
    if flag==0 then return; endif;
    if flag==2 then
      hold on; mark(m[1],m[2]); hold off;
    endif;
  end
endfunction
\end{verbatim}

\textgreater dragtest // lihat hasilnya dan cobalah lakukan!

\section{Gaya~Plot~2D}\label{gaya-plot-2d}

Secara default, EMT menghitung tick sumbu otomatis dan menambahkan label ke setiap tick. Ini dapat diubah dengan parameter grid. Gaya default sumbu dan label dapat dimodifikasi. Selain itu, label dan judul dapat ditambahkan secara manual. Untuk mengatur ulang ke gaya default, gunakan reset().

\textgreater aspect();

\textgreater figure(3,4); \ldots{}\\
\textgreater{} figure(1); plot2d(``x\^{}3-x'',grid=0); \ldots{} // no grid, frame or axis

\textgreater{} figure(2); plot2d(``x\^{}3-x'',grid=1); \ldots{} // x-y-axis

\textgreater{} figure(3); plot2d(``x\^{}3-x'',grid=2); \ldots{} // default ticks

\textgreater{} figure(4); plot2d(``x\^{}3-x'',grid=3); \ldots{} // x-y- axis with labels inside

\textgreater{} figure(5); plot2d(``x\^{}3-x'',grid=4); \ldots{} // no ticks, only labels

\textgreater{} figure(6); plot2d(``x\^{}3-x'',grid=5); \ldots{} // default, but no margin

\textgreater{} figure(7); plot2d(``x\^{}3-x'',grid=6); \ldots{} // axes only

\textgreater{} figure(8); plot2d(``x\^{}3-x'',grid=7); \ldots{} // axes only, ticks at axis

\textgreater{} figure(9); plot2d(``x\^{}3-x'',grid=8); \ldots{} // axes only, finer ticks at axis

\textgreater{} figure(10); plot2d(``x\^{}3-x'',grid=9); \ldots{} // default, small ticks inside

\textgreater{} figure(11); plot2d(``x\^{}3-x'',grid=10); \ldots// no ticks, axes only

\textgreater{} figure(0):

\begin{figure}
\centering
\pandocbounded{\includegraphics[keepaspectratio]{images/Nazwa Yuan Adelia Putri_23030630095_EMT4Plot2D-072.png}}
\caption{images/Nazwa\%20Yuan\%20Adelia\%20Putri\_23030630095\_EMT4Plot2D-072.png}
\end{figure}

Parameter \textless frame mematikan frame, dan framecolor=blue mengatur frame ke warna biru.

Jika Anda ingin centang sendiri, Anda dapat menggunakan style=0, dan menambahkan semuanya nanti.

\textgreater aspect(1.5);

\textgreater plot2d(``x\^{}3-x'',grid=0); // plot

\textgreater frame; xgrid({[}-1,0,1{]}); ygrid(0): // add frame and grid

\begin{figure}
\centering
\pandocbounded{\includegraphics[keepaspectratio]{images/Nazwa Yuan Adelia Putri_23030630095_EMT4Plot2D-073.png}}
\caption{images/Nazwa\%20Yuan\%20Adelia\%20Putri\_23030630095\_EMT4Plot2D-073.png}
\end{figure}

Untuk judul plot dan label sumbu, lihat contoh berikut.

\textgreater plot2d(``exp(x)'',-1,1);

\textgreater textcolor(black); // set the text color to black

\textgreater title(latex(``y=e\^{}x'')); // title above the plot

\textgreater xlabel(latex(``x'')); // ``x'' for x-axis

\textgreater ylabel(latex(``y''),\textgreater vertical); // vertical ``y'' for y-axis

\textgreater label(latex(``(0,1)''),0,1,color=blue): // label a point

\begin{figure}
\centering
\pandocbounded{\includegraphics[keepaspectratio]{images/Nazwa Yuan Adelia Putri_23030630095_EMT4Plot2D-074.png}}
\caption{images/Nazwa\%20Yuan\%20Adelia\%20Putri\_23030630095\_EMT4Plot2D-074.png}
\end{figure}

Sumbu dapat digambar secara terpisah dengan xaxis() dan yaxis().

\textgreater plot2d(``x\^{}3-x'',\textless grid,\textless frame);

\textgreater xaxis(0,xx=-2:1,style=``-\textgreater{}''); yaxis(0,yy=-5:5,style=``-\textgreater{}''):

\begin{figure}
\centering
\pandocbounded{\includegraphics[keepaspectratio]{images/Nazwa Yuan Adelia Putri_23030630095_EMT4Plot2D-075.png}}
\caption{images/Nazwa\%20Yuan\%20Adelia\%20Putri\_23030630095\_EMT4Plot2D-075.png}
\end{figure}

Teks pada plot dapat diatur dengan label(). Dalam contoh berikut, ``lc'' berarti tengah bawah. Ini mengatur posisi label relatif terhadap koordinat plot.

\textgreater function f(x) \&= x\^{}3-x

\begin{verbatim}
                                 3
                                x  - x
\end{verbatim}

\textgreater plot2d(f,-1,1,\textgreater square);

\textgreater x0=fmin(f,0,1); // compute point of minimum

\textgreater label(``Rel. Min.'',x0,f(x0),pos=``lc''): // add a label there

\begin{figure}
\centering
\pandocbounded{\includegraphics[keepaspectratio]{images/Nazwa Yuan Adelia Putri_23030630095_EMT4Plot2D-076.png}}
\caption{images/Nazwa\%20Yuan\%20Adelia\%20Putri\_23030630095\_EMT4Plot2D-076.png}
\end{figure}

Ada juga kotak teks.

\textgreater plot2d(\&f(x),-1,1,-2,2); // function

\textgreater plot2d(\&diff(f(x),x),\textgreater add,style=``--'',color=red); // derivative

\textgreater labelbox({[}``f'',``f'''{]},{[}``-'',``--''{]},{[}black,red{]}): // label box

\begin{figure}
\centering
\pandocbounded{\includegraphics[keepaspectratio]{images/Nazwa Yuan Adelia Putri_23030630095_EMT4Plot2D-077.png}}
\caption{images/Nazwa\%20Yuan\%20Adelia\%20Putri\_23030630095\_EMT4Plot2D-077.png}
\end{figure}

\textgreater plot2d({[}``exp(x)'',``1+x''{]},color={[}black,blue{]},style={[}``-'',``-.-''{]}):

\begin{figure}
\centering
\pandocbounded{\includegraphics[keepaspectratio]{images/Nazwa Yuan Adelia Putri_23030630095_EMT4Plot2D-078.png}}
\caption{images/Nazwa\%20Yuan\%20Adelia\%20Putri\_23030630095\_EMT4Plot2D-078.png}
\end{figure}

\textgreater gridstyle(``-\textgreater{}'',color=gray,textcolor=gray,framecolor=gray); \ldots{}\\
\textgreater{} plot2d(``x\^{}3-x'',grid=1); \ldots{}\\
\textgreater{} settitle(``y=x\^{}3-x'',color=black); \ldots{}\\
\textgreater{} label(``x'',2,0,pos=``bc'',color=gray); \ldots{}\\
\textgreater{} label(``y'',0,6,pos=``cl'',color=gray); \ldots{}\\
\textgreater{} reset():

\begin{figure}
\centering
\pandocbounded{\includegraphics[keepaspectratio]{images/Nazwa Yuan Adelia Putri_23030630095_EMT4Plot2D-079.png}}
\caption{images/Nazwa\%20Yuan\%20Adelia\%20Putri\_23030630095\_EMT4Plot2D-079.png}
\end{figure}

Untuk kontrol lebih, sumbu x dan sumbu y dapat dilakukan secara manual.

Perintah fullwindow() memperluas jendela plot karena kita tidak lagi membutuhkan tempat untuk label di luar jendela plot. Gunakan shrinkwindow() atau reset() untuk mengatur ulang ke default.

\textgreater fullwindow; \ldots{}\\
\textgreater{} gridstyle(color=darkgray,textcolor=darkgray); \ldots{}\\
\textgreater{} plot2d({[}``2\textsuperscript{x'',''1'',''2}(-x)''{]},a=-2,b=2,c=0,d=4,\textless grid,color=4:6,\textless frame); \ldots{}\\
\textgreater{} xaxis(0,-2:1,style=``-\textgreater{}''); xaxis(0,2,``x'',\textless axis); \ldots{}\\
\textgreater{} yaxis(0,4,``y'',style=``-\textgreater{}''); \ldots{}\\
\textgreater{} yaxis(-2,1:4,\textgreater left); \ldots{}\\
\textgreater{} yaxis(2,2\^{}(-2:2),style=``.'',\textless left); \ldots{}\\
\textgreater{} labelbox({[}``2\textsuperscript{x'',''1'',''2}-x''{]},colors=4:6,x=0.8,y=0.2); \ldots{}\\
\textgreater{} reset:

\begin{figure}
\centering
\pandocbounded{\includegraphics[keepaspectratio]{images/Nazwa Yuan Adelia Putri_23030630095_EMT4Plot2D-080.png}}
\caption{images/Nazwa\%20Yuan\%20Adelia\%20Putri\_23030630095\_EMT4Plot2D-080.png}
\end{figure}

Berikut adalah contoh lain, di mana string Unicode digunakan dan sumbu di luar area plot.

\textgreater aspect(1.5);

\textgreater plot2d({[}``sin(x)'',``cos(x)''{]},0,2pi,color={[}red,green{]},\textless grid,\textless frame); \ldots{}\\
\textgreater{} xaxis(-1.1,(0:2)*pi,xt={[}``0'',u''π``,u''2π''{]},style=``-'',\textgreater ticks,\textgreater zero); \ldots{}\\
\textgreater{} xgrid((0:0.5:2)*pi,\textless ticks); \ldots{}\\
\textgreater{} yaxis(-0.1*pi,-1:0.2:1,style=``-'',\textgreater zero,\textgreater grid); \ldots{}\\
\textgreater{} labelbox({[}``sin'',``cos''{]},colors={[}red,green{]},x=0.5,y=0.2,\textgreater left); \ldots{}\\
\textgreater{} xlabel(u''φ``); ylabel(u''f(φ)``):

\begin{figure}
\centering
\pandocbounded{\includegraphics[keepaspectratio]{images/Nazwa Yuan Adelia Putri_23030630095_EMT4Plot2D-081.png}}
\caption{images/Nazwa\%20Yuan\%20Adelia\%20Putri\_23030630095\_EMT4Plot2D-081.png}
\end{figure}

\chapter{Merencanakan Data 2D}\label{merencanakan-data-2d}

Jika x dan y adalah vektor data, data ini akan digunakan sebagai koordinat x dan y dari suatu kurva. Dalam hal ini, a, b, c, dan d, atau radius r dapat ditentukan, atau jendela plot akan menyesuaikan secara otomatis dengan data. Atau, \textgreater persegi dapat diatur untuk menjaga rasio aspek persegi.

Memplot ekspresi hanyalah singkatan untuk plot data. Untuk plot data, Anda memerlukan satu atau beberapa baris nilai x, dan satu atau beberapa baris nilai y. Dari rentang dan nilai-x, fungsi plot2d akan menghitung data yang akan diplot, secara default dengan evaluasi fungsi yang adaptif. Untuk plot titik gunakan ``\textgreater titik'', untuk garis campuran dan titik gunakan ``\textgreater tambahan''.

Tapi Anda bisa memasukkan data secara langsung.

\begin{itemize}
\item
  Gunakan vektor baris untuk x dan y untuk satu fungsi.
\item
  Matriks untuk x dan y diplot baris demi baris.
\end{itemize}

Berikut adalah contoh dengan satu baris untuk x dan y.

\textgreater x=-10:0.1:10; y=exp(-x\^{}2)*x; plot2d(x,y):

\begin{figure}
\centering
\pandocbounded{\includegraphics[keepaspectratio]{images/Nazwa Yuan Adelia Putri_23030630095_EMT4Plot2D-082.png}}
\caption{images/Nazwa\%20Yuan\%20Adelia\%20Putri\_23030630095\_EMT4Plot2D-082.png}
\end{figure}

Data juga dapat diplot sebagai titik. Gunakan poin=true untuk ini. Plotnya bekerja seperti poligon, tetapi hanya menggambar sudut-sudutnya.

\begin{itemize}
\tightlist
\item
  style=``\ldots{}'': Pilih dari ``{[}{]}'', ``\textless\textgreater{}'', ``o'', ``.'', ``..'', ``+'', ``*``,''{[}{]}\#``,
\item
  ``\textless{} \textgreater\#'', ``o\#'', ``..\#'', ``\#'', ``\textbar{}''.
\end{itemize}

Untuk memplot set poin gunakan \textgreater points. Jika warna adalah vektor warna, setiap titik

mendapat warna yang berbeda. Untuk matriks koordinat dan vektor kolom, warna berlaku untuk baris matriks.

Parameter \textgreater addpoints menambahkan titik ke segmen garis untuk plot data.

\textgreater xdata={[}1,1.5,2.5,3,4{]}; ydata={[}3,3.1,2.8,2.9,2.7{]}; // data

\textgreater plot2d(xdata,ydata,a=0.5,b=4.5,c=2.5,d=3.5,style=``.''); // lines

\textgreater plot2d(xdata,ydata,\textgreater points,\textgreater add,style=``o''): // add points

\begin{figure}
\centering
\pandocbounded{\includegraphics[keepaspectratio]{images/Nazwa Yuan Adelia Putri_23030630095_EMT4Plot2D-083.png}}
\caption{images/Nazwa\%20Yuan\%20Adelia\%20Putri\_23030630095\_EMT4Plot2D-083.png}
\end{figure}

\textgreater p=polyfit(xdata,ydata,1); // get regression line

\textgreater plot2d(``polyval(p,x)'',\textgreater add,color=red): // add plot of line

\begin{figure}
\centering
\pandocbounded{\includegraphics[keepaspectratio]{images/Nazwa Yuan Adelia Putri_23030630095_EMT4Plot2D-084.png}}
\caption{images/Nazwa\%20Yuan\%20Adelia\%20Putri\_23030630095\_EMT4Plot2D-084.png}
\end{figure}

\chapter{Menggambar Daerah Yang Dibatasi Kurva}\label{menggambar-daerah-yang-dibatasi-kurva}

Plot data benar-benar poligon. Kita juga dapat memplot kurva atau kurva terisi.

\begin{itemize}
\item
  terisi=benar mengisi plot.
\item
  style=``\ldots{}'': Pilih dari ``\#'', ``/'', ``",''/``.
\item
  fillcolor: Lihat di atas untuk warna yang tersedia.
\end{itemize}

Warna isian ditentukan oleh argumen ``fillcolor'', dan pada \textless outline opsional mencegah menggambar batas untuk semua gaya kecuali yang default.

\textgreater t=linspace(0,2pi,1000); // parameter for curve

\textgreater x=sin(t)*exp(t/pi); y=cos(t)*exp(t/pi); // x(t) and y(t)

\textgreater figure(1,2); aspect(16/9)

\textgreater figure(1); plot2d(x,y,r=10); // plot curve

\textgreater figure(2); plot2d(x,y,r=10,\textgreater filled,style=``/'',fillcolor=red); // fill curve

\textgreater figure(0):

\begin{figure}
\centering
\pandocbounded{\includegraphics[keepaspectratio]{images/Nazwa Yuan Adelia Putri_23030630095_EMT4Plot2D-085.png}}
\caption{images/Nazwa\%20Yuan\%20Adelia\%20Putri\_23030630095\_EMT4Plot2D-085.png}
\end{figure}

Dalam contoh berikut kami memplot elips terisi dan dua segi enam terisi menggunakan kurva tertutup dengan 6 titik dengan gaya isian berbeda.

\textgreater x=linspace(0,2pi,1000); plot2d(sin(x),cos(x)*0.5,r=1,\textgreater filled,style=``/''):

\begin{figure}
\centering
\pandocbounded{\includegraphics[keepaspectratio]{images/Nazwa Yuan Adelia Putri_23030630095_EMT4Plot2D-086.png}}
\caption{images/Nazwa\%20Yuan\%20Adelia\%20Putri\_23030630095\_EMT4Plot2D-086.png}
\end{figure}

\textgreater t=linspace(0,2pi,6); \ldots{}\\
\textgreater{} plot2d(cos(t),sin(t),\textgreater filled,style=``/'',fillcolor=red,r=1.2):

\begin{figure}
\centering
\pandocbounded{\includegraphics[keepaspectratio]{images/Nazwa Yuan Adelia Putri_23030630095_EMT4Plot2D-087.png}}
\caption{images/Nazwa\%20Yuan\%20Adelia\%20Putri\_23030630095\_EMT4Plot2D-087.png}
\end{figure}

\textgreater t=linspace(0,2pi,6); plot2d(cos(t),sin(t),\textgreater filled,style=``\#''):

\begin{figure}
\centering
\pandocbounded{\includegraphics[keepaspectratio]{images/Nazwa Yuan Adelia Putri_23030630095_EMT4Plot2D-088.png}}
\caption{images/Nazwa\%20Yuan\%20Adelia\%20Putri\_23030630095\_EMT4Plot2D-088.png}
\end{figure}

Contoh lainnya adalah segi empat, yang kita buat dengan 7 titik pada lingkaran satuan.

\textgreater t=linspace(0,2pi,7); \ldots{}\\
\textgreater{} plot2d(cos(t),sin(t),r=1,\textgreater filled,style=``/'',fillcolor=red):

\begin{figure}
\centering
\pandocbounded{\includegraphics[keepaspectratio]{images/Nazwa Yuan Adelia Putri_23030630095_EMT4Plot2D-089.png}}
\caption{images/Nazwa\%20Yuan\%20Adelia\%20Putri\_23030630095\_EMT4Plot2D-089.png}
\end{figure}

Berikut ini adalah himpunan nilai maksimal dari empat kondisi linier yang kurang dari atau sama dengan 3. Ini adalah A{[}k{]}.v\textless=3 untuk semua baris A. Untuk mendapatkan sudut yang bagus, kita menggunakan n yang relatif besar.

\textgreater A={[}2,1;1,2;-1,0;0,-1{]};

\textgreater function f(x,y) := max({[}x,y{]}.A');

\textgreater plot2d(``f'',r=4,level={[}0;3{]},color=green,n=111):

\begin{figure}
\centering
\pandocbounded{\includegraphics[keepaspectratio]{images/Nazwa Yuan Adelia Putri_23030630095_EMT4Plot2D-090.png}}
\caption{images/Nazwa\%20Yuan\%20Adelia\%20Putri\_23030630095\_EMT4Plot2D-090.png}
\end{figure}

Poin utama dari bahasa matriks adalah memungkinkan untuk menghasilkan tabel fungsi dengan mudah.

\textgreater t=linspace(0,2pi,1000); x=cos(3*t); y=sin(4*t);

Kami sekarang memiliki vektor x dan y nilai. plot2d() dapat memplot nilai-nilai ini

sebagai kurva yang menghubungkan titik-titik. Plotnya bisa diisi. Pada kasus ini

ini menghasilkan hasil yang bagus karena aturan lilitan, yang digunakan untuk

isi.

\textgreater plot2d(x,y,\textless grid,\textless frame,\textgreater filled):

\begin{figure}
\centering
\pandocbounded{\includegraphics[keepaspectratio]{images/Nazwa Yuan Adelia Putri_23030630095_EMT4Plot2D-091.png}}
\caption{images/Nazwa\%20Yuan\%20Adelia\%20Putri\_23030630095\_EMT4Plot2D-091.png}
\end{figure}

Sebuah vektor interval diplot terhadap nilai x sebagai daerah terisi

antara nilai interval bawah dan atas.

Hal ini dapat berguna untuk memplot kesalahan perhitungan. Tapi itu bisa

juga digunakan untuk memplot kesalahan statistik.

\textgreater t=0:0.1:1; \ldots{}\\
\textgreater{} plot2d(t,interval(t-random(size(t)),t+random(size(t))),style=``\textbar{}''); \ldots{}\\
\textgreater{} plot2d(t,t,add=true):

\begin{figure}
\centering
\pandocbounded{\includegraphics[keepaspectratio]{images/Nazwa Yuan Adelia Putri_23030630095_EMT4Plot2D-092.png}}
\caption{images/Nazwa\%20Yuan\%20Adelia\%20Putri\_23030630095\_EMT4Plot2D-092.png}
\end{figure}

Jika x adalah vektor yang diurutkan, dan y adalah vektor interval, maka plot2d akan memplot rentang interval yang terisi dalam bidang. Gaya isian sama dengan gaya poligon.

\textgreater t=-1:0.01:1; x=\textsubscript{t-0.01,t+0.01}; y=x\^{}3-x;

\textgreater plot2d(t,y):

\begin{figure}
\centering
\pandocbounded{\includegraphics[keepaspectratio]{images/Nazwa Yuan Adelia Putri_23030630095_EMT4Plot2D-093.png}}
\caption{images/Nazwa\%20Yuan\%20Adelia\%20Putri\_23030630095\_EMT4Plot2D-093.png}
\end{figure}

Dimungkinkan untuk mengisi wilayah nilai untuk fungsi tertentu. Untuk

ini, level harus berupa matriks 2xn. Baris pertama adalah batas bawah

dan baris kedua berisi batas atas.

\textgreater expr := ``2*x\textsuperscript{2+x*y+3*y}4+y''; // define an expression f(x,y)

\textgreater plot2d(expr,level={[}0;1{]},style=``-'',color=blue): // 0 \textless= f(x,y) \textless= 1

\begin{figure}
\centering
\pandocbounded{\includegraphics[keepaspectratio]{images/Nazwa Yuan Adelia Putri_23030630095_EMT4Plot2D-094.png}}
\caption{images/Nazwa\%20Yuan\%20Adelia\%20Putri\_23030630095\_EMT4Plot2D-094.png}
\end{figure}

Kami juga dapat mengisi rentang nilai seperti

\textgreater plot2d(``(x\textsuperscript{2+y}2)\textsuperscript{2-x}2+y\^{}2'',r=1.2,level={[}-1;0{]},style=``/''):

\begin{figure}
\centering
\pandocbounded{\includegraphics[keepaspectratio]{images/Nazwa Yuan Adelia Putri_23030630095_EMT4Plot2D-095.png}}
\caption{images/Nazwa\%20Yuan\%20Adelia\%20Putri\_23030630095\_EMT4Plot2D-095.png}
\end{figure}

\textgreater plot2d(``cos(x)'',``sin(x)\^{}3'',xmin=0,xmax=2pi,\textgreater filled,style=``/''):

\begin{figure}
\centering
\pandocbounded{\includegraphics[keepaspectratio]{images/Nazwa Yuan Adelia Putri_23030630095_EMT4Plot2D-096.png}}
\caption{images/Nazwa\%20Yuan\%20Adelia\%20Putri\_23030630095\_EMT4Plot2D-096.png}
\end{figure}

\chapter{Grafik Fungsi Parametrik}\label{grafik-fungsi-parametrik}

Nilai-x tidak perlu diurutkan. (x,y) hanya menggambarkan kurva. Jika x diurutkan, kurva tersebut merupakan grafik fungsi.

Dalam contoh berikut, kami memplot spiral

Kita perlu menggunakan banyak titik untuk tampilan yang halus atau fungsi adaptif() untuk mengevaluasi ekspresi (lihat fungsi adaptif() untuk lebih jelasnya).

\textgreater t=linspace(0,1,1000); \ldots{}\\
\textgreater{} plot2d(t*cos(2*pi*t),t*sin(2*pi*t),r=1):

\begin{figure}
\centering
\pandocbounded{\includegraphics[keepaspectratio]{images/Nazwa Yuan Adelia Putri_23030630095_EMT4Plot2D-097.png}}
\caption{images/Nazwa\%20Yuan\%20Adelia\%20Putri\_23030630095\_EMT4Plot2D-097.png}
\end{figure}

Atau, dimungkinkan untuk menggunakan dua ekspresi untuk kurva. Berikut ini plot kurva yang sama seperti di atas.

\textgreater plot2d(``x*cos(2*pi*x)'',``x*sin(2*pi*x)'',xmin=0,xmax=1,r=1):

\begin{figure}
\centering
\pandocbounded{\includegraphics[keepaspectratio]{images/Nazwa Yuan Adelia Putri_23030630095_EMT4Plot2D-098.png}}
\caption{images/Nazwa\%20Yuan\%20Adelia\%20Putri\_23030630095\_EMT4Plot2D-098.png}
\end{figure}

\textgreater t=linspace(0,1,1000); r=exp(-t); x=r*cos(2pi*t); y=r*sin(2pi*t);

\textgreater plot2d(x,y,r=1):

\begin{figure}
\centering
\pandocbounded{\includegraphics[keepaspectratio]{images/Nazwa Yuan Adelia Putri_23030630095_EMT4Plot2D-099.png}}
\caption{images/Nazwa\%20Yuan\%20Adelia\%20Putri\_23030630095\_EMT4Plot2D-099.png}
\end{figure}

Dalam contoh berikutnya, kami memplot kurva

dengan

\textgreater t=linspace(0,2pi,1000); r=1+sin(3*t)/2; x=r*cos(t); y=r*sin(t); \ldots{}\\
\textgreater{} plot2d(x,y,\textgreater filled,fillcolor=red,style=``/'',r=1.5):

\begin{figure}
\centering
\pandocbounded{\includegraphics[keepaspectratio]{images/Nazwa Yuan Adelia Putri_23030630095_EMT4Plot2D-100.png}}
\caption{images/Nazwa\%20Yuan\%20Adelia\%20Putri\_23030630095\_EMT4Plot2D-100.png}
\end{figure}

\chapter{Menggambar Grafik Bilangan Kompleks}\label{menggambar-grafik-bilangan-kompleks}

Array bilangan kompleks juga dapat diplot. Kemudian titik-titik grid akan terhubung. Jika sejumlah garis kisi ditentukan (atau vektor garis kisi 1x2) dalam argumen cgrid, hanya garis kisi tersebut yang terlihat.

Matriks bilangan kompleks akan secara otomatis diplot sebagai kisi di bidang kompleks.

Dalam contoh berikut, kami memplot gambar lingkaran satuan di bawah fungsi eksponensial. Parameter cgrid menyembunyikan beberapa kurva grid.

\textgreater aspect(); r=linspace(0,1,50); a=linspace(0,2pi,80)'; z=r*exp(I*a);\ldots{}\\
\textgreater{} plot2d(z,a=-1.25,b=1.25,c=-1.25,d=1.25,cgrid=10):

\begin{figure}
\centering
\pandocbounded{\includegraphics[keepaspectratio]{images/Nazwa Yuan Adelia Putri_23030630095_EMT4Plot2D-101.png}}
\caption{images/Nazwa\%20Yuan\%20Adelia\%20Putri\_23030630095\_EMT4Plot2D-101.png}
\end{figure}

\textgreater aspect(1.25); r=linspace(0,1,50); a=linspace(0,2pi,200)'; z=r*exp(I*a);

\textgreater plot2d(exp(z),cgrid={[}40,10{]}):

\begin{figure}
\centering
\pandocbounded{\includegraphics[keepaspectratio]{images/Nazwa Yuan Adelia Putri_23030630095_EMT4Plot2D-102.png}}
\caption{images/Nazwa\%20Yuan\%20Adelia\%20Putri\_23030630095\_EMT4Plot2D-102.png}
\end{figure}

\textgreater r=linspace(0,1,10); a=linspace(0,2pi,40)'; z=r*exp(I*a);

\textgreater plot2d(exp(z),\textgreater points,\textgreater add):

\begin{figure}
\centering
\pandocbounded{\includegraphics[keepaspectratio]{images/Nazwa Yuan Adelia Putri_23030630095_EMT4Plot2D-103.png}}
\caption{images/Nazwa\%20Yuan\%20Adelia\%20Putri\_23030630095\_EMT4Plot2D-103.png}
\end{figure}

Sebuah vektor bilangan kompleks secara otomatis diplot sebagai kurva pada bidang kompleks dengan bagian real dan bagian imajiner.

Dalam contoh, kami memplot lingkaran satuan dengan

\textgreater t=linspace(0,2pi,1000); \ldots{}\\
\textgreater{} plot2d(exp(I*t)+exp(4*I*t),r=2):

\begin{figure}
\centering
\pandocbounded{\includegraphics[keepaspectratio]{images/Nazwa Yuan Adelia Putri_23030630095_EMT4Plot2D-104.png}}
\caption{images/Nazwa\%20Yuan\%20Adelia\%20Putri\_23030630095\_EMT4Plot2D-104.png}
\end{figure}

\chapter{Plot~Statistik}\label{plot-statistik}

Ada banyak fungsi yang dikhususkan pada plot statistik. Salah satu plot yang sering digunakan adalah plot kolom.

Jumlah kumulatif dari nilai terdistribusi 0-1-normal menghasilkan jalan acak.

\textgreater plot2d(cumsum(randnormal(1,1000))):

\begin{figure}
\centering
\pandocbounded{\includegraphics[keepaspectratio]{images/Nazwa Yuan Adelia Putri_23030630095_EMT4Plot2D-105.png}}
\caption{images/Nazwa\%20Yuan\%20Adelia\%20Putri\_23030630095\_EMT4Plot2D-105.png}
\end{figure}

Menggunakan dua baris menunjukkan jalan dalam dua dimensi.

\textgreater X=cumsum(randnormal(2,1000)); plot2d(X{[}1{]},X{[}2{]}):

\begin{figure}
\centering
\pandocbounded{\includegraphics[keepaspectratio]{images/Nazwa Yuan Adelia Putri_23030630095_EMT4Plot2D-106.png}}
\caption{images/Nazwa\%20Yuan\%20Adelia\%20Putri\_23030630095\_EMT4Plot2D-106.png}
\end{figure}

\textgreater columnsplot(cumsum(random(10)),style=``/'',color=blue):

\begin{figure}
\centering
\pandocbounded{\includegraphics[keepaspectratio]{images/Nazwa Yuan Adelia Putri_23030630095_EMT4Plot2D-107.png}}
\caption{images/Nazwa\%20Yuan\%20Adelia\%20Putri\_23030630095\_EMT4Plot2D-107.png}
\end{figure}

Itu juga dapat menampilkan string sebagai label.

\textgreater months={[}``Jan'',``Feb'',``Mar'',``Apr'',``May'',``Jun'', \ldots{}\\
\textgreater{} ``Jul'',``Aug'',``Sep'',``Oct'',``Nov'',``Dec''{]};

\textgreater values={[}10,12,12,18,22,28,30,26,22,18,12,8{]};

\textgreater columnsplot(values,lab=months,color=red,style=``-'');

\textgreater title(``Temperature''):

\begin{figure}
\centering
\pandocbounded{\includegraphics[keepaspectratio]{images/Nazwa Yuan Adelia Putri_23030630095_EMT4Plot2D-108.png}}
\caption{images/Nazwa\%20Yuan\%20Adelia\%20Putri\_23030630095\_EMT4Plot2D-108.png}
\end{figure}

\textgreater k=0:10;

\textgreater plot2d(k,bin(10,k),\textgreater bar):

\begin{figure}
\centering
\pandocbounded{\includegraphics[keepaspectratio]{images/Nazwa Yuan Adelia Putri_23030630095_EMT4Plot2D-109.png}}
\caption{images/Nazwa\%20Yuan\%20Adelia\%20Putri\_23030630095\_EMT4Plot2D-109.png}
\end{figure}

\textgreater plot2d(k,bin(10,k)); plot2d(k,bin(10,k),\textgreater points,\textgreater add):

\begin{figure}
\centering
\pandocbounded{\includegraphics[keepaspectratio]{images/Nazwa Yuan Adelia Putri_23030630095_EMT4Plot2D-110.png}}
\caption{images/Nazwa\%20Yuan\%20Adelia\%20Putri\_23030630095\_EMT4Plot2D-110.png}
\end{figure}

\textgreater plot2d(normal(1000),normal(1000),\textgreater points,grid=6,style=``..''):

\begin{figure}
\centering
\pandocbounded{\includegraphics[keepaspectratio]{images/Nazwa Yuan Adelia Putri_23030630095_EMT4Plot2D-111.png}}
\caption{images/Nazwa\%20Yuan\%20Adelia\%20Putri\_23030630095\_EMT4Plot2D-111.png}
\end{figure}

\textgreater plot2d(normal(1,1000),\textgreater distribution,style=``O''):

\begin{figure}
\centering
\pandocbounded{\includegraphics[keepaspectratio]{images/Nazwa Yuan Adelia Putri_23030630095_EMT4Plot2D-112.png}}
\caption{images/Nazwa\%20Yuan\%20Adelia\%20Putri\_23030630095\_EMT4Plot2D-112.png}
\end{figure}

\textgreater plot2d(``qnormal'',0,5;2.5,0.5,\textgreater filled):

\begin{figure}
\centering
\pandocbounded{\includegraphics[keepaspectratio]{images/Nazwa Yuan Adelia Putri_23030630095_EMT4Plot2D-113.png}}
\caption{images/Nazwa\%20Yuan\%20Adelia\%20Putri\_23030630095\_EMT4Plot2D-113.png}
\end{figure}

Untuk memplot distribusi statistik eksperimental, Anda dapat menggunakan distribution=n dengan plot2d.

\textgreater w=randexponential(1,1000); // exponential distribution

\textgreater plot2d(w,\textgreater distribution): // or distribution=n with n intervals

\begin{figure}
\centering
\pandocbounded{\includegraphics[keepaspectratio]{images/Nazwa Yuan Adelia Putri_23030630095_EMT4Plot2D-114.png}}
\caption{images/Nazwa\%20Yuan\%20Adelia\%20Putri\_23030630095\_EMT4Plot2D-114.png}
\end{figure}

Atau Anda dapat menghitung distribusi dari data dan memplot hasilnya dengan \textgreater bar di plot3d, atau dengan plot kolom.

\textgreater w=normal(1000); // 0-1-normal distribution

\textgreater\{x,y\}=histo(w,10,v={[}-6,-4,-2,-1,0,1,2,4,6{]}); // interval bounds v

\textgreater plot2d(x,y,\textgreater bar):

\begin{figure}
\centering
\pandocbounded{\includegraphics[keepaspectratio]{images/Nazwa Yuan Adelia Putri_23030630095_EMT4Plot2D-115.png}}
\caption{images/Nazwa\%20Yuan\%20Adelia\%20Putri\_23030630095\_EMT4Plot2D-115.png}
\end{figure}

Fungsi statplot() menyetel gaya dengan string sederhana.

\textgreater statplot(1:10,cumsum(random(10)),``b''):

\begin{figure}
\centering
\pandocbounded{\includegraphics[keepaspectratio]{images/Nazwa Yuan Adelia Putri_23030630095_EMT4Plot2D-116.png}}
\caption{images/Nazwa\%20Yuan\%20Adelia\%20Putri\_23030630095\_EMT4Plot2D-116.png}
\end{figure}

\textgreater n=10; i=0:n; \ldots{}\\
\textgreater{} plot2d(i,bin(n,i)/2\^{}n,a=0,b=10,c=0,d=0.3); \ldots{}\\
\textgreater{} plot2d(i,bin(n,i)/2\^{}n,points=true,style=``ow'',add=true,color=blue):

\begin{figure}
\centering
\pandocbounded{\includegraphics[keepaspectratio]{images/Nazwa Yuan Adelia Putri_23030630095_EMT4Plot2D-117.png}}
\caption{images/Nazwa\%20Yuan\%20Adelia\%20Putri\_23030630095\_EMT4Plot2D-117.png}
\end{figure}

Selain itu, data dapat diplot sebagai batang. Dalam hal ini, x harus diurutkan dan satu elemen lebih panjang dari y. Bilah akan memanjang dari x{[}i{]} ke x{[}i+1{]} dengan nilai y{[}i{]}. Jika x memiliki ukuran yang sama dengan y, maka akan diperpanjang satu elemen dengan spasi terakhir.

Gaya isian dapat digunakan seperti di atas.

\textgreater n=10; k=bin(n,0:n); \ldots{}\\
\textgreater{} plot2d(-0.5:n+0.5,k,bar=true,fillcolor=lightgray):

\begin{figure}
\centering
\pandocbounded{\includegraphics[keepaspectratio]{images/Nazwa Yuan Adelia Putri_23030630095_EMT4Plot2D-118.png}}
\caption{images/Nazwa\%20Yuan\%20Adelia\%20Putri\_23030630095\_EMT4Plot2D-118.png}
\end{figure}

Data untuk plot batang (bar=1) dan histogram (histogram=1) dapat dinyatakan secara eksplisit dalam xv dan yv, atau dapat dihitung dari distribusi empiris dalam xv dengan \textgreater distribusi (atau distribusi=n). Histogram nilai xv akan dihitung secara otomatis dengan \textgreater histogram. Jika \textgreater genap ditentukan, nilai xv akan dihitung dalam interval bilangan bulat.

\textgreater plot2d(normal(10000),distribution=50):

\begin{figure}
\centering
\pandocbounded{\includegraphics[keepaspectratio]{images/Nazwa Yuan Adelia Putri_23030630095_EMT4Plot2D-119.png}}
\caption{images/Nazwa\%20Yuan\%20Adelia\%20Putri\_23030630095\_EMT4Plot2D-119.png}
\end{figure}

\textgreater k=0:10; m=bin(10,k); x=(0:11)-0.5; plot2d(x,m,\textgreater bar):

\begin{figure}
\centering
\pandocbounded{\includegraphics[keepaspectratio]{images/Nazwa Yuan Adelia Putri_23030630095_EMT4Plot2D-120.png}}
\caption{images/Nazwa\%20Yuan\%20Adelia\%20Putri\_23030630095\_EMT4Plot2D-120.png}
\end{figure}

\textgreater columnsplot(m,k):

\begin{figure}
\centering
\pandocbounded{\includegraphics[keepaspectratio]{images/Nazwa Yuan Adelia Putri_23030630095_EMT4Plot2D-121.png}}
\caption{images/Nazwa\%20Yuan\%20Adelia\%20Putri\_23030630095\_EMT4Plot2D-121.png}
\end{figure}

\textgreater plot2d(random(600)*6,histogram=6):

\begin{figure}
\centering
\pandocbounded{\includegraphics[keepaspectratio]{images/Nazwa Yuan Adelia Putri_23030630095_EMT4Plot2D-122.png}}
\caption{images/Nazwa\%20Yuan\%20Adelia\%20Putri\_23030630095\_EMT4Plot2D-122.png}
\end{figure}

Untuk distribusi, ada parameter distribusi=n, yang menghitung nilai secara otomatis dan mencetak distribusi relatif dengan n sub-interval.

\textgreater plot2d(normal(1,1000),distribution=10,style=``\textbackslash/''):

\begin{figure}
\centering
\pandocbounded{\includegraphics[keepaspectratio]{images/Nazwa Yuan Adelia Putri_23030630095_EMT4Plot2D-123.png}}
\caption{images/Nazwa\%20Yuan\%20Adelia\%20Putri\_23030630095\_EMT4Plot2D-123.png}
\end{figure}

Dengan parameter even=true, ini akan menggunakan interval integer.

\textgreater plot2d(intrandom(1,1000,10),distribution=10,even=true):

\begin{figure}
\centering
\pandocbounded{\includegraphics[keepaspectratio]{images/Nazwa Yuan Adelia Putri_23030630095_EMT4Plot2D-124.png}}
\caption{images/Nazwa\%20Yuan\%20Adelia\%20Putri\_23030630095\_EMT4Plot2D-124.png}
\end{figure}

Perhatikan bahwa ada banyak plot statistik, yang mungkin berguna. Silahkan lihat tutorial tentang statistik.

\textgreater columnsplot(getmultiplicities(1:6,intrandom(1,6000,6))):

\begin{figure}
\centering
\pandocbounded{\includegraphics[keepaspectratio]{images/Nazwa Yuan Adelia Putri_23030630095_EMT4Plot2D-125.png}}
\caption{images/Nazwa\%20Yuan\%20Adelia\%20Putri\_23030630095\_EMT4Plot2D-125.png}
\end{figure}

\textgreater plot2d(normal(1,1000),\textgreater distribution); \ldots{}\\
\textgreater{} plot2d(``qnormal(x)'',color=red,thickness=2,\textgreater add):

\begin{figure}
\centering
\pandocbounded{\includegraphics[keepaspectratio]{images/Nazwa Yuan Adelia Putri_23030630095_EMT4Plot2D-126.png}}
\caption{images/Nazwa\%20Yuan\%20Adelia\%20Putri\_23030630095\_EMT4Plot2D-126.png}
\end{figure}

Ada juga banyak plot khusus untuk statistik. Sebuah boxplot menunjukkan kuartil dari distribusi ini dan banyak dari

outlier. Menurut definisi, outlier dalam boxplot adalah data yang melebihi 1,5 kali kisaran 50\% tengah plot.

\textgreater M=normal(5,1000); boxplot(quartiles(M)):

\begin{figure}
\centering
\pandocbounded{\includegraphics[keepaspectratio]{images/Nazwa Yuan Adelia Putri_23030630095_EMT4Plot2D-127.png}}
\caption{images/Nazwa\%20Yuan\%20Adelia\%20Putri\_23030630095\_EMT4Plot2D-127.png}
\end{figure}

\chapter{Fungsi Implisit}\label{fungsi-implisit}

Plot implisit menunjukkan garis level yang menyelesaikan f(x,y)=level, di mana ``level'' dapat berupa nilai tunggal atau vektor nilai. Jika level=``auto'', akan ada garis level nc, yang akan menyebar antara fungsi minimum dan maksimum secara merata. Warna yang lebih gelap atau lebih terang dapat ditambahkan dengan \textgreater hue untuk menunjukkan nilai fungsi. Untuk fungsi implisit, xv harus berupa fungsi atau ekspresi dari parameter x dan y, atau, sebagai alternatif, xv dapat berupa matriks nilai.

Euler dapat menandai garis level

dari fungsi apapun.

Untuk menggambar himpunan f(x,y)=c untuk satu atau lebih konstanta c, Anda dapat menggunakan plot2d() dengan plot implisitnya di dalam bidang. Parameter untuk c adalah level=c, di mana c dapat berupa vektor garis level. Selain itu, skema warna dapat digambar di latar belakang untuk menunjukkan nilai fungsi untuk setiap titik dalam plot. Parameter ``n'' menentukan kehalusan plot.

\textgreater aspect(1.5);

\textgreater plot2d(``x\textsuperscript{2+y}2-x*y-x'',r=1.5,level=0,contourcolor=red):

\begin{figure}
\centering
\pandocbounded{\includegraphics[keepaspectratio]{images/Nazwa Yuan Adelia Putri_23030630095_EMT4Plot2D-128.png}}
\caption{images/Nazwa\%20Yuan\%20Adelia\%20Putri\_23030630095\_EMT4Plot2D-128.png}
\end{figure}

\textgreater expr := ``2*x\textsuperscript{2+x*y+3*y}4+y''; // define an expression f(x,y)

\textgreater plot2d(expr,level=0): // Solutions of f(x,y)=0

\begin{figure}
\centering
\pandocbounded{\includegraphics[keepaspectratio]{images/Nazwa Yuan Adelia Putri_23030630095_EMT4Plot2D-129.png}}
\caption{images/Nazwa\%20Yuan\%20Adelia\%20Putri\_23030630095\_EMT4Plot2D-129.png}
\end{figure}

\textgreater plot2d(expr,level=0:0.5:20,\textgreater hue,contourcolor=white,n=200): // nice

\begin{figure}
\centering
\pandocbounded{\includegraphics[keepaspectratio]{images/Nazwa Yuan Adelia Putri_23030630095_EMT4Plot2D-130.png}}
\caption{images/Nazwa\%20Yuan\%20Adelia\%20Putri\_23030630095\_EMT4Plot2D-130.png}
\end{figure}

\textgreater plot2d(expr,level=0:0.5:20,\textgreater hue,\textgreater spectral,n=200,grid=4): // nicer

\begin{figure}
\centering
\pandocbounded{\includegraphics[keepaspectratio]{images/Nazwa Yuan Adelia Putri_23030630095_EMT4Plot2D-131.png}}
\caption{images/Nazwa\%20Yuan\%20Adelia\%20Putri\_23030630095\_EMT4Plot2D-131.png}
\end{figure}

Ini berfungsi untuk plot data juga. Tetapi Anda harus menentukan rentangnya untuk label sumbu.

\textgreater x=-2:0.05:1; y=x'; z=expr(x,y);

\textgreater plot2d(z,level=0,a=-1,b=2,c=-2,d=1,\textgreater hue):

\begin{figure}
\centering
\pandocbounded{\includegraphics[keepaspectratio]{images/Nazwa Yuan Adelia Putri_23030630095_EMT4Plot2D-132.png}}
\caption{images/Nazwa\%20Yuan\%20Adelia\%20Putri\_23030630095\_EMT4Plot2D-132.png}
\end{figure}

\textgreater plot2d(``x\textsuperscript{3-y}2'',\textgreater contour,\textgreater hue,\textgreater spectral):

\begin{figure}
\centering
\pandocbounded{\includegraphics[keepaspectratio]{images/Nazwa Yuan Adelia Putri_23030630095_EMT4Plot2D-133.png}}
\caption{images/Nazwa\%20Yuan\%20Adelia\%20Putri\_23030630095\_EMT4Plot2D-133.png}
\end{figure}

\textgreater plot2d(``x\textsuperscript{3-y}2'',level=0,contourwidth=3,\textgreater add,contourcolor=red):

\begin{figure}
\centering
\pandocbounded{\includegraphics[keepaspectratio]{images/Nazwa Yuan Adelia Putri_23030630095_EMT4Plot2D-134.png}}
\caption{images/Nazwa\%20Yuan\%20Adelia\%20Putri\_23030630095\_EMT4Plot2D-134.png}
\end{figure}

\textgreater z=z+normal(size(z))*0.2;

\textgreater plot2d(z,level=0.5,a=-1,b=2,c=-2,d=1):

\begin{figure}
\centering
\pandocbounded{\includegraphics[keepaspectratio]{images/Nazwa Yuan Adelia Putri_23030630095_EMT4Plot2D-135.png}}
\caption{images/Nazwa\%20Yuan\%20Adelia\%20Putri\_23030630095\_EMT4Plot2D-135.png}
\end{figure}

\textgreater plot2d(expr,level={[}0:0.2:5;0.05:0.2:5.05{]},color=lightgray):

\begin{figure}
\centering
\pandocbounded{\includegraphics[keepaspectratio]{images/Nazwa Yuan Adelia Putri_23030630095_EMT4Plot2D-136.png}}
\caption{images/Nazwa\%20Yuan\%20Adelia\%20Putri\_23030630095\_EMT4Plot2D-136.png}
\end{figure}

\textgreater plot2d(``x\textsuperscript{2+y}3+x*y'',level=1,r=4,n=100):

\begin{figure}
\centering
\pandocbounded{\includegraphics[keepaspectratio]{images/Nazwa Yuan Adelia Putri_23030630095_EMT4Plot2D-137.png}}
\caption{images/Nazwa\%20Yuan\%20Adelia\%20Putri\_23030630095\_EMT4Plot2D-137.png}
\end{figure}

\textgreater plot2d(``x\textsuperscript{2+2*y}2-x*y'',level=0:0.1:10,n=100,contourcolor=white,\textgreater hue):

\begin{figure}
\centering
\pandocbounded{\includegraphics[keepaspectratio]{images/Nazwa Yuan Adelia Putri_23030630095_EMT4Plot2D-138.png}}
\caption{images/Nazwa\%20Yuan\%20Adelia\%20Putri\_23030630095\_EMT4Plot2D-138.png}
\end{figure}

Juga dimungkinkan untuk mengisi set

dengan rentang tingkat.

Dimungkinkan untuk mengisi wilayah nilai untuk fungsi tertentu. Untuk ini, level harus berupa matriks 2xn. Baris pertama adalah batas bawah dan baris kedua berisi batas atas.

\textgreater plot2d(expr,level={[}0;1{]},style=``-'',color=blue): // 0 \textless= f(x,y) \textless= 1

\begin{figure}
\centering
\pandocbounded{\includegraphics[keepaspectratio]{images/Nazwa Yuan Adelia Putri_23030630095_EMT4Plot2D-139.png}}
\caption{images/Nazwa\%20Yuan\%20Adelia\%20Putri\_23030630095\_EMT4Plot2D-139.png}
\end{figure}

Plot implisit juga dapat menunjukkan rentang level. Kemudian level harus berupa matriks 2xn dari interval level, di mana baris pertama berisi awal dan baris kedua adalah akhir dari setiap interval. Atau, vektor baris sederhana dapat digunakan untuk level, dan parameter dl memperluas nilai level ke interval.

\textgreater plot2d(``x\textsuperscript{4+y}4'',r=1.5,level={[}0;1{]},color=blue,style=``/''):

\begin{figure}
\centering
\pandocbounded{\includegraphics[keepaspectratio]{images/Nazwa Yuan Adelia Putri_23030630095_EMT4Plot2D-140.png}}
\caption{images/Nazwa\%20Yuan\%20Adelia\%20Putri\_23030630095\_EMT4Plot2D-140.png}
\end{figure}

\textgreater plot2d(``x\textsuperscript{2+y}3+x*y'',level={[}0,2,4;1,3,5{]},style=``/'',r=2,n=100):

\begin{figure}
\centering
\pandocbounded{\includegraphics[keepaspectratio]{images/Nazwa Yuan Adelia Putri_23030630095_EMT4Plot2D-141.png}}
\caption{images/Nazwa\%20Yuan\%20Adelia\%20Putri\_23030630095\_EMT4Plot2D-141.png}
\end{figure}

\textgreater plot2d(``x\textsuperscript{2+y}3+x*y'',level=-10:20,r=2,style=``-'',dl=0.1,n=100):

\begin{figure}
\centering
\pandocbounded{\includegraphics[keepaspectratio]{images/Nazwa Yuan Adelia Putri_23030630095_EMT4Plot2D-142.png}}
\caption{images/Nazwa\%20Yuan\%20Adelia\%20Putri\_23030630095\_EMT4Plot2D-142.png}
\end{figure}

\textgreater plot2d(``sin(x)*cos(y)'',r=pi,\textgreater hue,\textgreater levels,n=100):

\begin{figure}
\centering
\pandocbounded{\includegraphics[keepaspectratio]{images/Nazwa Yuan Adelia Putri_23030630095_EMT4Plot2D-143.png}}
\caption{images/Nazwa\%20Yuan\%20Adelia\%20Putri\_23030630095\_EMT4Plot2D-143.png}
\end{figure}

Dimungkinkan juga untuk menandai suatu wilayah

\[a \le f(x,y) \le b.\]Ini dilakukan dengan menambahkan level dengan dua baris.

\textgreater plot2d(``(x\textsuperscript{2+y}2-1)\textsuperscript{3-x}2*y\^{}3'',r=1.3, \ldots{}\\
\textgreater{} style=``\#'',color=red,\textless outline, \ldots{}\\
\textgreater{} level={[}-2;0{]},n=100):

\begin{figure}
\centering
\pandocbounded{\includegraphics[keepaspectratio]{images/Nazwa Yuan Adelia Putri_23030630095_EMT4Plot2D-145.png}}
\caption{images/Nazwa\%20Yuan\%20Adelia\%20Putri\_23030630095\_EMT4Plot2D-145.png}
\end{figure}

Dimungkinkan untuk menentukan level tertentu. Misalnya, kita dapat memplot solusi persamaan seperti

\textgreater plot2d(``x\textsuperscript{3-x*y+x}2*y\^{}2'',r=6,level=1,n=100):

\begin{figure}
\centering
\pandocbounded{\includegraphics[keepaspectratio]{images/Nazwa Yuan Adelia Putri_23030630095_EMT4Plot2D-146.png}}
\caption{images/Nazwa\%20Yuan\%20Adelia\%20Putri\_23030630095\_EMT4Plot2D-146.png}
\end{figure}

\textgreater function starplot1 (v, style=``/'', color=green, lab=none) \ldots{}

\begin{verbatim}
  if !holding() then clg; endif;
  w=window(); window(0,0,1024,1024);
  h=holding(1);
  r=max(abs(v))*1.2;
  setplot(-r,r,-r,r);
  n=cols(v); t=linspace(0,2pi,n);
  v=v|v[1]; c=v*cos(t); s=v*sin(t);
  cl=barcolor(color); st=barstyle(style);
  loop 1 to n
    polygon([0,c[#],c[#+1]],[0,s[#],s[#+1]],1);
    if lab!=none then
      rlab=v[#]+r*0.1;
      {col,row}=toscreen(cos(t[#])*rlab,sin(t[#])*rlab);
      ctext(""+lab[#],col,row-textheight()/2);
    endif;
  end;
  barcolor(cl); barstyle(st);
  holding(h);
  window(w);
endfunction
\end{verbatim}

Tidak ada kotak atau sumbu kutu di sini. Selain itu, kami menggunakan jendela penuh untuk plot.

Kami memanggil reset sebelum kami menguji plot ini untuk mengembalikan default grafis. Ini tidak perlu, jika Anda yakin plot Anda berhasil.

\textgreater reset; starplot1(normal(1,10)+5,color=red,lab=1:10):

\begin{figure}
\centering
\pandocbounded{\includegraphics[keepaspectratio]{images/Nazwa Yuan Adelia Putri_23030630095_EMT4Plot2D-147.png}}
\caption{images/Nazwa\%20Yuan\%20Adelia\%20Putri\_23030630095\_EMT4Plot2D-147.png}
\end{figure}

Terkadang, Anda mungkin ingin merencanakan sesuatu yang tidak dapat dilakukan plot2d, tetapi hampir.

Dalam fungsi berikut, kami melakukan plot impuls logaritmik. plot2d dapat melakukan plot logaritmik, tetapi tidak untuk batang impuls.

\textgreater function logimpulseplot1 (x,y) \ldots{}

\begin{verbatim}
  {x0,y0}=makeimpulse(x,log(y)/log(10));
  plot2d(x0,y0,>bar,grid=0);
  h=holding(1);
  frame();
  xgrid(ticks(x));
  p=plot();
  for i=-10 to 10;
    if i<=p[4] and i>=p[3] then
       ygrid(i,yt="10^"+i);
    endif;
  end;
  holding(h);
endfunction
\end{verbatim}

Mari kita uji dengan nilai yang terdistribusi secara eksponensial.

\textgreater aspect(1.5); x=1:10; y=-log(random(size(x)))*200; \ldots{}\\
\textgreater{} logimpulseplot1(x,y):

\begin{figure}
\centering
\pandocbounded{\includegraphics[keepaspectratio]{images/Nazwa Yuan Adelia Putri_23030630095_EMT4Plot2D-148.png}}
\caption{images/Nazwa\%20Yuan\%20Adelia\%20Putri\_23030630095\_EMT4Plot2D-148.png}
\end{figure}

Mari kita menganimasikan kurva 2D menggunakan plot langsung. Perintah plot(x,y) hanya memplot kurva ke jendela plot. setplot(a,b,c,d) mengatur jendela ini.

Fungsi wait(0) memaksa plot untuk muncul di jendela grafik. Jika tidak, menggambar ulang terjadi dalam interval waktu yang jarang.

\textgreater function animliss (n,m) \ldots{}

\begin{verbatim}
t=linspace(0,2pi,500);
f=0;
c=framecolor(0);
l=linewidth(2);
setplot(-1,1,-1,1);
repeat
  clg;
  plot(sin(n*t),cos(m*t+f));
  wait(0);
  if testkey() then break; endif;
  f=f+0.02;
end;
framecolor(c);
linewidth(l);
endfunction
\end{verbatim}

Tekan sembarang tombol untuk menghentikan animasi ini.

\textgreater animliss(2,3); // lihat hasilnya, jika sudah puas, tekan ENTER

\chapter{Plot Logaritmik}\label{plot-logaritmik}

EMT menggunakan parameter ``logplot'' untuk skala logaritmik.

Plot logaritma dapat diplot baik menggunakan skala logaritma dalam y dengan logplot=1, atau menggunakan skala logaritma dalam x dan y dengan logplot=2, atau dalam x dengan logplot=3.

\begin{itemize}
\tightlist
\item
  logplot=1: y-logaritma\\
\item
  logplot=2: x-y-logaritma\\
\item
  logplot=3: x-logaritma
\end{itemize}

\textgreater plot2d(``exp(x\textsuperscript{3-x)*x}2'',1,5,logplot=1):

\begin{figure}
\centering
\pandocbounded{\includegraphics[keepaspectratio]{images/Nazwa Yuan Adelia Putri_23030630095_EMT4Plot2D-149.png}}
\caption{images/Nazwa\%20Yuan\%20Adelia\%20Putri\_23030630095\_EMT4Plot2D-149.png}
\end{figure}

\textgreater plot2d(``exp(x+sin(x))'',0,100,logplot=1):

\begin{figure}
\centering
\pandocbounded{\includegraphics[keepaspectratio]{images/Nazwa Yuan Adelia Putri_23030630095_EMT4Plot2D-150.png}}
\caption{images/Nazwa\%20Yuan\%20Adelia\%20Putri\_23030630095\_EMT4Plot2D-150.png}
\end{figure}

\textgreater plot2d(``exp(x+sin(x))'',10,100,logplot=2):

\begin{figure}
\centering
\pandocbounded{\includegraphics[keepaspectratio]{images/Nazwa Yuan Adelia Putri_23030630095_EMT4Plot2D-151.png}}
\caption{images/Nazwa\%20Yuan\%20Adelia\%20Putri\_23030630095\_EMT4Plot2D-151.png}
\end{figure}

\textgreater plot2d(``gamma(x)'',1,10,logplot=1):

\begin{figure}
\centering
\pandocbounded{\includegraphics[keepaspectratio]{images/Nazwa Yuan Adelia Putri_23030630095_EMT4Plot2D-152.png}}
\caption{images/Nazwa\%20Yuan\%20Adelia\%20Putri\_23030630095\_EMT4Plot2D-152.png}
\end{figure}

\textgreater plot2d(``log(x*(2+sin(x/100)))'',10,1000,logplot=3):

\begin{figure}
\centering
\pandocbounded{\includegraphics[keepaspectratio]{images/Nazwa Yuan Adelia Putri_23030630095_EMT4Plot2D-153.png}}
\caption{images/Nazwa\%20Yuan\%20Adelia\%20Putri\_23030630095\_EMT4Plot2D-153.png}
\end{figure}

Ini juga berfungsi dengan plot data.

\textgreater x=10\^{}(1:20); y=x\^{}2-x;

\textgreater plot2d(x,y,logplot=2):

\begin{figure}
\centering
\pandocbounded{\includegraphics[keepaspectratio]{images/Nazwa Yuan Adelia Putri_23030630095_EMT4Plot2D-154.png}}
\caption{images/Nazwa\%20Yuan\%20Adelia\%20Putri\_23030630095\_EMT4Plot2D-154.png}
\end{figure}

\backmatter
\end{document}
